
\begin{frame}{Determination of the photon PDF}
  \begin{columns}[T]
    \begin{column}{0.59\textwidth}
      Initially the photon PDF has been determined in different ways:
      \begin{itemize}
        \item physical model: sensitive to underlying model
        \item fitting: data does not provide strong constraints
      \end{itemize}

      \vspace*{0.5em}
      However with the LUXqed approach it can be computed perturbatively \\
      based on the observation that the heavy-lepton production cross-section can be written in two ways:
      \begin{itemize}
        \item in terms of structure functions $F_2$, $F_L$
        \item in terms of PDFs (including the photon)
      \end{itemize}

      \vspace*{0.5em}
      luxQED result {\color{gray}\small[Manohar, Nason, Salam, Zanderighi: 1607.04266, 1708.01256]}:
      \vspace*{-0.8em}
      \begin{equation*}
        \begin{split}
          & x \gamma(x, \mu^2)
          =
          \frac{2}{\alpha (\mu^2)} \int\limits_x^1 \frac{dz}{z}
          \Biggl\{ \int_{m_p^2x^2 \over 1-z}^{\mu^2 \over 1-z} \frac{dQ^2}{Q^2}
          \alpha^2(Q^2) \Biggl[ -z^2 F_L(x/z, Q^2) \\
          & + \left( z P_{\gamma q}(z) + \frac{2 x^2 m_p^2}{Q^2} \right)
          F_2(x/z, Q^2)\Biggr] - \alpha^2(\mu^2) z^2 F_2(x/z, \mu^2)\Biggr\}
        \end{split}
      \end{equation*}
    \end{column}

    \begin{column}{0.39\textwidth}
      \vspace*{-2.5em}
      \begin{figure}
        \includegraphics[width=0.89\textwidth]{figures/dataluxqed.png}
        \caption*{Input to construct $F_2$ and $F_L$}
        \includegraphics[width=0.89\textwidth]{figures/luxQED_uncs.png}
        \caption*{Sources of uncertainty}
      \end{figure}
    \end{column}
  \end{columns}
\end{frame}


\begin{frame}{LUXqed PDF determinations}
  LUXqed has been used in all of the most recent QED PDFs:
  \begin{itemize}
      \item LUXqed\_plus\_PDF4LHC15 {\color{gray}\small [1607.04266]}
      \item LUXqed17\_plus\_PDF4LHC15 {\color{gray}\small [1708.01256]}
      \item MMHT2015qed {\color{gray}\small [1907.02750]}
      \item NNPDF3.1luxQED {\color{gray}\small [1712.07053]}
      \item CT18lux and CT18qed {\color{gray}\small [2106.10299]}
      \item MSHT20QED {\color{gray}\small [2111.05357]}
      \item MSHT20qed\_an3lo {\color{gray}\small [2312.07665]}
      \item NNPDF4.0QED {\color{gray}\small [2401.08749 ]}
  \end{itemize}
\end{frame}

% \begin{frame}{Results: photon PDF and luminosity}
%   \begin{center}
%     \includegraphics[width=0.3\textwidth]{figures/photon_comparison.pdf}
%     \includegraphics[width=0.3\textwidth]{figures/pp_lumi_comparison.png}
%     \includegraphics[width=0.3\textwidth]{figures/gp_lumi_comparison.pdf}
%   \end{center}
%   \begin{itemize}
%     \item Because all groups use the luxQED formalism, the photon PDFs agree at percent level
%     \item Luminosity generally in agreement, but differ at very small and very large invariant mass
%   \end{itemize}
% \end{frame}


% ============================================================================


\begin{frame}{Incomplete higher order uncertainties covmat}
  \begin{itemize}
    \item We construct an IHOU matrix following a similar approach by varying the subleading functions
    \item IHOU are independent of MHOU so the uncertainties are added in quadrature
    $$C = C_\mathrm{exp}+C_\mathrm{MHOU}+C_\mathrm{IHOU}$$
  \end{itemize}

  \begin{columns}
    \begin{column}{0.49\textwidth}
      \begin{figure}[!t]
        \centering
        \includegraphics[width=.9\textwidth]{figures/diag_cov_dis_ihou.pdf}
        \caption*{IHOU have a large effect on small-$x$, low-$Q$ DIS data
        }
      \end{figure}
    \end{column}
    \begin{column}{0.49\textwidth}
      \begin{figure}[!t]
        \centering
        \includegraphics[width=.9\textwidth]{figures/diag_cov_dy_ihou_3pt_mhou.pdf}
        \caption*{NNLO MHOU included where N3LO not available \\
          MHOU can similar magnitude as the experimental uncertainty
        }
      \end{figure}
    \end{column}
  \end{columns}


\end{frame}

% \begin{frame}{Magnitude of theory uncertainties}
% % show that for certain processes th unc is of same size as exp unc.
% \end{frame}

% ============================================================================

\begin{frame}{Impact of MHOUs at N3LO}
  \begin{figure}[!t]
    \centering
    \includegraphics[width=0.45\textwidth]{figures/gg_plot_lumi1d.pdf}
    \includegraphics[width=0.45\textwidth]{figures/qqbar_plot_lumi1d.pdf}
  \end{figure}
  \begin{itemize}
    \item Non-negligible impact of MHOUs even at N3LO
    \item[$\Rightarrow$] reason to include exact N3LO calculations for hadronic processes
  \end{itemize}
\end{frame}


% \begin{frame}{Comparison to MSHT20}
%   \begin{figure}[!t]
%     \centering
%     \includegraphics[width=0.45\textwidth]{figures/gg_plot_lumi1d_msht20.pdf}
%     \includegraphics[width=0.45\textwidth]{figures/qqbar_plot_lumi1d_msht20.pdf}
%   \end{figure}
%   \begin{itemize}
%     \item Good agreement with MSHT20 for the quark luminosities
%     \item Also for gluon luminosities, except around the Higgs mass and high-mass
%     \item Similar data but different methodology (including splitting function parametrization)
%   \end{itemize}
% \end{frame}

