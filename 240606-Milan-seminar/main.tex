\documentclass[8pt,t]{beamer}
\geometry{paperwidth=160mm,paperheight=0.75\paperwidth} % I can't make the fontsize smaller (than 8pt), but I can make the page bigger
\graphicspath{{figures/}} % Setting the graphicspath

% Theme settings
\usetheme{Madrid}
\usecolortheme{default}
\setbeamertemplate{navigation symbols}{}   % removes navigation symbols such as 'next page'
\setbeamertemplate{footline}{}             % remove line with name, date, page nr.
\setbeamercolor*{frametitle}{bg=white}     % remove background from frametitle
\usepackage{caption}
% \captionsetup[figure]{labelformat=empty}% redefines the caption setup of the figures environment in the beamer class.
\setbeamersize{text margin left=20pt,text margin right=10pt}
\usefonttheme[onlymath]{serif} % makes beamer math look like article math
\usepackage{hyperref}
\usepackage{tikz}
\usepackage{amsmath}

%======================= title page info =======================
\title{The proton content at approximate N3LO accuracy}
\date{Milan Joint Phenomenology Seminar  \\[0.1cm] 6 June 2024, Milan}
\author{Roy Stegeman}
\institute{\small The University of Edinburgh}


%======================= page numbering =======================
\addtobeamertemplate{navigation symbols}{}{%
  \ifnum\thepage>1% don't display frame number on the first slide
    \usebeamerfont{footline}\insertframenumber\hspace*{2em}\vspace*{2em}% display frame number
  \fi%
}



%=================================== colors ====================================
\definecolor{RoyBlue}{RGB}{22, 46, 69}
\definecolor{RoyGrey}{RGB}{64, 88, 128}

\setbeamercolor{structure}{fg=RoyBlue} % itemize, enumerate, etc
\setbeamercolor{frametitle}{fg=RoyGrey}
\setbeamercolor{section in head/foot}{bg=RoyBlue}


%======================= add progress dots to headline =========================
% \setbeamertemplate{headline}{%
%     \begin{beamercolorbox}[ht=4mm,dp=4mm]{section in head/foot}
%         \insertnavigation{\paperwidth}
%     \end{beamercolorbox}%
% }%
% \makeatother


%======================= add section title page ================================
\newcommand{\SectionTitleFrame}[1][]{%
  \begin{frame}
    \vfill
    \centering
    \begin{beamercolorbox}[sep=8pt,center,shadow=true,rounded=true]{title}
      \usebeamerfont{title}\insertsection\par
    \end{beamercolorbox}
    % Include optional text if provided
    \ifx\relax#1\relax\else
      \vspace{0.5cm}
      \textbf{#1}
    \fi
    \vfill
  \end{frame}
}

% Use \SectionTitleFrame in \AtBeginSection
\AtBeginSection[]{
  \SectionTitleFrame
}


%=================================== titlepage =================================
\titlegraphic{\vspace*{6mm}
  \includegraphics[height=1.5cm]{logos/edi_logo.png} \hspace{10mm}
  % \includegraphics[height=0.8cm]{logos/nnpdf_logo_official.pdf} \hspace{10mm}
  \includegraphics[height=1.5cm]{logos/higgs_logo.jpg}
}

\defbeamertemplate{title page}{noinstitute}[1][]
{
  \vbox{}
  \vfill
  \begingroup
    \centering
    \begin{beamercolorbox}[sep=8pt,center,#1]{title}
      \usebeamerfont{title}\inserttitle\par%
      \ifx\insertsubtitle\@empty%
      \else%
        \vskip0.25em%
        {\usebeamerfont{subtitle}\usebeamercolor[fg]{subtitle}\insertsubtitle\par}%
      \fi%
    \end{beamercolorbox}%
    \vskip2em\par
    \begin{beamercolorbox}[sep=0pt,center,#1]{author}
      \usebeamerfont{author}\insertauthor
    \end{beamercolorbox}
  \begin{beamercolorbox}[sep=0pt,center,#1]{author}
    \usebeamerfont{institute}\insertinstitute
  \end{beamercolorbox}
  \vspace*{8pt}
  \vspace*{16pt}
    \begin{beamercolorbox}[sep=0pt,center,#1]{date}
      \usebeamerfont{date}\insertdate
    \end{beamercolorbox}\vskip0.5em
    {\usebeamercolor[fg]{titlegraphic}\inserttitlegraphic\par}
  \endgroup
  \vfill
}

\makeatletter
\setbeamertemplate{title page}[noinstitute][colsep=-4bp,rounded=true,shadow=\beamer@themerounded@shadow]
\makeatother


\begin{document}
{
\setbeamertemplate{headline}{} % remove headline from titlepage
\begin{frame}
  \titlepage
\end{frame}
}

\setbeamertemplate{enumerate items}[default]

\pgfdeclarelayer{bg}    % declare background layer
\pgfsetlayers{bg,main}  % set the order of the layers (main is the standard layer)



% Title: The proton content at N3LO accuracy

% Abstract: In recent years the accuracy of PDF determinations has significantly
% improved due to a combination of experimental and theoretical developments. In
% this talk, I will discuss recent progress towards the extension of the PDF
% determination to approximate N3LO in QCD and NLO in QED, and their
% phenomenological implications for a number of LHC processes. I will also
% present a simultaneous extraction of the strong coupling constant and the PDFs
% based on the NNPDF4.0 dataset, taking correlations between them into account.
% We show how we can validate our determination of the strong coupling by
% performing a closure test.

% SLIDES =======================================================================


\begin{frame}{Motivation: theoretical uncertainties at the LHC}
  % $$\sigma(x,Q^2)=\sum_i \int_x^1 \frac{dz}{z} \mathcal{L}_{ij}(z,\mu^2)\hat{\sigma}_{ij}\left(\frac{x}{z},\frac{Q^2}{\mu^2},\alpha_s\right)$$

  \begin{columns}
    \begin{column}{0.55\textwidth}
      The dominant uncertainties in theoretical predictions at the LHC are:
      \begin{itemize}
        \item Missing higher order uncertainties (from scale variations)
        \item PDF uncertainties
        \item Uncertainties on $\alpha_s$
      \end{itemize}

      \vspace*{0.5em}
      Progress towards improved theoretical accuracy:
      \begin{itemize}
        \item QED effects
        \item approximate N3LO
        \item Accounting for missing higher order uncertainties
        \item A simultaneous determination of the PDFs and $\alpha_s$
      \end{itemize}
    \end{column}
    \begin{column}{0.44\textwidth}
      \begin{figure}
        \centering
        \includegraphics[width=0.99\textwidth]{figures/sources_of_unc_higgs.png}
        \caption*{ \small Uncertainties for inclusive Higgs production \\  {\color{gray}\footnotesize Dulat, Lazopoulos, Mistleberger, 1802.00827}}
      \end{figure}
    \end{column}
  \end{columns}
\end{frame}


\section*{approximate N3LO}
\SectionTitleFrame[\hyperlink{https://arxiv.org/abs/2402.18635}{arXiv: 2402.18635}]


\begin{frame}{Theory requirements for PDFs at N3LO}
  Several theory inputs are needed in a PDF fit:
  \begin{itemize}
    \item Splitting functions for DGLAP evolution
    \item Matching conditions for heavy-quark mass schemes \\
    $ f_i^{\left(n_f+1\right)}=A_{i j} f_j^{\left(n_f\right)} $

    \item DIS coefficient functions
    \item Hadronic cross sections,
  \end{itemize}

  \vspace*{2em}
  Not all available at N3LO, but information is available for all. What is the best we can do?
  \begin{itemize}
    \item Use N3LO calculations where known
    \item Construct approximate results where possible
    \item Account for theory uncertainties of the missing or incomplete higher order
  \end{itemize}

  \vspace*{1em}
  No need to wait for complete N3LO results and more information can be included as it becomes available
\end{frame}


\begin{frame}{Splitting functions}
  Complete results for the N3LO splitting functions are not yet available, but a lot of information exists (with important contributions from Liverpool and Edinburgh):
  \begin{itemize}
    \item Small-x limits (BFKL resummation) {\color{gray}\footnotesize [Bonvini and Marzani: 1805.06460] [Davies, Kom, Moch, Vogt, 2202.10362]}
    \item Large-x limits (threshold resummation) {\color{gray}\footnotesize [Soar, Moch, Vermaseren, Vogt: 0912.0369], [Henn, Korchemsky, Mistlberger, 1911.10174], [Duhr, Mistlberger, Vita 2205.04493]}
    \item Large-$n_f$ limit {\color{gray}\footnotesize [Davies, Ruijl, Ueda, Vermaseren, Vogt: 1610.0744], [Gehrmann, Manteuffel, Sotnikov, Yan, 2308.07958]}
    \item Mellin moments
    {\color{gray}\footnotesize [Falcioni, Herzog, Moch, Ruijl, Ueda, Vermaseren, Vogt, 1707.08315, 2111.15561, 2302.07593, 2307.04158]}
  \end{itemize}

  \vspace*{2em}
  \begin{center}
    How can we use this information to construct approximate splitting functions?
  \end{center}

\end{frame}


\begin{frame}{Splitting functions}

  \begin{enumerate}
    \item   The approximation is performed in Mellin space as an expansion in $n_f$, where any double counting terms present in the resummed small-$x$ and large-$x$ expressions are removed

    $$\gamma_{i j}^{(3)}=\gamma_{i j, n_f}^{(3)}+\gamma_{i j, N \rightarrow \infty}^{(3)}+\gamma_{i j, N \rightarrow 0}^{(3)}+\gamma_{i j, N \rightarrow 1}^{(3)}+\widetilde{\gamma}_{i j}^{(3)}$$

    \item   The remainder term $\widetilde{\gamma}_{i j}^{(3)}$ is constructed as a linear combination of interpolating functions:
    \begin{itemize}
      \item A function for the leading unknown large-N contribution
      \item A function for the two leading unknown small-N contribution
      \item Functions for the subleading small-N and large-N contributions
    \end{itemize}

    \item The weights of these interpolating functions are determined by equating to the known moments
    \item Then, vary the subleading contributions included in the basis of interpolating functions to estimate incomplete higher order uncertainties (IHOU) on the splitting functions
  \end{enumerate}

  \vspace*{1em}
  More details on how to account for IHOUs in a fit follows later
\end{frame}


\begin{frame}{Splitting functions}
  \begin{figure}
    \centering
    \includegraphics[width=.4\textwidth]{figures/gamma_gg_totu_logx.pdf}
    % \includegraphics[width=.4\textwidth]{figures/gamma_gq_totu_logx.pdf} \\
    \includegraphics[width=.4\textwidth]{figures/gamma_qg_totu_logx.pdf}
    % \includegraphics[width=.4\textwidth]{figures/gamma_qq_totu_logx.pdf}
  \end{figure}
  \begin{itemize}
    \item Dark blue band is IHOU only, light blue is sum in quadrature of MHOU and IHOU
    \item Good perturbative agreement at large-$x$
    \item IHOU are not negligible
  \end{itemize}
\end{frame}



\begin{frame}{DGLAP evolution}
  NNPDF4.0 evolved from $Q=1.65$ GeV to $Q=100$ GeV

  \begin{columns}
    \begin{column}{0.59\textwidth}
      \begin{figure}[!t]
        \centering
        \includegraphics[width=0.49\textwidth]{figures/N3LOevolution-q100gev-ratios_expanded_0.pdf}
        \includegraphics[width=0.49\textwidth]{figures/N3LOevolution-q100gev-ratios_expanded_1.pdf}\\
        \includegraphics[width=0.49\textwidth]{figures/N3LOevolution-q100gev-ratios_expanded_2.pdf}
        \includegraphics[width=0.49\textwidth]{figures/N3LOevolution-q100gev-ratios_expanded_3.pdf}
      \end{figure}
    \end{column}
    \begin{column}{0.39\textwidth}
      \begin{itemize}
        \item Effects of N3LO corrections to DGLAP evolution at most percent level, except at small-$x$ and large-$x$
        \item Good perturbative convergence
      \end{itemize}
    \end{column}
  \end{columns}
\end{frame}

\begin{frame}{DIS coefficient functions}
  \begin{itemize}
    \item DIS coefficient functions are known up to N3LO in the massless limit (again with contributions from Liverpool) {\color{gray}\small [Larin, Nogueira, Van Ritbergen, Vermaseren: 9605317], [Moch Vermaseren Vogt: 0411112, 0504242], [Davies, Moch, Vermaseren, Vogt: 0812.4168, 1606.08907]}
    \item Massive coefficient functions can be constructed by smoothly joining the known limits from high energy and threshold resummations and the massless limit ($Q^2 \rightarrow m_h^2$, $x\rightarrow 0$, and $Q^2\gg m_h^2$) {\color{gray}\small [Barontini, Bonvini, Laurenti: in preparation]}
  \end{itemize}

  \vspace*{-0.5em}
  \begin{columns}
    \begin{column}{0.49\textwidth}
      \begin{equation*}
        C^{(3)}(x,m_h^2/Q^2) = C^{(3),{\rm thr}}(x,m_h^2/Q^2) f_1(x) + C^{(3),{\rm asy}}(x,m_h^2/Q^2) f_2(x)
      \end{equation*}
      \begin{align*}
        \begin{split}
            f_1(x) \xrightarrow[x \to 0]{} 0, & \quad  f_1(x) \xrightarrow[x \to x_{\rm max}]{} 1 \, , \\
            f_2(x) \xrightarrow[x \to 0]{} 1, & \quad  f_1(x) \xrightarrow[x \to x_{\rm max}]{} 0 \, ,
        \end{split}
      \end{align*}
    \end{column}
    \begin{column}{0.49\textwidth}
      \vspace*{1.5em}
      \begin{figure}[!t]
        \centering
        \includegraphics[width=.7\textwidth]{figures/C2g_2_Q2m2_2.0.pdf}
        \caption*{We can validate the procedure at NNLO}
      \end{figure}
    \end{column}
  \end{columns}
\end{frame}


\begin{frame}{DIS variable flavor number scheme (VFNS)}
  \begin{columns}
    \begin{column}{0.59\textwidth}
      \begin{itemize}
        \item In a PDF fit different flavour number schemes are joined in a variable flavour number scheme (VFNS) to ensure reliable results from $Q^2\sim m_h^2$ up to $Q^2\gg m_h^2$
        \item The matching conditions encoding the transition between schemes have almost completely been computed up to N3LO
        \item The VFNS used in NNPDF is the FONLL scheme below
        \item FONLL extended for arbitrary number of mass scales in the recent \texttt{EKO} (DGLAP) and \texttt{yadism} (DIS) codes
      \end{itemize}
    \end{column}
    \begin{column}{0.39\textwidth}
      \vspace*{-2em}
      \begin{figure}[!t]
        \centering
        \includegraphics[width=0.89\textwidth]{figures/F2_charm_n3lo.pdf}
        \caption*{$F_2^{(c)}$}
      \end{figure}
    \end{column}
  \end{columns}

  \vspace*{1em}
  \begin{equation*}
    F_\mathrm{FONLL}(Q^2,m_c) =F^{(n_f+1)}(Q^2,m_h=0)
    +F^{(n_f)}(Q^2,m_c)-\lim_{m_c\rightarrow 0}F^{(n_f)}(Q^2,m_h)
  \end{equation*}
  % Up to NNLO, FONLL was implemented expressing the terms in the r.h.s in terms of $\alpha_s$ and PDFs defined in the massless scheme

  % FONLL is now implemented at ``observable level'' with simultaneous PDFs in different flavour number schemes, made possible thanks to the new \texttt{EKO} evolution code and \texttt{yadism} DIS library


\end{frame}




\begin{frame}{DIS structure functions}
  \begin{figure}[!t]
    \centering
    \includegraphics[width=0.75\textwidth]{figures/F2_total.pdf}
  \end{figure}
  \begin{itemize}
    \item The uncertainty band corresponds to IHOU of the massive coefficient functions
    \item N3LO corrections are significant at low-$Q$
  \end{itemize}
\end{frame}


\begin{frame}{Hadronic processes}
  \begin{itemize}
    \item Corrections to collider DY and $W$ production can be included through k-factors
    \item N3LO effects around 1 to 2\% for LHC observables
    \item For many processes N3LO corrections are not available, for those we introduce account for MHOU through $\mu_r$ variations
  \end{itemize}
  \begin{figure}[!t]
    \centering
    \includegraphics[width=.80\textwidth]{figures/kfactor_ATLASZHIGHMASS49FB.pdf}
  \end{figure}
\end{frame}

% \begin{frame}{Theory errors from scale variations}
%   \begin{itemize}
%     \item Missing higher order uncertainties are estimated through variations of the nonphysical factorization ($\mu_f$) and renormalization ($\mu_r$) scales
%     \item $\mu_r$ and $\mu_f$ are varied sumultaneously following the 7-point prescription
%   \end{itemize}
%   \begin{figure}
%     \includegraphics[width=.2\textwidth]{figures/5point.png}
%     \includegraphics[width=.2\textwidth]{figures/7point.png}
%     \includegraphics[width=.2\textwidth]{figures/9point.png}
%     \caption*{5,7,9 point prescription}
%   \end{figure}
%   \begin{itemize}
%     \item Factorization scale variations estimate MHOUs in DGLAP evolution
%     \item Renormalization scale variations estimate MHOUs in matrix elements
%   \end{itemize}
% \end{frame}

\begin{frame}{Missing higher order uncertainties covmat}
  How can we account for theory uncertainties in a PDF fit?

  \vspace*{1em}
  \begin{itemize}
    \item In a fit we minimize the $\chi^2$: \\
    $P(T|D) \propto \exp\left[-\frac{1}{2}\left(T-D\right)C_\mathrm{exp}^{-1}\left(T-D\right)\right] \propto \exp\left[-\frac{1}{2}\chi^2\right]$
    \item Include theory covmat $C_\mathrm{MHOU}$ at same footing as exp covmat $C_\mathrm{exp}$: $C_\mathrm{exp}\rightarrow C_\mathrm{exp}+C_\mathrm{MHOU}$ \\
    $$C_{\mathrm{MHOU},ij} = n_{m}\sum_{V_{m}}\left(T_{i}(\rho_f, \rho_r) - T_{i}(0, 0)\right)\left(T_{j}(\rho_f, \rho_r) - T_{j}(0, 0)\right)$$
  \end{itemize}

  \vspace*{1em}
  \begin{center}
    Can we trust the faithfulness of these uncertainties on the unknown order?
  \end{center}

\end{frame}


\begin{frame}{Missing higher order uncertainties covmat}
  Validate the MHOU procedure by testing the NLO MHOU covmat
  \begin{figure}[!t]
    \centering
      \includegraphics[width=0.8\textwidth]{figures/shift_validation.pdf}
  \end{figure}
\end{frame}


\begin{frame}{Fit quality}
  \begin{figure}[!t]
    \includegraphics[width=.4\textwidth]{figures/chi2_n3lo_summary.pdf}
  \end{figure}
  \begin{itemize}
    \item Without MHOUs the $\chi^2$ improves with the perturbative accuracy
    \item With MHOUs the $\chi^2$ stabilizes significantly
    \item At N3LO MHOUs have a small impact on the $\chi^2$
  \end{itemize}
\end{frame}

\begin{frame}{Perturbative convergence}
  \begin{figure}[!t]
    \includegraphics[width=.4\textwidth]{figures/gg_plot_lumi1d_convergence.pdf}
    \includegraphics[width=.4\textwidth]{figures/qqbar_plot_lumi1d_convergence.pdf}
  \end{figure}
  \begin{itemize}
    \item Good perturbative convergence
    \item Moderate impact of N3LO corrections, especially for the quark luminosities
    \item $\sim2\%$ suppression of $gg$ luminosity around the Higgs mass
  \end{itemize}
\end{frame}


\begin{frame}{Higgs production}
  \begin{figure}[!t]
    \centering
    \includegraphics[width=0.49\linewidth]{figures/higgs-ggF-n3lo.pdf}
    \includegraphics[width=0.49\linewidth]{figures/H_VBF-n3lo.pdf}
  \end{figure}
  \begin{itemize}
    \item Matrix elements for both Higgs in gluon fusiona and VBF available at N3LO
    \item N3LO correction to Higgs in gluon fusion, small suppression compared to NNLO
    \item Higgs in VBF perturbatively stable
  \end{itemize}
\end{frame}


\begin{frame}{Drell-Yan}
  \begin{figure}[!t]
    \centering
    \includegraphics[width=0.49\linewidth]{figures/Z_60_120-n3lo.pdf}
    \includegraphics[width=0.49\linewidth]{figures/Wp_60_120-n3lo.pdf}
  \end{figure}
  \begin{itemize}
    \item Good convergence also for quark initiated processes
  \end{itemize}
\end{frame}




\section*{QED}
\SectionTitleFrame[\hyperlink{https://arxiv.org/abs/2401.08749}{arXiv: 2401.08749}]


\begin{frame}{Including QED corrections in a PDF set}
  The current standard for PDFs determination is at NNLO in QCD, however  $\alpha(M_z) \sim \alpha_s^2(M_Z)$

  \begin{columns}
    \begin{column}{0.49\textwidth}

      \vspace*{1em}
      Including QED corrections in PDFs consists of \vspace*{0.5em}
      \begin{itemize}
        \item QED corrections to DGLAP (at $\mathcal{O}(\alpha)$, $\mathcal{O}(\alpha \alpha_s)$ and $\mathcal{O}(\alpha^2))$: \\
        $P_{QED}=\alpha P_{ij}^{(0,1)}+\alpha \alpha_s P_{ij}^{(1,1)}+\alpha^2 P_{ij}^{(0,2)}+\ldots$
        \vspace*{0.5em}
        \item Adding a photon PDF and including photon initiated contributions to cross-sections \\
        The momenum sumrule is modified accordingly:
        \begin{equation*}
          \int_0^1 dx\, s \left(  \Sigma(x) + g(x) + \gamma(x) \right) =1
        \end{equation*}
      \end{itemize}
    \end{column}

    \begin{column}{0.49\textwidth}
      \begin{figure}
        \includegraphics[width=0.99\textwidth]{figures/ewcorrections_dy.png}
        \caption*{Example: EW corrections in DY\\ {\color{gray}\footnotesize [C. Schwan DIS 2021]}}
      \end{figure}
    \end{column}
  \end{columns}
\end{frame}


\begin{frame}{The photon PDF}
  \begin{columns}
    \begin{column}{0.59\textwidth}
      \begin{itemize}
        \item Data does not provide strong constraints on the photon so fitting is not optimal

        \item The photon PDF can be computed from DIS structure functions using the luxQED result {\color{gray}\footnotesize[Manohar, Nason, Salam, Zanderighi, 1607.04266, 1708.01256]}
        \begin{equation*}
          \begin{split}
            & x \gamma(x, \mu^2)
            =
            \frac{2}{\alpha (\mu^2)} \int\limits_x^1 \frac{dz}{z}
            \Biggl\{ \int_{m_p^2x^2 \over 1-z}^{\mu^2 \over 1-z} \frac{dQ^2}{Q^2}
            \alpha^2(Q^2) \Biggl[ -z^2 F_L(x/z, Q^2) \\
            & + \left( z P_{\gamma q}(z) + \frac{2 x^2 m_p^2}{Q^2} \right)
            F_2(x/z, Q^2)\Biggr] - \alpha^2(\mu^2) z^2 F_2(x/z, \mu^2)\Biggr\}
          \end{split}
        \end{equation*}

        \item An iterative procedure  is used to address the interplay between the photon and other PDFs due to the momentum sumrule
        \begin{equation*}
          \int_0^1 dx\, x \left( \Sigma(x) + g(x) + \gamma(x) \right) =1
        \end{equation*}
      \end{itemize}
    \end{column}

    \begin{column}{0.39\textwidth}
      \vspace*{-1.5em}
      \begin{figure}
        \includegraphics[width=0.9\textwidth]{figures/luxqed_iteration.pdf}
        \caption*{\color{gray}\footnotesize [NNPDF3.1QED: 1712.07053]}
      \end{figure}
    \end{column}
  \end{columns}

\end{frame}


\begin{frame}{Results: Impact of the photon on other PDFs}

  \begin{figure}[!t]
    \centering
    \includegraphics[width=.49\textwidth]{figures/plot_pdfs_u_qed.pdf}
    \includegraphics[width=.49\textwidth]{figures/plot_pdfs_g_qed.pdf}\\
  \end{figure}

  \begin{itemize}
    \item Non-negligible impact, but PDFs are in agreement within uncertainty
    \item Gluon reduced due to momentum sum rule with photon carrying additional momentum
  \end{itemize}
\end{frame}


% \begin{frame}{Results: photon PDF and luminosity}
%   \begin{center}
%     \includegraphics[width=0.3\textwidth]{figures/photon_comparison.pdf}
%     \includegraphics[width=0.3\textwidth]{figures/pp_lumi_comparison.png}
%     \includegraphics[width=0.3\textwidth]{figures/gp_lumi_comparison.pdf}
%   \end{center}
%   \begin{itemize}
%     \item Because all groups use the luxQED formalism, the photon PDFs agree at percent level
%     \item Luminosity generally in agreement, but differ at very small and very large invariant mass
%   \end{itemize}
% \end{frame}


\begin{frame}{Results: phenomenological impact}
  \begin{columns}[T]
    \begin{column}{0.49\textwidth}
      \includegraphics[width=0.9\textwidth]{figures/NNPDF_WPWM_14TEV_40_PHENO-internal.pdf}
    \end{column}
    \begin{column}{0.49\textwidth}
      \includegraphics[width=0.9\textwidth]{figures/NNPDF_DY_14TEV_40_PHENO-internal.pdf}
    \end{column}
  \end{columns}

  \vspace*{1em}
  \begin{itemize}
    \item NLO in both QCD and QED, only PDF uncertainties are shown
    \item Non-negligable QED corrections (up to 5\%) in the large invariant mass and large-$p_T$ regions relevant for new physics searches
    \item In most other cases studied, QED corrections are at the percent level
  \end{itemize}
\end{frame}


\section*{$\alpha_s$ from NNPDF4.0}
\SectionTitleFrame[In preparation]

\begin{frame}{Propagating uncertainties in NNPDF}
  \begin{columns}
    \begin{column}{0.49\textwidth}
      Data is fully defined by central values $\mu_i$ and covariance matrix $\operatorname{cov}_{ij}$
      \begin{enumerate}
        \item Generate $N_{\rm rep}$ { Monte Carlo data} ``replicas'' $\hat{\mu}_i$ such that as $N_{\rm rep}\to \infty$ \\
        $\mu_i = \frac{1}{N_{\rm rep}}\sum_{i=1}^{N_{\rm rep}} \hat{\mu}_i$\\
        $\operatorname{cov}_{ij} = \operatorname{cov}[\hat{\mu}_i, \hat{\mu}_j]$
        \item Perform a { PDF fit} to each replica
        \item Compute observables $X$ and their uncertainties\\
        $\left\langle X\left[f\right]\right\rangle=\frac{1}{N_{\rm rep}} \sum_{r=1}^N X\left[f^{(r)}\right]$\\
        $\operatorname{Var}\left[X\left[f\right]\right] = \frac{1}{N_{\rm rep}} \sum_{r=1}^{N_{\rm rep}}\left(X\left[f^{(r)}\right]-\left\langle X\left[f\right]\right\rangle\right)^2$
      \end{enumerate}
    \end{column}
    \begin{column}{0.49\textwidth}
      \begin{center}
        \includegraphics[width=0.7\textwidth]{replicas_g.pdf}\\
        \includegraphics[width=0.7\textwidth]{band_g.pdf}
      \end{center}
    \end{column}
  \end{columns}
\end{frame}




\begin{frame}{Correlations between $\alpha_s$ and the PDFs}
  \begin{columns}[T]
    \begin{column}{0.49\textwidth}
      \begin{itemize}
        \item Usually $\alpha_s$ determination is done by repeating a PDF fit at different values of $\alpha_s$ and performing a parabolic fit
      \end{itemize}
    \end{column}
    \begin{column}{0.49\textwidth}
      \only<1>{
      \begin{figure}
        \includegraphics[width=0.7\textwidth]{exp_method.png}
        \caption*{\color{gray}\footnotesize \hyperlink{https://arxiv.org/pdf/1110.2483}{NNPDF, 1110.2483}}
      \end{figure}
      }
    \end{column}
  \end{columns}
\end{frame}

\begin{frame}{Correlations between $\alpha_s$ and the PDFs}
  \begin{columns}[T]
    \begin{column}{0.49\textwidth}
      \begin{itemize}
        \item Usually $\alpha_s$ determination is done by repeating a PDF fit at different values of $\alpha_s$ and performing a parabolic fit
        \item Such a determination misses correlations between $\alpha_s$ and the PDF parameters $\theta$, leading to \textbf{underestimated uncertainties}
      \end{itemize}
    \end{column}
    \begin{column}{0.49\textwidth}
      \includegraphics[width=0.9\textwidth]{ellipse.pdf}
    \end{column}
  \end{columns}
\end{frame}


\begin{frame}{A simultaneous optimization of $\alpha_s$}
  Ideally we would minimize simultaneously $\alpha_s$ and the PDF parameters but due to theories being stored in pre-computed grids at fixed values of $\alpha_s$ this is impractical

  % A simultaneous minimization implies solving the system of coupled equations
  % \begin{align}
  %   \label{eq:partial_theta}
  %   \frac{\partial}{\partial \theta} \chi^2(\alpha_s, \theta) &= 0 \\
  %   \frac{\partial}{\partial \alpha_s} \chi^2(\alpha_s, \theta) &= 0
  % \end{align}

  \vspace*{1em}

  A possible way to find the minimum in ($\alpha_s, \theta$) space is:
  \begin{enumerate}
    \item generate a set of pseudodata replicas
    \item fit these data replicas for different values of $\alpha_s$ (thus finding minima for $\theta$)
    \item for each replica, fit a parabola to get a $\chi^2(\alpha_s)$ profile
    \item each minimum corresponds to a sampled $\alpha_s$ value
  \end{enumerate}


  \begin{figure}
    \includegraphics[width=0.5\textwidth]{alphas_density.png}
  \end{figure}

\end{frame}


\begin{frame}{$\alpha_s$ from correlated theory uncertainties}

  In a PDF fit we minimize $\chi^2$:\\
  $P(T|D) \propto \exp\left[-\frac{1}{2}\left(T-D\right)^T\left(\mathrm{Cov}_\mathrm{exp}+\mathrm{Cov}_\mathrm{th}\right)^{-1}\left(T-D\right)\right]
  = \exp\left[-\frac{1}{2}\chi^2\right]$

  \vspace*{1em}
  We can model the theory uncertainty as a correlated shift: \\
  $T \rightarrow T + \lambda \beta $, \quad $\beta=\frac{\partial}{\partial \alpha_s}T$ (numerically, $\beta$ is constructed form discrete shifts)\\
  $P(T|D,\lambda) \propto \exp\left[-\frac{1}{2}\left(T+\lambda\beta-D\right)^T\mathrm{Cov}_\mathrm{exp}^{-1}\left(T+\lambda\beta-D\right)\right]$

  \vspace*{1em}
  We want to find $P(T|D)$ using Bayes' theorem, so we have to choose a prior for $\lambda$. Let us take a unit-width Gaussian\\
  $P(\lambda) \propto \exp\left[-\frac{1}{2}\lambda^2\right]$

  \vspace*{1em}
  Marginalizing over $\lambda$ we find \\
  $P(T|D) \propto \int d\lambda \exp\left[Z^{-1}\left(\lambda-\bar{\lambda}\right)^2\right] \exp\left[-\frac{1}{2}\left(T-D\right)^T\left(\mathrm{Cov}_\mathrm{exp}+\beta\beta^T\right)^{-1}\left(T-D\right)\right]$ \\
  for some functions $Z$ and $\bar{\lambda}$,

  \vspace*{1em}
  Taking $\mathrm{Cov_\mathrm{th}}=\beta\beta^T$, we thus recover \\
  $P(T|D)\propto \exp\left[-\frac{1}{2}\chi^2\right]$

  \vspace*{1em}
  Finally, we can compute
  $P(\lambda|T,D) = \frac{P(T|D,\lambda)P(\lambda)}{P(T|D)}=\exp\left[-\frac{1}{2}Z^{-1}(\lambda - \bar{\lambda})\right]$


  \vfill
  {\color{gray} \footnotesize Ball, Pearson, \hyperlink{https://arxiv.org/abs/2105.05114}{2105.05114}}
\end{frame}


\begin{frame}{Validating the methodology}
  We thus have two possible methodologies to determine $\alpha_s$, but how can we trust they give the correct answer?

  \vspace*{1em}
  \begin{columns}

    \begin{column}{0.49\textwidth}
      Basic idea is that of a closure test:
      \begin{enumerate}
        \item generate data with theory predictions $T_0$ for a given value of $\alpha_s$, e.g. $0.118$ -- centre of the diagram

        \item sample the covariance matrix to simulate experimental sampling $\eta$ \\
        $y_0 = T_0 + \eta$, \quad $\eta  \overset{\text{i.i.d.}}{\sim} \mathcal{N}(0,\mathrm{Cov})$

        \item create many replicas by sampling on top of $y_0$ \\
        $y_\mathrm{rep} = y_0 + \delta$, \quad $\delta  \overset{\text{i.i.d.}}{\sim} \mathcal{N}(0,\mathrm{Cov})$ -- the green circle

        \item perform fits to all replicas $y_\mathrm{rep}$ -- the blue circle

        \item repeat for many $y_0$

        \item[$\bullet$] usual PDF-level closure test: check if for 68\% of $y_0$ samples, $E_\eta$ is within 1$\sigma$ from $T0$

        \item closure test of $\alpha_s$: check if the same applies for the predictions of the $\alpha_s$ value. Remember: we constructed our data samples to know the answer should be $\alpha_s = 0.118$!
      \end{enumerate}
    \end{column}
    \begin{column}{0.49\textwidth}
      \begin{figure}
        \includegraphics[width=0.99\textwidth]{geometric_closure_test.png}
        \caption*{\color{gray} \footnotesize Del Debbio, Giani, Wilson, \hyperlink{https://arxiv.org/abs/2111.05787}{2111.05787}}

      \end{figure}
    \end{column}
  \end{columns}

  \begin{center}
    \vspace*{1em}
    \only<2>{\textbf{Preliminary results based on 25 sampled $y_0$ suggests the ``correlated replicas'' method faithfully determines $\alpha_s$}}
  \end{center}
\end{frame}


\begin{frame}{Process sensitivity}
  \begin{itemize}
    \item Impact of PDF-$\alpha_s$ correlations
    \item Estimate the pull from different processes by comparing $\alpha_s$ determinations from $\chi^2$ calculated to a subset of the data
    \item Global fits account for the effects of all data!
  \end{itemize}
  \begin{figure}
    \includegraphics[width=0.7\textwidth]{as_determination_central_NNLO.pdf}
    \caption*{\color{gray} \footnotesize NNPDF, \hyperlink{https://arxiv.org/abs/1802.03398}{1802.03398}}
  \end{figure}
\end{frame}


\begin{frame}{PRELIMINARY NNPDF4.0 results at NNLO}
  Only \textbf{PDF uncertainty:} \\
  NNPDF3.1-like dataset: $\alpha_s(m_Z) = 0.1185 \pm 0.0006$ (same as 2018 NNPDF3.1 result)\\
  NNPDF4.0 dataset: $\alpha_s(m_Z) = 0.1204 ± 0.0004$

  \vspace*{1em}
  \textbf{PDF uncertainty + MHOU:} \\
  NNPDF4.0 dataset: $\alpha_s(m_Z)=0.1194 \pm 0.0007$ \\
  NNPDF3.1 dataset: $\alpha_s(m_Z) = 0.1182 \pm 0.0008$

  \vspace*{1em}
  LHC data seems to prefer a larger value of $\alpha_s$

  \vspace*{1em}
  Other sources of uncertainty that need to be studied:
  \begin{itemize}
    \item Methodological
    \item Higher twist
    \item Nuclear corrections
  \end{itemize}

  \vspace*{5em}
  \begin{center}
    \textbf{Finally, extend up to aN3LO}
  \end{center}

\end{frame}


\section{Summary and outlook}

\begin{frame}[c]{Summary and outlook}
  \begin{itemize}\setlength{\itemsep}{30pt}
    \item N3LO PDFs and QED corrections are a requirement for LHC predictions at 1\% accuracy
    \item Good perturbative convergence is observed, and NNLO and aN3LO agree with uncertainties
    \item A PDF determination at aN3LO QCD and NLO QED to appear soon
    \item An $\alpha_s$ determination at aN3LO with estimated MHOU and validated methodologies is underway
  \end{itemize}

  \vspace*{7em}
  \only<2>{
  \begin{center}
      {\Large \textbf{Thank you for your attention!}}
  \end{center}
  }
\end{frame}



\appendix
\section{Backup}


\appendix

\section{Backup}
\begin{frame}{K-folding}
\begin{figure}[t]
  \centering
  \begin{tikzpicture}[node distance = 1.0cm]\small
    \node[roundtext, fill=green!30] (hyperopt) {\texttt{hyperopt}};
    \coordinate [above = 1.5cm of hyperopt] (abovehyperopt) {};

    \node[roundtext, right = 2.5cm of abovehyperopt] (xplain) {Generate new hyperparameter configuration};
    \draw[myarrow] (hyperopt) -- (abovehyperopt) -- (xplain);

    \coordinate [below = 1.85cm of xplain.west] (fold4v) {};
    \coordinate [below = 1.85cm of xplain.east] (fold1v) {};
    \coordinate (arrowcenter) at ($(fold4v)!0.5!(fold1v)$);
    \coordinate (fold3v) at ($(fold4v)!0.66!(arrowcenter)$);
    \coordinate (fold2v) at ($(arrowcenter)!0.33!(fold1v)$);

    \node [roundtext, fill=green!30, above = 0.33cm of arrowcenter] (fitto) {Fit to subset of folds};
    \draw[thick] (xplain) -- (fitto);

    \draw[thick] (fold4v) -- (fold1v);
    \draw[thick] (fitto) -- (arrowcenter);

    \node[roundtext, below = 0.4cm of fold4v] (fold4) {folds 1,2,3};
    \node[roundtext, below = 0.4cm of fold1v] (fold1) {folds 2,3,4};
    \node[roundtext, below = 0.4cm of fold3v] (fold3) {folds 1,2,4};
    \node[roundtext, below = 0.4cm of fold2v] (fold2) {folds 1,3,4};
    \draw[myarrow] (fold1v) -- (fold1);
    \draw[myarrow] (fold4v) -- (fold4);
    \draw[myarrow] (fold2v) -- (fold2);
    \draw[myarrow] (fold3v) -- (fold3);

    \node[roundtext, fill=green!30, below = 0.30cm of fold4] (chi24) {$\chi^{2}_{4}$};
    \node[roundtext, fill=green!30, below = 0.30cm of fold3] (chi23) {$\chi^{2}_{3}$};
    \node[roundtext, fill=green!30, below = 0.30cm of fold2] (chi22) {$\chi^{2}_{2}$};
    \node[roundtext, fill=green!30, below = 0.30cm of fold1] (chi21) {$\chi^{2}_{1}$};

    \draw[thick] (fold1) -- (chi21);
    \draw[thick] (fold2) -- (chi22);
    \draw[thick] (fold3) -- (chi23);
    \draw[thick] (fold4) -- (chi24);

    \coordinate [below = 0.3cm of chi24] (below4) {};
    \coordinate [below = 0.3cm of chi21] (below1) {};
    \coordinate [below = 0.3cm of chi22] (below2) {};
    \coordinate [below = 0.3cm of chi23] (below3) {};

    \draw[thick] (below1) -- (below4);
    \draw[thick] (chi24) -- (below4);
    \draw[thick] (chi23) -- (below3);
    \draw[thick] (chi22) -- (below2);
    \draw[thick] (chi21) -- (below1);

    \coordinate (belowcenter) at ($(below4)!0.5!(below1)$);
    \node[operations, below = 0.5cm of belowcenter] (loss) {$L = \frac{1}{4}\displaystyle\sum^{4}_{i}\chi^{2}_{i}$};
    \draw[myarrow] (belowcenter) -- (loss);
    \path let \p1 = (hyperopt), \p2 = (loss)
      in coordinate (lleft) at (\x1,\y2);

    \draw[myarrow] (loss) -- (lleft) -- (hyperopt);

  \end{tikzpicture}
\end{figure}
\end{frame}



\begin{frame}[t]{Self-correlation of PDF sets}
The PDF$_i$-PDF$_j$ correlation for a given flavour is defined as
$$corr_{i,j}(x)= \frac{\sum_{n=1}^{N_{rep}} (f_{i,n}(x) -  f_{i,0}(x) )(f_{j,n}(x) -  f_{j,0}(x)) }{\sqrt{\sum_{n=1}^{N_{rep}}(f_{i,n}(x) - f_{i,0}(x))^2} \sqrt{\sum_{n=1}^{N_{rep}}(f_{j,n}(x) -  f_{j,0}(x))^2}}$$
where $f_{i,n}(x)$ is a PDF replica, and $n=0$ corresponds to the central value of the PDF set. 

\end{frame}

\begin{frame}[t]{Correlated combination of PDFs}

When combining PDF sets $f_i$ in a correlated way, for each momentum fraction $x$ and flavour, the central value is calculated using
$$
\left\langle f\right\rangle_{comb}=\sum_{i=1}^{N_{sets}} w_{i} f_{i,0}
$$

and the variance using
$$
V_{comb}=\sum_{i, j=1}^{N_{sets}} w_{i} \sigma_{i j} w_{j}
$$

where $\sigma$ is the covariance matrix, $f_{0,i}$ is the central value of PDF set $i$, and $w_i$ are the weights 
$$
w_{i}=\frac{\sum_{j=1}^{N_{sets}}\left(\sigma^{-1}\right)_{i j}}{\sum_{k, l=1}^{N_{sets}}\left(\sigma^{-1}\right)_{k l}}
$$

\end{frame}


\end{document}
