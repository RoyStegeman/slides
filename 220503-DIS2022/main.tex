\documentclass[aspectratio=169,9pt]{beamer}
\graphicspath{{figures/}} % Setting the graphicspath

% Theme settings
\usetheme{Madrid}
\usecolortheme{default}
\setbeamertemplate{navigation symbols}{}   % removes navigation symbols such as 'next page'
\setbeamertemplate{footline}{}             % remove line with name, date, page nr. 
\setbeamercolor*{frametitle}{bg=white}     % remove background from frametitle
\usepackage{caption}
% \captionsetup[figure]{labelformat=empty}% redefines the caption setup of the figures environment in the beamer class.
\setbeamersize{text margin left=20pt,text margin right=10pt}

\usefonttheme[onlymath]{serif} % makes beamer math look like article math


%======================= import packages =======================
\usepackage{pifont}       % Pi fonts (Digbats, symbol, etc.)
\usepackage{graphicx}     % More options for \includegraphics
\usepackage{tikz}
\usepackage{appendixnumberbeamer} % separate appendix numbering
\usepackage{booktabs}
\usepackage{hyperref}
\usepackage{tabularx}
\usepackage{amsmath, nccmath}


%======================= page numbering =======================
\addtobeamertemplate{navigation symbols}{}{ \usebeamerfont{footline}
  \insertframenumber / \inserttotalframenumber \hspace*{2mm} \\ \vspace*{1mm} 
}


%=================================== colors ===================================
\definecolor{RoyBlue}{RGB}{22, 46, 69}
\definecolor{RoyGrey}{RGB}{64, 88, 128} 

\newcommand{\hlme}[1]{{\color{red}\bf #1}} % highlight me

\setbeamercolor{structure}{fg=RoyBlue} % itemize, enumerate, etc
\setbeamercolor{frametitle}{fg=RoyGrey}
 \setbeamercolor{section in head/foot}{bg=RoyBlue}


%======================= add progress dots to headline =======================
\setbeamertemplate{headline}{%
    \begin{beamercolorbox}[ht=4mm,dp=4mm]{section in head/foot}
        \insertnavigation{\paperwidth}
    \end{beamercolorbox}%
}%
\makeatother


%======================= add section title page =======================
\AtBeginSection[]{
  \begin{frame}
  \vfill
  \centering
    \usebeamerfont{title}\insertsection\par%
  \vfill
  \end{frame}
}


% \AtBeginSection[]{
%   \begin{frame}
%   %\vfill
%   \centering
%   \begin{beamercolorbox}[sep=8pt,center,shadow=true,rounded=true]{testttt}
%     \usebeamerfont{asdfasdfa}\insertsection\par%
%   \end{beamercolorbox}
%   %\vfill
%   \end{frame}
% }



%=================================== titlepage ===================================
\title{The NNPDF4.0 global analysis of the proton structure}
\date{DIS 2022, 3 May 2022}
\author{Roy Stegeman}
\institute{University of Milan and INFN Milan}
\titlegraphic{\vspace*{6mm}
    \includegraphics[height=0.8cm]{logos/LOGO-ERC.jpg} \hspace{10mm}
	\includegraphics[height=0.8cm]{logos/n3pdflogo_noback.png} \hspace{10mm}
	\includegraphics[height=0.6cm]{logos/nnpdf_logo_official.pdf} \hspace{10mm}
	\includegraphics[height=0.8cm]{logos/Logo_Università_degli_Studi_di_Milano(not_mandatory).png}
	\includegraphics[height=0.8cm]{logos/INFN_logo.png}
    \vspace*{5mm} \\
	\centering{ 
	\fontsize{7.0pt}{0.0pt}\selectfont This project has received funding from the European Union’s Horizon 2020 \\	
    \vspace*{-1mm}
	research and innovation programme under grant agreement No 740006.
	}
}

\defbeamertemplate{title page}{noinstitute}[1][]
{
  \vbox{}
  \vfill
  \begingroup
    \centering
    \begin{beamercolorbox}[sep=8pt,center,#1]{title}
      \usebeamerfont{title}\inserttitle\par%
      \ifx\insertsubtitle\@empty%
      \else%
        \vskip0.25em%
        {\usebeamerfont{subtitle}\usebeamercolor[fg]{subtitle}\insertsubtitle\par}%
      \fi%     
    \end{beamercolorbox}%
    \vskip1em\par
    \begin{beamercolorbox}[sep=0pt,center,#1]{author}
      \usebeamerfont{author}\insertauthor
    \end{beamercolorbox}
	\begin{beamercolorbox}[sep=0pt,center,#1]{author}
		\usebeamerfont{institute}\insertinstitute
	\end{beamercolorbox}
	\vspace*{8pt}
	\begin{beamercolorbox}[sep=0pt,center,#1]{author}
		On behalf of the NNPDF Collaboration \\
        {\small Based on: \href{https://arxiv.org/abs/2109.02653}{arXiv:2109.02653}}
	\end{beamercolorbox}
	\vspace*{16pt}
    \begin{beamercolorbox}[sep=0pt,center,#1]{date}
      \usebeamerfont{date}\insertdate
    \end{beamercolorbox}\vskip0.5em
    {\usebeamercolor[fg]{titlegraphic}\inserttitlegraphic\par}
  \endgroup
  \vfill
}

\makeatletter
\setbeamertemplate{title page}[noinstitute][colsep=-4bp,rounded=true,shadow=\beamer@themerounded@shadow]
\makeatother




\definecolor{Red}{rgb}{1,0,0}
\definecolor{Green}{rgb}{0,1,0}
\definecolor{Blue}{rgb}{0,0,1}
\definecolor{Gray}{gray}{0.9}
\definecolor{springgreen}   {cmyk}{0.26, 0   , 0.76, 0   }
\definecolor{olivegreen}    {cmyk}{0.64, 0   , 0.95, 0.40}
\definecolor{emerald}       {cmyk}{1   , 0   , 0.50, 0   }
\definecolor{junglegreen}   {cmyk}{0.99, 0   , 0.52, 0   }
\definecolor{seagreen}      {cmyk}{0.69, 0   , 0.50, 0   }
\definecolor{green}         {cmyk}{1   , 0   , 1   , 0   }
\definecolor{forestgreen}   {cmyk}{0.91, 0   , 0.88, 0.12}
\definecolor{pinegreen}     {cmyk}{0.92, 0   , 0.59, 0.25}
\definecolor{sepia}         {cmyk}{0   , 0.83, 1   , 0.70}
\definecolor{cerulean}      {cmyk}{0.94, 0.11, 0   , 0   }
\definecolor{salmon}        {cmyk}{0   , 0.53, 0.38, 0   }
\definecolor{greenyellow}   {cmyk}{0.15, 0   , 0.69, 0   }
\definecolor{arsenic}       {rgb}{0.23, 0.27, 0.29}
\definecolor{britishracinggreen}{rgb}{0.0, 0.26, 0.15}
\definecolor{oxfordblue}{rgb}{0.0, 0.13, 0.28}
\definecolor{bostonuniversityred}{rgb}{0.8, 0.0, 0.0}
\definecolor{goldenyellow}{rgb}{1.0, 0.87, 0.0}

\definecolor{darkgreen}{rgb}{0.0, 0.5, 0.13}
\definecolor{darkred}{rgb}{0.55, 0.0, 0.0}
\newcommand{\gct}{\color{darkgreen}\checkmark}
\newcommand{\rma}{\color{red}\ding{55}}
\newcommand{\bct}{\color{blue}\checkmark}
\newcommand{\arrowdownunder}{\begin{center}$\big\downarrow$\end{center}\vspace{-0.3cm}}
\newcommand{\mycolutitle}[1]{\vspace{-0.7cm}\begin{center}#1\end{center}\vspace{-0.1cm}}



\begin{document}
{
\setbeamertemplate{headline}{} % remove headline from titlepage
\begin{frame}
  \titlepage
\end{frame}
}




%======================= tikz settings =======================
\usetikzlibrary{shapes, arrows}
\usetikzlibrary{decorations.pathreplacing}
\usetikzlibrary{positioning, calc}
\tikzstyle{fitted} = [rectangle, minimum width=5cm, minimum height=1cm, text centered, draw=black, fill=red!30]
\tikzstyle{operations} = [rectangle, rounded corners, minimum width=2cm,text centered, draw=black, fill=red!30]
\tikzstyle{roundtext} = [rectangle, rounded corners, minimum width=2cm, minimum height=0.8cm, text centered, draw=black, fill=red!30]
\tikzstyle{n3py} = [rectangle, rounded corners, minimum width=3cm, minimum height=1cm, text centered, draw=black, fill=green!30]
\tikzstyle{myarrow} = [thick,->,>=stealth]
\tikzstyle{line} =[draw, -latex']
\tikzstyle{decision} = [diamond, draw, fill=red!20, text width=7.5em, text centered,  inner sep=0pt, minimum height=2em, aspect=4]
\tikzstyle{cloud} = [draw, ellipse,fill=green!20, minimum height=2em]
\tikzstyle{inout} = [rectangle, draw, fill=green!20, text width=9.5em, text centered, rounded corners, minimum height=2em, minimum width=10em]
\tikzstyle{block}=[rectangle, draw, fill=blue!20, text width=9.5em, 
                   text centered, rounded corners, minimum height=2em, 
                   minimum width=10em]
\tikzstyle{arrow} = [thick,->,>=stealth]

\pgfdeclarelayer{bg}    % declare background layer
\pgfsetlayers{bg,main}  % set the order of the layers (main is the standard layer)




% INTRO ========================================================================
\section*{NNPDF4.0}

\begin{frame}[t]{High-precision: gluon}
	\begin{equation*}
	\mathcal{L}_{i j}\left(M_{X}, y, \sqrt{s}\right)
	=\frac{1}{s} \sum_{i, j} f_{i}\left(\frac{M_{X} e^{y}}{\sqrt{s}}, M_{X}\right) f_{j}\left(\frac{M_{X} e^{-y}}{\sqrt{s}}, M_{X}\right)
	\end{equation*}
	\includegraphics[width=0.45\textwidth]{plot_lumi2d_uncertainty_NNPDF31_gg}
	\includegraphics[width=0.45\textwidth]{plot_lumi2d_uncertainty_NNPDF40_gg}
    \begin{center}
	    \textbf{How did we get here?}
	\end{center}
\end{frame}

\begin{frame}[t]{High-precision: singlet }
	\begin{equation*}
	\mathcal{L}_{i j}\left(M_{X}, y, \sqrt{s}\right)
	=\frac{1}{s} \sum_{i, j} f_{i}\left(\frac{M_{X} e^{y}}{\sqrt{s}}, M_{X}\right) f_{j}\left(\frac{M_{X} e^{-y}}{\sqrt{s}}, M_{X}\right)
	\end{equation*}
	\includegraphics[width=0.45\textwidth]{plot_lumi2d_uncertainty_NNPDF31_qq}
	\includegraphics[width=0.45\textwidth]{plot_lumi2d_uncertainty_NNPDF40_qq}\\
	\begin{center}
	    \textbf{How did we get here?}
	\end{center}
\end{frame}






% DATA =========================================================================
\section{Data}

\begin{frame}{Data from NNPDF1.0 to NNPDF4.0}
	\begin{center}
		\includegraphics[width=0.5\textwidth]{NNPDF_data_history.pdf}
	\end{center}
	The number of datasets -- normally corresponding to different processes -- is generally more relevant than the number of datapoints
\end{frame}


\begin{frame}{Experimental data in NNPDF4.0}
    \begin{columns}
        \column{0.7\linewidth}
            \includegraphics[width=1.0\textwidth]{Markers0_plot_xq2}
        \column{0.25\linewidth}
            New processes:
            \begin{itemize}
                \item direct photon
                \item single top
                \item dijets
                \item W+jet
                \item DIS jet
            \end{itemize}
    % \begin{block}{\footnotesize Theoretical improvement}
    % {\footnotesize
    % Nuclear uncertainties are included
    % }
    % \end{block}
    \end{columns}
\end{frame}








% METHODOLOGY ==================================================================
\section{Methodology}

\begin{frame}[t]{Improved fitting methodology}
    \begin{columns}[T]
        \begin{column}{0.48\textwidth}
            \begin{itemize}
                \item \textbf{Stochastic Gradient Descent} for NN training using TensorFlow
                \item Automated optimization of \\ \textbf{ model hyperparameters}
                \item Methodology is validated using 
                {\bf closure tests} (data region), {\bf future tests} (extrapolation region), and {\bf parametrization basis independence}
            \end{itemize}
        \vspace*{1em}
        Physical constraints:
        \begin{itemize}
            \item PDF positivity
            \item Integrability of nonsinglet distributions (Gottfried sum rules)
        \end{itemize}
        \end{column}
        \begin{column}{0.48\textwidth}
            \vspace*{-3em}
            \includegraphics[width=1.0\textwidth]{NNarch}
            \begin{equation*}
                f_{i}\left(x, Q_{0}\right)=x^{-\alpha_{i}}(1-x)^{\beta_{i}} \mathrm{NN}_{i}(x)
            \end{equation*}
        \end{column}
    \end{columns}
\end{frame}


\begin{frame}[t]{Automated model selection}
	NNPDF aims to minimize sources of bias in the PDF:
	\begin{itemize}
	    \item Functional form $\rightarrow$ Neural Network
	    \item Model parameters $\rightarrow$ ?
	\end{itemize}
\end{frame}


\begin{frame}[t]{Automated model selection}
	NNPDF aims to minimize sources of bias in the PDF:
	\begin{itemize}
	    \item Functional form $\rightarrow$ Neural Network
	    \item Model parameters $\rightarrow$ \textbf{Hyperoptimization}
	\end{itemize}
    \begin{columns}
        \begin{column}{0.48\textwidth}
            Scan over thousands of hyperparameter combinations and select the best one \\
            \vspace*{0.8em}
            {\bf k-fold cross-validation}: used to define the reward function based on a {\bf test dataset}\\ 
            \vspace*{0.8em}
            Objective function: \\
            $L=\textrm{mean}(\chi_1^2,\chi_3^2,\chi_2^2,\ldots, \chi_k^2)$
        \end{column}
        \begin{column}{0.48\textwidth}
            \begin{center}
                \includegraphics[width=0.48\textwidth]{sec_methodology_hyperopt_plot_initializer.pdf}
                \includegraphics[width=0.48\textwidth]{sec_methodology_hyperopt_plot_lr.pdf} \\
                \includegraphics[width=0.48\textwidth]{sec_methodology_hyperopt_plot_number_of_layers.pdf}
                \includegraphics[width=0.48\textwidth]{sec_methodology_hyperopt_plot_optimizers.pdf}
            \end{center}
        \end{column}
    \end{columns}
\end{frame}





% Stability =======================================================
\section{Stability}
\begin{frame}{Parametrization basis independence}
    \begin{columns}
        \begin{column}[T]{0.48\textwidth}
        \vspace*{0pt}%
	        \begin{center}
	            \includegraphics[width=0.8\textwidth]{flavour_evolution_V} \\
	        \end{center}
        \end{column}
        \begin{column}[t]{0.48\textwidth}
        \vspace{0pt}%
	        \begin{center}
	            \includegraphics[width=0.8\textwidth]{flavour_evolution_T3} \\
	        \end{center}
        \end{column}
    \end{columns}
    \begin{columns}
        \column{0.4\linewidth}
		    Evolution Basis:
		    {\footnotesize
		    \begin{fleqn}
		    \begin{align*}
		       \qquad x V\left(x, Q_{0}\right) &\propto \mathrm{NN}_{V}(x)\\
		        x T_{3}\left(x, Q_{0}\right) &\propto \mathrm{NN}_{T_{3}}(x)
		    \end{align*}
		    \end{fleqn}
		    }
        \column{0.55\linewidth}
            \begin{block}{}
                Different strategies to parametrize the quark PDF flavour combinations leave the uncertainties essentially unchanged
            \end{block}
    \end{columns}
    \vspace*{-0.5em}
    Flavour Basis:
    {\footnotesize
    \begin{fleqn}
    \begin{align*}
        \qquad x V\left(x, Q_{0}\right) &\propto\left(\mathrm{NN}_{u}(x)-\mathrm{NN}_{\bar{u}}(x)+\mathrm{NN}_{d}(x)-\mathrm{NN}_{\bar{d}}(x)+\mathrm{NN}_{s}(x)-\mathrm{NN}_{\bar{s}}(x)\right) \\
        x T_{3}\left(x, Q_{0}\right) &\propto\left(\mathrm{NN}_{u}(x)+\mathrm{NN}_{\bar{u}}(x)-\mathrm{NN}_{d}(x)-\mathrm{NN}_{\bar{d}}(x)\right)
    \end{align*}
    \end{fleqn}
    }
\end{frame}

\begin{frame}[t]{Impact of the new data}
	\includegraphics[width=0.45\textwidth]{lumi1d_gg_NNPDF40meth_NNPDF31data}
	\includegraphics[width=0.45\textwidth]{lumi1d_qq_NNPDF40meth_NNPDF31data}
	Individual datasets have a limited impact, but collectively they result in:
	\begin{itemize}
	    \item Moderate reduction of PDF uncertainties
	    \item Shifts in central value at the one-sigma level
	\end{itemize}
\end{frame}


\begin{frame}[t]{Impact of the new fitting methodology}
	\includegraphics[width=0.45\textwidth]{lumi1d_gg_NNPDF31meth_NNPDF40data}
	\includegraphics[width=0.45\textwidth]{lumi1d_qq_NNPDF31meth_NNPDF40data}
	\begin{columns}
	    \column{0.45\linewidth}
			\begin{itemize}
	    	        \item Significant reduction of PDF uncertainties
		        \item Good agreement between the central values
		    \end{itemize}
        \column{0.5\linewidth}
            \begin{block}{}
                \fontsize{7}{6}\selectfont
                PDF uncertainties are validated using closure tests and future tests\\
                Validation tests successful for both NNPDF4.0 and NNPDF3.1 
            \end{block}
    \end{columns}
\end{frame}



% PHENOMENOLOGY ================================================================
\section{LHC phenomenology}
\begin{frame}[t]{Implications for LHC phenomenology}
    \begin{center}
        Reduced luminosity uncertainties $\rightarrow$ Reduced uncertainty at the level of observables\\
        \vspace*{-0.5em}
        \begin{columns}
          \begin{column}{0.48\textwidth}
              \begin{center}
                  \includegraphics[width=0.78\textwidth]{NNPDF_TTB_14TEV_40_PHENO-internal} 
              \end{center}
          \end{column}
          \begin{column}{0.48\textwidth}
              \includegraphics[width=0.7\textwidth]{NNPDF_H_14TEV_40_PHENO-integrated}\\
                  \includegraphics[width=0.7\textwidth]{NNPDF_TTB_14TEV_40_PHENO-integrated}
          \end{column}
        \end{columns}
    \end{center}
\end{frame}


% DELIVERY ====================================================================
\section{Open-source code}
\begin{frame}[t]{The open-source NNPDF code}
    The full NNPDF code has been made public along with user friendly documentation\\
    \vspace*{1em}
    This includes: fitting, hyperoptimization, theory, data processing, visualization\\
    \vspace*{1em}
    It is possible to reproduce all results of NNPDF4.0 and more!\\
    \vspace*{2em}
    \begin{block}{}
        \centering
		\href{https://link.springer.com/article/10.1140/epjc/s10052-021-09747-9}{Eur.Phys.J.C 81 (2021) 10, 958} \\
		\url{https://github.com/NNPDF/nnpdf} \\
		\url{https://docs.nnpdf.science}
    \end{block}
\end{frame}



% CONCLUSION ===================================================================
% \section{Beyond NNPDF4.0}

% \begin{frame}{Importance of scaling $x$}
%     \begin{center}
%         \includegraphics[width=0.45\textwidth]{pdf_g_without_xlogx.pdf}
%     \end{center}
%     \textbf{Problem:} Improper scaling of $x$ can lead to saturation of the neural network and affect the resulting PDFs, NNPDF solves this by passing ($x,\log x$) to the network \\
%     \vspace*{0.5em}
%     The convergence Gradient descent based algorithms struggles with inputs spanning different orders of magnitude
% \end{frame}


% \begin{frame}{Feature Scaling}{\href{https://link.springer.com/article/10.1140/epjc/s10052-022-10136-z}{\color{blue} Eur.Phys.J.C 82 (2022) 2}}
%     \textbf{Solution:} enforce all inputs $X$ to be on a similar scale \\
%     \textbf{How?} Map $X$ to grid of equidistant points using the emprical Cumulative Distribution Funciton (eCDF) \\
%     $ \hat{F}_n(x) = \frac{1}{n} \sum_{i=1}^{n} \mathbf{1}_{X_{i} \leq x}  $
%     \begin{center}
%         \includegraphics[width=0.45\textwidth]{pdf_gluon_log_ecdf_nnpdf40.pdf}
%     \end{center}
% \end{frame}


% \begin{frame}{Feature Scaling}{\href{https://link.springer.com/article/10.1140/epjc/s10052-022-10136-z}{\color{blue} Eur.Phys.J.C 82 (2022) 2}}
%     Convergence without preprocessing \\
%     \vspace*{0.5em}
%     $
%     f_{i}\left(x, Q_{0}\right)=x^{-\alpha_{i}}(1-x)^{\beta_{i}} \mathrm{NN}_{i}(x,\log x) 
%     \rightarrow
%     f_{i}\left(x, Q_{0}\right)= \mathrm{NN}_{i}(\hat{F}_n(x)) 
%     $\\
%     \vspace*{0.5em}
%     Saturation in the small-$x$ extrapolation region
%     \begin{center}
%         \includegraphics[width=0.45\textwidth]{pdf_gluon_log_feature_vs_nnpdf40.pdf}
%     \end{center}
% \end{frame}



% \begin{frame}[t]{Beyond $k$-folds hyperoptimization}

%     Statistical effects can screen out overfitting during hyperoptimization:

%     \begin{center}
%       \includegraphics[width=0.4\textwidth]{hyperopt_choice_charm_plot_pdfs_c.pdf}
%       \includegraphics[width=0.4\textwidth]{hyperopt_choice_strange_plot_pdfs_s.pdf}
%     \end{center}
%     \begin{center}
%         \textbf{Can we quantify overfitting?}
%     \end{center}
% \end{frame}


% \begin{frame}[t]{A measure for overfitting}

%     \textbf{Idea:} for any PDF the validation loss to the fitted pseudodata $\chi^{r}_\text{val}$ should be equal to the loss calculated for any other pseudodata set  $\chi^{\hat{r}}_\text{val}$
%     \\\vspace*{1em}

%     Thus a metric for overfitting is
%     $$
%     \Delta\chi^2_{\text{overfit}}=\langle \chi^{2}_\text{val,$\hat{r}$} - \chi^{2}_\text {val,r}\rangle\quad (<0 \text{ if overfitted})
%     $$

%     % While \textbf{underfitted} setups will be filtered due to their higher $\chi^2$ values 

%     \begin{columns}
%         \begin{column}{0.48\textwidth}
%             \centering
%             \includegraphics[width=0.8\textwidth]{hyperopt_choice_charm_plot_pdfs_c.pdf}
%             $\Delta\chi^2_{\text{overfit}}=-0.0459 \pm 0.0078$ \\
%             $5.9\sigma$ from 0
%         \end{column}
%         \begin{column}{0.48\textwidth}
%             \centering
%             \includegraphics[width=0.8\textwidth]{NNPDF40_fit_charm_plot_pdfs_c.pdf}\\ 
%             $\Delta\chi^2_{\text{overfit}}=-0.0012 \pm 0.0130$ \\
%             $0.1\sigma$ from 0
%         \end{column}
%     \end{columns}


% \end{frame}


% CONCLUSION ===================================================================
\section{Summary and Outlook}
\begin{frame}[t]{Summary and Outlook}
    \begin{itemize}
        \item NNPDF4.0 is the latest release in the NNPDF family of PDF sets
        \item 44 new datasets from many new processes are included
        \item Improved methodology with Stochastic Gradient Descent and hyperoptimization
        \item Validation of PDF uncertainties using closure test, future test and parametrization basis independence
        \item[$\Rightarrow$] NNPDF4.0 achieves a high precision over a broad kinematic range
    \end{itemize}
	\vspace*{1em}
    \begin{itemize}
        \item The current level of PDF uncertainties challenges the accuracy of theoretical predictions and demands an increased effort towards the systematic inclusion in the fit of theoretical uncertainties (nuclear, higher orders, SM parameters, \ldots ) and higher-order QCD and EW corrections
    \end{itemize}


    \vspace*{1em}
    \only<2>{
    \begin{center}
        {\Large \textbf{Thank you!}}
    \end{center}
    }
\end{frame}




\appendix

\section{Backup}
\begin{frame}{K-folding}
\begin{figure}[t]
  \centering
  \begin{tikzpicture}[node distance = 1.0cm]\small
    \node[roundtext, fill=green!30] (hyperopt) {\texttt{hyperopt}};
    \coordinate [above = 1.5cm of hyperopt] (abovehyperopt) {};

    \node[roundtext, right = 2.5cm of abovehyperopt] (xplain) {Generate new hyperparameter configuration};
    \draw[myarrow] (hyperopt) -- (abovehyperopt) -- (xplain);

    \coordinate [below = 1.85cm of xplain.west] (fold4v) {};
    \coordinate [below = 1.85cm of xplain.east] (fold1v) {};
    \coordinate (arrowcenter) at ($(fold4v)!0.5!(fold1v)$);
    \coordinate (fold3v) at ($(fold4v)!0.66!(arrowcenter)$);
    \coordinate (fold2v) at ($(arrowcenter)!0.33!(fold1v)$);

    \node [roundtext, fill=green!30, above = 0.33cm of arrowcenter] (fitto) {Fit to subset of folds};
    \draw[thick] (xplain) -- (fitto);

    \draw[thick] (fold4v) -- (fold1v);
    \draw[thick] (fitto) -- (arrowcenter);

    \node[roundtext, below = 0.4cm of fold4v] (fold4) {folds 1,2,3};
    \node[roundtext, below = 0.4cm of fold1v] (fold1) {folds 2,3,4};
    \node[roundtext, below = 0.4cm of fold3v] (fold3) {folds 1,2,4};
    \node[roundtext, below = 0.4cm of fold2v] (fold2) {folds 1,3,4};
    \draw[myarrow] (fold1v) -- (fold1);
    \draw[myarrow] (fold4v) -- (fold4);
    \draw[myarrow] (fold2v) -- (fold2);
    \draw[myarrow] (fold3v) -- (fold3);

    \node[roundtext, fill=green!30, below = 0.30cm of fold4] (chi24) {$\chi^{2}_{4}$};
    \node[roundtext, fill=green!30, below = 0.30cm of fold3] (chi23) {$\chi^{2}_{3}$};
    \node[roundtext, fill=green!30, below = 0.30cm of fold2] (chi22) {$\chi^{2}_{2}$};
    \node[roundtext, fill=green!30, below = 0.30cm of fold1] (chi21) {$\chi^{2}_{1}$};

    \draw[thick] (fold1) -- (chi21);
    \draw[thick] (fold2) -- (chi22);
    \draw[thick] (fold3) -- (chi23);
    \draw[thick] (fold4) -- (chi24);

    \coordinate [below = 0.3cm of chi24] (below4) {};
    \coordinate [below = 0.3cm of chi21] (below1) {};
    \coordinate [below = 0.3cm of chi22] (below2) {};
    \coordinate [below = 0.3cm of chi23] (below3) {};

    \draw[thick] (below1) -- (below4);
    \draw[thick] (chi24) -- (below4);
    \draw[thick] (chi23) -- (below3);
    \draw[thick] (chi22) -- (below2);
    \draw[thick] (chi21) -- (below1);

    \coordinate (belowcenter) at ($(below4)!0.5!(below1)$);
    \node[operations, below = 0.5cm of belowcenter] (loss) {$L = \frac{1}{4}\displaystyle\sum^{4}_{i}\chi^{2}_{i}$};
    \draw[myarrow] (belowcenter) -- (loss);
    \path let \p1 = (hyperopt), \p2 = (loss)
      in coordinate (lleft) at (\x1,\y2);

    \draw[myarrow] (loss) -- (lleft) -- (hyperopt);

  \end{tikzpicture}
\end{figure}
\end{frame}



\begin{frame}[t]{Self-correlation of PDF sets}
The PDF$_i$-PDF$_j$ correlation for a given flavour is defined as
$$corr_{i,j}(x)= \frac{\sum_{n=1}^{N_{rep}} (f_{i,n}(x) -  f_{i,0}(x) )(f_{j,n}(x) -  f_{j,0}(x)) }{\sqrt{\sum_{n=1}^{N_{rep}}(f_{i,n}(x) - f_{i,0}(x))^2} \sqrt{\sum_{n=1}^{N_{rep}}(f_{j,n}(x) -  f_{j,0}(x))^2}}$$
where $f_{i,n}(x)$ is a PDF replica, and $n=0$ corresponds to the central value of the PDF set. 

\end{frame}

\begin{frame}[t]{Correlated combination of PDFs}

When combining PDF sets $f_i$ in a correlated way, for each momentum fraction $x$ and flavour, the central value is calculated using
$$
\left\langle f\right\rangle_{comb}=\sum_{i=1}^{N_{sets}} w_{i} f_{i,0}
$$

and the variance using
$$
V_{comb}=\sum_{i, j=1}^{N_{sets}} w_{i} \sigma_{i j} w_{j}
$$

where $\sigma$ is the covariance matrix, $f_{0,i}$ is the central value of PDF set $i$, and $w_i$ are the weights 
$$
w_{i}=\frac{\sum_{j=1}^{N_{sets}}\left(\sigma^{-1}\right)_{i j}}{\sum_{k, l=1}^{N_{sets}}\left(\sigma^{-1}\right)_{k l}}
$$

\end{frame}







\end{document}
