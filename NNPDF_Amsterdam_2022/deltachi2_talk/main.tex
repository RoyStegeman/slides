\documentclass[aspectratio=169,11pt]{beamer}
\graphicspath{{figures/}} % Setting the graphicspath

% Theme settings
\usetheme{Madrid}
\usecolortheme{default}
\setbeamertemplate{navigation symbols}{}   % removes navigation symbols such as 'next page'
\setbeamertemplate{footline}{}             % remove line with name, date, page nr. 
\setbeamercolor*{frametitle}{bg=white}     % remove background from frametitle
\usepackage{caption}
% \captionsetup[figure]{labelformat=empty}% redefines the caption setup of the figures environment in the beamer class.
\setbeamersize{text margin left=20pt,text margin right=10pt}

\usefonttheme[onlymath]{serif} % makes beamer math look like article math


%======================= import packages =======================
\usepackage{graphicx}     % More options for \includegraphics
\usepackage{appendixnumberbeamer} % separate appendix numbering
\usepackage{hyperref}


%======================= page numbering =======================
\addtobeamertemplate{navigation symbols}{}{ \usebeamerfont{footline}
  \insertframenumber / \inserttotalframenumber \hspace*{2mm} \\ \vspace*{1mm} 
}


%=================================== colors ===================================
\definecolor{RoyBlue}{RGB}{22, 46, 69}
\definecolor{RoyGrey}{RGB}{64, 88, 128} 

\newcommand{\hlme}[1]{{\color{red}\bf #1}} % highlihgt me

\setbeamercolor{structure}{fg=RoyBlue} % itemize, enumerate, etc
\setbeamercolor{frametitle}{fg=RoyGrey}
\setbeamercolor{section in head/foot}{bg=RoyBlue}


%======================= add progress dots to headline =======================
\setbeamertemplate{headline}{%
    \begin{beamercolorbox}[ht=4mm,dp=4mm]{section in head/foot}
        \insertnavigation{\paperwidth}
    \end{beamercolorbox}%
}%
\makeatother


%======================= add section title page =======================
\AtBeginSection[]{
  \begin{frame}
  \vfill
  \centering
    \usebeamerfont{title}\insertsectionhead\par%
  \vfill
  \end{frame}
}


%================================== TITLEPAGE ==================================
\title{Hyperoptimization - detecting overfitting}
\date{NNPDF meeting, 11 April 2022, Amsterdam}
\author{Roy Stegeman}
\institute{University of Milan and INFN Milan}
\titlegraphic{\vspace*{6mm}
    \includegraphics[height=0.8cm]{logos/LOGO-ERC.jpg} \hspace{10mm}
	\includegraphics[height=0.8cm]{logos/n3pdflogo_noback.png} \hspace{10mm}
	\includegraphics[height=0.6cm]{logos/nnpdf_logo_official.pdf} \hspace{10mm}
	\includegraphics[height=0.8cm]{logos/Logo_Università_degli_Studi_di_Milano(not_mandatory).png}
	\includegraphics[height=0.8cm]{logos/INFN_logo.png}
    \vspace*{5mm} \\
	\centering{ 
	\fontsize{7.0pt}{0.0pt}\selectfont This project has received funding from the European Union’s Horizon 2020 \\	
    \vspace*{-1mm}
	research and innovation programme under grant agreement No 740006.
	}
}




%================================== SLIDES ==================================

\begin{document}
{
\setbeamertemplate{headline}{} % remove headline from titlepage
\begin{frame}
  \titlepage
\end{frame}
}




\section*{Current status}

\begin{frame}[t]{Hyperparameter selection}
  \begin{columns}[T]
    \begin{column}{0.6\textwidth}
      Currently: k-folds Hyperoptimization\\\vspace*{1em}

      This results in the possiblity of overfitted or underfitted setups, in part due to fluctuations (think preprocessing exponents)\\\vspace*{1em}

      To get a ``nice'' PDF we do a manual selection after the automated hyperoptimization, re-introducing human bias\\\vspace*{1em}

      To reduce bias we would like a numerical objective metric for overfitting or underfitting

    \end{column}
    \begin{column}{0.4\textwidth}
      \vspace*{-12mm}
      \includegraphics[height=0.45\textheight]{hyperopt_choice_charm_plot_pdfs_c.pdf}\\
      \includegraphics[height=0.45\textheight]{hyperopt_choice_strange_plot_pdfs_s.pdf}
    \end{column}
  \end{columns}
\end{frame}






\section*{A new metric}

\begin{frame}[t]{The idea}

  Ideally, we have an objective metric that is not relative (such as arc-length), but absolute\\\vspace*{1em}

  Correlation between PDFs and validation data suggests overfitting\\\vspace*{1em}

  \textbf{How can we detect when this happens?}
\end{frame}


\begin{frame}[t]{The idea}

  \textbf{Realization: for any PDF the validation loss $\chi^{r}_\text{val}$ should be equal to the ``validation loss'' calculated for any other pseudodata set  $\chi^{\hat{r}}_\text{val}$ (with the same tr/vl mask)}\\\vspace*{1em}

  % For an overfitted PDF we \textbf{expect} $\chi^{r}_\text{val} < \chi^{\hat{r}}_\text{val}$\\\vspace*{1em}

  Thus as a metric for overfitting we might consider
  $$
  \Delta\chi^2_{\text{overfit}}=\langle \chi^{2}_\text{val,$\hat{r}$} - \chi^{2}_\text {val,r}\rangle\quad (<0 \text{ if overfitted})
  $$

  % Though we will need to account for statistical fluctuations:
  % $$ 
  % R_{\text{overfit}}= 
  % \frac{\langle \chi^{\hat{r}}_\text{val} - \chi^{r}_\text {val}\rangle_{\hat{r}}}
  % {\sqrt{\langle ( \chi^{\hat{r}}_\text{val} - \chi^{r}_\text {val}-\langle \chi^{\hat{r}}_\text{val} - \chi^{r}_\text {val}\rangle_{\hat{r}})^2 \rangle_{\hat{r}}}}
  % = \frac{\langle \chi^{\hat{r}}_\text{val} - \chi^{r}_\text {val}\rangle_{\hat{r}}}{\sigma_\text{bootstrap}}
  % $$\\\vspace*{1em}

  While \textbf{underfitted} setups will be filtered due to their higher $\chi^2$ values 

\end{frame}


\section*{Examples}

\begin{frame}[t]{(How) does this work?}

  Let's have a look at a clearly overfitted PDF (preferred by hyperopt):

  \begin{center}
    \includegraphics[width=0.4\textwidth]{hyperopt_choice_charm_plot_pdfs_c.pdf}
    \includegraphics[width=0.4\textwidth]{hyperopt_choice_strange_plot_pdfs_s.pdf}
  \end{center}

  \begin{center}
    $\Delta\chi^2_{\text{overfit}}=-0.0459 \pm 0.0078$ \quad $5.9\sigma$ from 0
  \end{center}
  \textit{The $\Delta\chi^2_{\text{overfit}}$ values and bootstrap errors in these slides are determined using PDFs with $N_{\text{rep}}=100$}

\end{frame}


\begin{frame}[t]{(How) does this work?}

  And now for a PDF that is a bit smoother (ranked \#8 by hyperopt):

  \begin{center}
    \includegraphics[width=0.4\textwidth]{better_fit_charm_plot_pdfs_c.pdf}
    \includegraphics[width=0.4\textwidth]{better_fit_strange_plot_pdfs_s.pdf}
  \end{center}

  \begin{center}
    $\Delta\chi^2_{\text{overfit}}=-0.0168 \pm 0.0105$ \quad $1.6\sigma$ from 0
  \end{center}
  The distance from 0 decreases as expected

\end{frame}


\begin{frame}[t]{(How) does this work?}

  In the past we have had a scenario where this metric would have helped a lot:

  \begin{center}
    \includegraphics[width=0.4\textwidth]{clipnorm_fit_charm_plot_pdfs_c.pdf}
    \includegraphics[width=0.4\textwidth]{clipnorm_fit_strange_plot_pdfs_s.pdf}
  \end{center}

  \begin{center}
    $\Delta\chi^2_{\text{overfit}}=-0.0236 \pm 0.0126$ \quad $1.9\sigma$ from 0
  \end{center}

  A clear indicator that the clipnorm bugged fit is overfitted!

\end{frame}


\begin{frame}[t]{(How) does this work?}

  And what about NNPDF4.0?

  \begin{center}
    \includegraphics[width=0.4\textwidth]{NNPDF40_fit_charm_plot_pdfs_c.pdf}
    \includegraphics[width=0.4\textwidth]{NNPDF40_fit_strange_plot_pdfs_s.pdf}
  \end{center}

  \begin{center}
    $\Delta\chi^2_{\text{overfit}}=-0.0012 \pm 0.0130$ \quad $0.1\sigma$ from 0
  \end{center}

\end{frame}


\begin{frame}[t]{How can this be used in NNPDF?}

  As an a-posteriory check similar to (but cheaper than) the closure test\\\vspace*{1em}

  \setbeamertemplate{enumerate items}[default]
  \begin{enumerate}
    \item Run hyperoptimization
    \item Select N best setups and do full ~100 replica fits for each
    \item Calculate the estimators for all
    \item Discard setups with e.g. $R_{\text{overfit}} < -1$
    \item
    \begin{enumerate}
      \item Increase number of replicas and repeat...
      \item or select the best of the remaining fit
    \end{enumerate}
  \end{enumerate}\vspace*{1em}

  5.1: If the bootstrap error becomes small enough we will likely always get a negative $\Delta\chi^2_{\text{overfit}}$

  5.2: What is an acceptable $\Delta\chi^2_{\text{overfit}}$? \\
  \qquad How do we define the best fit ($\chi^2_{\text{val}}$, $\chi^2_{\text{tr}}$, $\chi^2_{\text{exp}}$, \ldots)?\\\vspace*{1em}

  \only<1>{\color{white}Conclusion: The $\Delta\chi^2_{\text{overfit}}$ provides a metric for overfitting that can be used to flag overfitted hyperparameter setups and thereby reduce human bias}
  \only<2>{Conclusion: The $\Delta\chi^2_{\text{overfit}}$ provides a metric for overfitting that can be used to flag overfitted hyperparameter setups and thereby reduce human bias}
\end{frame}



\appendix

\section{Backup}
\begin{frame}{K-folding}
\begin{figure}[t]
  \centering
  \begin{tikzpicture}[node distance = 1.0cm]\small
    \node[roundtext, fill=green!30] (hyperopt) {\texttt{hyperopt}};
    \coordinate [above = 1.5cm of hyperopt] (abovehyperopt) {};

    \node[roundtext, right = 2.5cm of abovehyperopt] (xplain) {Generate new hyperparameter configuration};
    \draw[myarrow] (hyperopt) -- (abovehyperopt) -- (xplain);

    \coordinate [below = 1.85cm of xplain.west] (fold4v) {};
    \coordinate [below = 1.85cm of xplain.east] (fold1v) {};
    \coordinate (arrowcenter) at ($(fold4v)!0.5!(fold1v)$);
    \coordinate (fold3v) at ($(fold4v)!0.66!(arrowcenter)$);
    \coordinate (fold2v) at ($(arrowcenter)!0.33!(fold1v)$);

    \node [roundtext, fill=green!30, above = 0.33cm of arrowcenter] (fitto) {Fit to subset of folds};
    \draw[thick] (xplain) -- (fitto);

    \draw[thick] (fold4v) -- (fold1v);
    \draw[thick] (fitto) -- (arrowcenter);

    \node[roundtext, below = 0.4cm of fold4v] (fold4) {folds 1,2,3};
    \node[roundtext, below = 0.4cm of fold1v] (fold1) {folds 2,3,4};
    \node[roundtext, below = 0.4cm of fold3v] (fold3) {folds 1,2,4};
    \node[roundtext, below = 0.4cm of fold2v] (fold2) {folds 1,3,4};
    \draw[myarrow] (fold1v) -- (fold1);
    \draw[myarrow] (fold4v) -- (fold4);
    \draw[myarrow] (fold2v) -- (fold2);
    \draw[myarrow] (fold3v) -- (fold3);

    \node[roundtext, fill=green!30, below = 0.30cm of fold4] (chi24) {$\chi^{2}_{4}$};
    \node[roundtext, fill=green!30, below = 0.30cm of fold3] (chi23) {$\chi^{2}_{3}$};
    \node[roundtext, fill=green!30, below = 0.30cm of fold2] (chi22) {$\chi^{2}_{2}$};
    \node[roundtext, fill=green!30, below = 0.30cm of fold1] (chi21) {$\chi^{2}_{1}$};

    \draw[thick] (fold1) -- (chi21);
    \draw[thick] (fold2) -- (chi22);
    \draw[thick] (fold3) -- (chi23);
    \draw[thick] (fold4) -- (chi24);

    \coordinate [below = 0.3cm of chi24] (below4) {};
    \coordinate [below = 0.3cm of chi21] (below1) {};
    \coordinate [below = 0.3cm of chi22] (below2) {};
    \coordinate [below = 0.3cm of chi23] (below3) {};

    \draw[thick] (below1) -- (below4);
    \draw[thick] (chi24) -- (below4);
    \draw[thick] (chi23) -- (below3);
    \draw[thick] (chi22) -- (below2);
    \draw[thick] (chi21) -- (below1);

    \coordinate (belowcenter) at ($(below4)!0.5!(below1)$);
    \node[operations, below = 0.5cm of belowcenter] (loss) {$L = \frac{1}{4}\displaystyle\sum^{4}_{i}\chi^{2}_{i}$};
    \draw[myarrow] (belowcenter) -- (loss);
    \path let \p1 = (hyperopt), \p2 = (loss)
      in coordinate (lleft) at (\x1,\y2);

    \draw[myarrow] (loss) -- (lleft) -- (hyperopt);

  \end{tikzpicture}
\end{figure}
\end{frame}



\begin{frame}[t]{Self-correlation of PDF sets}
The PDF$_i$-PDF$_j$ correlation for a given flavour is defined as
$$corr_{i,j}(x)= \frac{\sum_{n=1}^{N_{rep}} (f_{i,n}(x) -  f_{i,0}(x) )(f_{j,n}(x) -  f_{j,0}(x)) }{\sqrt{\sum_{n=1}^{N_{rep}}(f_{i,n}(x) - f_{i,0}(x))^2} \sqrt{\sum_{n=1}^{N_{rep}}(f_{j,n}(x) -  f_{j,0}(x))^2}}$$
where $f_{i,n}(x)$ is a PDF replica, and $n=0$ corresponds to the central value of the PDF set. 

\end{frame}

\begin{frame}[t]{Correlated combination of PDFs}

When combining PDF sets $f_i$ in a correlated way, for each momentum fraction $x$ and flavour, the central value is calculated using
$$
\left\langle f\right\rangle_{comb}=\sum_{i=1}^{N_{sets}} w_{i} f_{i,0}
$$

and the variance using
$$
V_{comb}=\sum_{i, j=1}^{N_{sets}} w_{i} \sigma_{i j} w_{j}
$$

where $\sigma$ is the covariance matrix, $f_{0,i}$ is the central value of PDF set $i$, and $w_i$ are the weights 
$$
w_{i}=\frac{\sum_{j=1}^{N_{sets}}\left(\sigma^{-1}\right)_{i j}}{\sum_{k, l=1}^{N_{sets}}\left(\sigma^{-1}\right)_{k l}}
$$

\end{frame}

\end{document}
