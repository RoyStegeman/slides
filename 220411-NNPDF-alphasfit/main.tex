\documentclass[aspectratio=169,10pt]{beamer}
\graphicspath{{figures/}} % Setting the graphicspath

% Theme settings
\usetheme{Madrid}
\usecolortheme{default}
\setbeamertemplate{navigation symbols}{}   % removes navigation symbols such as 'next page'
\setbeamertemplate{footline}{}             % remove line with name, date, page nr. 
\setbeamercolor*{frametitle}{bg=white}     % remove background from frametitle
\usepackage{caption}
% \captionsetup[figure]{labelformat=empty}% redefines the caption setup of the figures environment in the beamer class.
\setbeamersize{text margin left=20pt,text margin right=10pt}

\usefonttheme[onlymath]{serif} % makes beamer math look like article math


%======================= import packages =======================
\usepackage{graphicx}     % More options for \includegraphics
\usepackage{appendixnumberbeamer} % separate appendix numbering
\usepackage{hyperref}


%======================= page numbering =======================
\addtobeamertemplate{navigation symbols}{}{ \usebeamerfont{footline}
  \insertframenumber / \inserttotalframenumber \hspace*{2mm} \\ \vspace*{1mm} 
}


%=================================== colors ===================================
\definecolor{RoyBlue}{RGB}{22, 46, 69}
\definecolor{RoyGrey}{RGB}{64, 88, 128} 

\newcommand{\hlme}[1]{{\color{red}\bf #1}} % highlihgt me

\setbeamercolor{structure}{fg=RoyBlue} % itemize, enumerate, etc
\setbeamercolor{frametitle}{fg=RoyGrey}
\setbeamercolor{section in head/foot}{bg=RoyBlue}


%======================= add progress dots to headline =======================
\setbeamertemplate{headline}{%
    \begin{beamercolorbox}[ht=4mm,dp=4mm]{section in head/foot}
        \insertnavigation{\paperwidth}
    \end{beamercolorbox}%
}%
\makeatother


%======================= add section title page =======================
\AtBeginSection[]{
  \begin{frame}
  \vfill
  \centering
    \usebeamerfont{title}\insertsectionhead\par%
  \vfill
  \end{frame}
}


%================================== TITLEPAGE ==================================
\title{Simultaneous PDF+$\alpha_s$ fits: quick status update}
\date{NNPDF meeting, 11 April 2022, Amsterdam}
\author{Roy Stegeman}
\institute{University of Milan and INFN Milan}
\titlegraphic{\vspace*{6mm}
    \includegraphics[height=0.8cm]{logos/LOGO-ERC.jpg} \hspace{10mm}
	\includegraphics[height=0.8cm]{logos/n3pdflogo_noback.png} \hspace{10mm}
	\includegraphics[height=0.6cm]{logos/nnpdf_logo_official.pdf} \hspace{10mm}
	\includegraphics[height=0.8cm]{logos/Logo_Università_degli_Studi_di_Milano(not_mandatory).png}
	\includegraphics[height=0.8cm]{logos/INFN_logo.png}
    \vspace*{5mm} \\
	\centering{ 
	\fontsize{7.0pt}{0.0pt}\selectfont This project has received funding from the European Union’s Horizon 2020 \\	
    \vspace*{-1mm}
	research and innovation programme under grant agreement No 740006.
	}
}




%================================== SLIDES ==================================

\begin{document}
{
\setbeamertemplate{headline}{} % remove headline from titlepage
\begin{frame}
  \titlepage
\end{frame}
}




\section*{Previous status}

\begin{frame}[t]{Status at last PC discussion - \underline{\href{https://vp.nnpdf.science/AkI8zEchR4qU5tn7zxUZ0g==/}{validphys report}}}
  NNPDF3.1-like dataset w/o nucl. uncertainties\\\vspace*{0.5em}

  High training-validation losses\\
  % and non-homogenous distribution of $\alpha_s$ in the tr-vl plot \\
  $\rightarrow$ problem with optimization\\\vspace*{0.5em}

  Two peaks in the $\alpha_s$ distribution \\
  $\rightarrow$ peak at lower $\alpha_s$ corresponds to worse $\chi^2$ \\
  $\rightarrow$ problem with optimization

  \vspace*{-8mm}
  \includegraphics[width=0.32\textwidth]{PDFnormalize0_Basespecs0_PDFscalespecs0_plot_pdfs_g_prev.pdf}
  \includegraphics[width=0.32\textwidth]{alphas_hist_prev.pdf}
  \includegraphics[width=0.32\textwidth]{plot_training_validation_prev.pdf}

  \textbf{Idea:} freeze $\alpha_s$ during the initial training to improve stability and then small learning rate

\end{frame}





\section*{Freezing $\alpha_s$}

\begin{frame}[t]{Freezing $\alpha_s$ - \underline{\href{https://vp.nnpdf.science/J-wRSP3IRxeAMYoImWktBQ==/}{validphys report}}}

  $\alpha_s$ distribution only a single peak - though small sample size and coarse binning\\\vspace*{0.5em}
  
  tr-vl losses improved and no apparent correlations between loss and $\alpha_s$ value\\\vspace*{0.5em}

  Validation loss of 2.33 compared to 2.31 for fit with $\alpha_s$ fixed at 0.119\\\vspace*{0.5em}

  Correlated replica method found $\alpha_s= 0.11888 \pm 0.00053$\\\vspace*{0.5em}

  Gluon uncertainty decreased wrt the fit without early freezing of $\alpha_s$

  \vspace*{-8mm}
  \includegraphics[width=0.32\textwidth]{PDFnormalize0_Basespecs0_PDFscalespecs0_plot_pdfs_g_frac75.pdf}  
  \includegraphics[width=0.32\textwidth]{alphas_hist_frac75.pdf}
  \includegraphics[width=0.32\textwidth]{plot_training_validation_frac75.pdf}

\end{frame}



\section*{Freezing $\alpha_s$}

\begin{frame}[t]{Freezing $\alpha_s$ with training frac 0.50 - \underline{\href{https://vp.nnpdf.science/UklUxd06R7iHjZULhKPkHQ==/}{validphys report}}}

  Validation loss 2.33 compared to 2.32 for fit with $\alpha_s$ fixed at 0.118 (training fraction did not significantly affect the losses of the fit with fixed alphas: \underline{\href{https://vp.nnpdf.science/K9yvnY0wQGuUSmj2wsRXNQ==/}{validphys report}})\\\vspace*{0.5em}

  $\alpha_s$ values changed as the training fraction changed from 0.75 to 0.50\\\vspace*{0.5em}

  Conclusion: getting there but more things still to be done\\
  - check impact of initial $\alpha_s$, \\
  - understand the tr-vl plot at training fraction 0.75, \\
  - \ldots

  \vspace*{-8.5mm}
  \includegraphics[width=0.32\textwidth]{PDFnormalize0_Basespecs0_PDFscalespecs0_plot_pdfs_g_frac50.pdf}  
  \includegraphics[width=0.32\textwidth]{alphas_hist_frac50.pdf}
  \includegraphics[width=0.32\textwidth]{plot_training_validation_frac50.pdf}

\end{frame}



% 
\appendix

\section{Backup}
\begin{frame}{K-folding}
\begin{figure}[t]
  \centering
  \begin{tikzpicture}[node distance = 1.0cm]\small
    \node[roundtext, fill=green!30] (hyperopt) {\texttt{hyperopt}};
    \coordinate [above = 1.5cm of hyperopt] (abovehyperopt) {};

    \node[roundtext, right = 2.5cm of abovehyperopt] (xplain) {Generate new hyperparameter configuration};
    \draw[myarrow] (hyperopt) -- (abovehyperopt) -- (xplain);

    \coordinate [below = 1.85cm of xplain.west] (fold4v) {};
    \coordinate [below = 1.85cm of xplain.east] (fold1v) {};
    \coordinate (arrowcenter) at ($(fold4v)!0.5!(fold1v)$);
    \coordinate (fold3v) at ($(fold4v)!0.66!(arrowcenter)$);
    \coordinate (fold2v) at ($(arrowcenter)!0.33!(fold1v)$);

    \node [roundtext, fill=green!30, above = 0.33cm of arrowcenter] (fitto) {Fit to subset of folds};
    \draw[thick] (xplain) -- (fitto);

    \draw[thick] (fold4v) -- (fold1v);
    \draw[thick] (fitto) -- (arrowcenter);

    \node[roundtext, below = 0.4cm of fold4v] (fold4) {folds 1,2,3};
    \node[roundtext, below = 0.4cm of fold1v] (fold1) {folds 2,3,4};
    \node[roundtext, below = 0.4cm of fold3v] (fold3) {folds 1,2,4};
    \node[roundtext, below = 0.4cm of fold2v] (fold2) {folds 1,3,4};
    \draw[myarrow] (fold1v) -- (fold1);
    \draw[myarrow] (fold4v) -- (fold4);
    \draw[myarrow] (fold2v) -- (fold2);
    \draw[myarrow] (fold3v) -- (fold3);

    \node[roundtext, fill=green!30, below = 0.30cm of fold4] (chi24) {$\chi^{2}_{4}$};
    \node[roundtext, fill=green!30, below = 0.30cm of fold3] (chi23) {$\chi^{2}_{3}$};
    \node[roundtext, fill=green!30, below = 0.30cm of fold2] (chi22) {$\chi^{2}_{2}$};
    \node[roundtext, fill=green!30, below = 0.30cm of fold1] (chi21) {$\chi^{2}_{1}$};

    \draw[thick] (fold1) -- (chi21);
    \draw[thick] (fold2) -- (chi22);
    \draw[thick] (fold3) -- (chi23);
    \draw[thick] (fold4) -- (chi24);

    \coordinate [below = 0.3cm of chi24] (below4) {};
    \coordinate [below = 0.3cm of chi21] (below1) {};
    \coordinate [below = 0.3cm of chi22] (below2) {};
    \coordinate [below = 0.3cm of chi23] (below3) {};

    \draw[thick] (below1) -- (below4);
    \draw[thick] (chi24) -- (below4);
    \draw[thick] (chi23) -- (below3);
    \draw[thick] (chi22) -- (below2);
    \draw[thick] (chi21) -- (below1);

    \coordinate (belowcenter) at ($(below4)!0.5!(below1)$);
    \node[operations, below = 0.5cm of belowcenter] (loss) {$L = \frac{1}{4}\displaystyle\sum^{4}_{i}\chi^{2}_{i}$};
    \draw[myarrow] (belowcenter) -- (loss);
    \path let \p1 = (hyperopt), \p2 = (loss)
      in coordinate (lleft) at (\x1,\y2);

    \draw[myarrow] (loss) -- (lleft) -- (hyperopt);

  \end{tikzpicture}
\end{figure}
\end{frame}



\begin{frame}[t]{Self-correlation of PDF sets}
The PDF$_i$-PDF$_j$ correlation for a given flavour is defined as
$$corr_{i,j}(x)= \frac{\sum_{n=1}^{N_{rep}} (f_{i,n}(x) -  f_{i,0}(x) )(f_{j,n}(x) -  f_{j,0}(x)) }{\sqrt{\sum_{n=1}^{N_{rep}}(f_{i,n}(x) - f_{i,0}(x))^2} \sqrt{\sum_{n=1}^{N_{rep}}(f_{j,n}(x) -  f_{j,0}(x))^2}}$$
where $f_{i,n}(x)$ is a PDF replica, and $n=0$ corresponds to the central value of the PDF set. 

\end{frame}

\begin{frame}[t]{Correlated combination of PDFs}

When combining PDF sets $f_i$ in a correlated way, for each momentum fraction $x$ and flavour, the central value is calculated using
$$
\left\langle f\right\rangle_{comb}=\sum_{i=1}^{N_{sets}} w_{i} f_{i,0}
$$

and the variance using
$$
V_{comb}=\sum_{i, j=1}^{N_{sets}} w_{i} \sigma_{i j} w_{j}
$$

where $\sigma$ is the covariance matrix, $f_{0,i}$ is the central value of PDF set $i$, and $w_i$ are the weights 
$$
w_{i}=\frac{\sum_{j=1}^{N_{sets}}\left(\sigma^{-1}\right)_{i j}}{\sum_{k, l=1}^{N_{sets}}\left(\sigma^{-1}\right)_{k l}}
$$

\end{frame}

\end{document}
