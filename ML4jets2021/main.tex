\documentclass[aspectratio=169]{beamer}
\graphicspath{{figures/}} % Setting the graphicspath

% Theme settings
\usetheme{Madrid}
\usecolortheme{default}
\setbeamertemplate{navigation symbols}{}   % removes navigation symbols such as 'next page'
\setbeamertemplate{footline}{}             % remove line with name, date, page nr. 
\setbeamercolor*{frametitle}{bg=white}     % remove background from frametitle


%======================= import packages =======================
\usepackage{pifont}       % Pi fonts (Digbats, symbol, etc.)
\usepackage{graphicx}     % More options for \includegraphics



%======================= page numbering =======================
\addtobeamertemplate{navigation symbols}{}{ \usebeamerfont{footline}
  \insertframenumber / \inserttotalframenumber \hspace*{2mm} \\ \vspace*{1mm} 
}


%=================================== colors ===================================
\definecolor{RoyBlue}{RGB}{22, 46, 69}
\definecolor{RoyGrey}{RGB}{64, 88, 128} 

\setbeamercolor{structure}{fg=RoyBlue} % itemize, enumerate, etc
\setbeamercolor{frametitle}{fg=RoyGrey}
 \setbeamercolor{section in head/foot}{bg=RoyBlue}


%======================= add progress dots to headline =======================
\setbeamertemplate{headline}{%
    \begin{beamercolorbox}[ht=4mm,dp=4mm]{section in head/foot}
        \insertnavigation{\paperwidth}
    \end{beamercolorbox}%
}%
\makeatother



%=================================== titlepage ===================================
\title{Towards a new generation of PDFs using ML}
\date{ML4Jets2021, 6 July 2021}
\author{Roy Stegeman}
\institute{University of Milan and INFN Milan}
\titlegraphic{\vspace*{6mm}
    \includegraphics[height=0.8cm]{LOGO-ERC.jpg} \hspace{10mm}
	\includegraphics[height=0.8cm]{n3pdflogo_noback.png} \hspace{10mm}
	\includegraphics[height=0.6cm]{nnpdf_logo_official.pdf} \hspace{10mm}
	\includegraphics[height=0.8cm]{Logo_Università_degli_Studi_di_Milano(not_mandatory).png}
	\includegraphics[height=0.8cm]{INFN_logo.png}
    \vspace*{5mm} \\
	\centering{ 
	\fontsize{7.0pt}{0.0pt}\selectfont This project has received funding from the European Union’s Horizon 2020 \\	
    \vspace*{-1mm}
	research and innovation programme under grant agreement No 740006.
	}
}







\begin{document}


{
\setbeamertemplate{headline}{} % remove headline from titlepage
\begin{frame}
  \titlepage
\end{frame}
}


\section{Introduction}


\begin{frame}{PDFs as an ML problem}
    \begin{itemize}
        \item cannot be obtained from theory $\rightarrow$ inferred from data
        \item well defined input(x)-output(obs)
        \item show general NNPDF methodology
    \end{itemize}
\end{frame}


\begin{frame}{NNPDF3.1 methodology}
    \begin{itemize}
        \item GA
        \item custom c++ code
        \item manual parameter tuning
    \end{itemize}
\end{frame}



\begin{frame}{Need for modernization}
With modern ML techniques available, we would like to apply them to our problem
    \begin{itemize}
        \item reduce fit time
        \item tuning the methodology
        \item solve by using tf backend
        \item solve by making code modular
    \end{itemize}
\end{frame}


\section{NNPDF4.0 methodology}

\begin{frame}{NNPDF4.0 model}
    \begin{itemize}
        \item show the n3fit model
    \end{itemize}
\end{frame}


\begin{frame}{Improved performance}
    \begin{itemize}
        \item easier development
        \item faster, modularity allows use of new technologies
        \item conclusion: faster research, able to test many parameter configurations
    \end{itemize}
\end{frame}


\begin{frame}{Fitting the methodology}
    \begin{itemize}
        \item reduction of of human bias
        \item faster, modularity allows use of new technologies
        \item conclusion: faster research, able to test many parameter configurations
    \end{itemize}
\end{frame}


\begin{frame}{K-folding}
    \begin{itemize}
        \item show figure
    \end{itemize}
\end{frame}


\begin{frame}{Hyperparameter scan}
    \begin{itemize}
        \item show figure
    \end{itemize}
\end{frame}



\section{PDF validation}

\begin{frame}{Trusting uncertainties outside the data region}
    \begin{itemize}
        \item Smaller uncertainties because of new methodology
        \item Accuracy of uncertainties confirmed by experimental data in data ergion
        \item Ideally use data in the extrapolation region
        \item Chronologically 'future test'
    \end{itemize}
\end{frame}


\begin{frame}{Future test}
    \begin{itemize}
        \item show that chi2 decreases to value of just above 1
    \end{itemize}
\end{frame}



\section{Methodology correlations}

\begin{frame}{Self-correlation of PDF sets}
      \begin{itemize}
        \item explain concept of self-correlation
    \end{itemize}
\end{frame}


\section{Conclusions}


\begin{frame}{Summary}
    \begin{itemize}
        \item Faster and more stable results
        \item automatic parameter selection
        \item small uncertainties truthful(?)
        \item NNPDF code will be made publicly available with documentation
    \end{itemize}
\end{frame}





















\end{document}
