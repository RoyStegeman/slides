\documentclass[aspectratio=169,9pt]{beamer}
\graphicspath{{figures/}} % Setting the graphicspath

% Theme settings
\usetheme{Madrid}
\usecolortheme{default}
\setbeamertemplate{navigation symbols}{}   % removes navigation symbols such as 'next page'
\setbeamertemplate{footline}{}             % remove line with name, date, page nr. 
\setbeamercolor*{frametitle}{bg=white}     % remove background from frametitle
\usepackage{caption}
% \captionsetup[figure]{labelformat=empty}% redefines the caption setup of the figures environment in the beamer class.
\setbeamersize{text margin left=20pt,text margin right=10pt}

\usefonttheme[onlymath]{serif} % makes beamer math look like article math


%======================= import packages =======================================
\usepackage{pifont}       % Pi fonts (Digbats, symbol, etc.)
\usepackage{graphicx}     % More options for \includegraphics
\usepackage{tikz}
\usepackage{appendixnumberbeamer} % separate appendix numbering
\usepackage{booktabs}
\usepackage{hyperref}
\usepackage{tabularx}
\usepackage{amsmath, nccmath}
\usepackage{xcolor}
\usepackage[absolute,overlay]{textpos} % for texblock




%======================= page numbering =======================
\addtobeamertemplate{navigation symbols}{}{ \usebeamerfont{footline}
  \insertframenumber / \inserttotalframenumber \hspace*{2mm} \\ \vspace*{1mm} 
}


%=================================== colors ====================================
\definecolor{RoyBlue}{RGB}{22, 46, 69}
\definecolor{RoyGrey}{RGB}{64, 88, 128} 

\newcommand{\hlme}[1]{{\color{red}\bf #1}} % highlight me

\setbeamercolor{structure}{fg=RoyBlue} % itemize, enumerate, etc
\setbeamercolor{frametitle}{fg=RoyGrey}
 \setbeamercolor{section in head/foot}{bg=RoyBlue}


%======================= add progress dots to headline =========================
\setbeamertemplate{headline}{%
    \begin{beamercolorbox}[ht=4mm,dp=4mm]{section in head/foot}
        \insertnavigation{\paperwidth}
    \end{beamercolorbox}%
}%
\makeatother


%======================= add section title page ================================
\AtBeginSection[]{
  \begin{frame}
  \vfill
  \centering
    \usebeamerfont{title}\insertsection\par%
  \vfill
  \end{frame}
}


%=================================== titlepage =================================
\title{The NNPDF4.0 global analysis of the proton structure}
\date{ICHEP 2022, 7 July 2022}
\author{Roy Stegeman}
\institute{University of Milan and INFN Milan}
\titlegraphic{\vspace*{6mm}
    \includegraphics[height=0.8cm]{logos/LOGO-ERC.jpg} \hspace{10mm}
	\includegraphics[height=0.8cm]{logos/n3pdflogo_noback.png} \hspace{10mm}
	\includegraphics[height=0.6cm]{logos/nnpdf_logo_official.pdf} \hspace{10mm}
	\includegraphics[height=0.8cm]{logos/Logo_Università_degli_Studi_di_Milano(not_mandatory).png}
	\includegraphics[height=0.8cm]{logos/INFN_logo.png}
    \vspace*{5mm} \\
	\centering{ 
	\fontsize{7.0pt}{0.0pt}\selectfont This project has received funding from the European Union’s Horizon 2020 \\	
    \vspace*{-1mm}
	research and innovation programme under grant agreement No 740006.
	}
}

\defbeamertemplate{title page}{noinstitute}[1][]
{
  \vbox{}
  \vfill
  \begingroup
    \centering
    \begin{beamercolorbox}[sep=8pt,center,#1]{title}
      \usebeamerfont{title}\inserttitle\par%
      \ifx\insertsubtitle\@empty%
      \else%
        \vskip0.25em%
        {\usebeamerfont{subtitle}\usebeamercolor[fg]{subtitle}\insertsubtitle\par}%
      \fi%     
    \end{beamercolorbox}%
    \vskip1em\par
    \begin{beamercolorbox}[sep=0pt,center,#1]{author}
      \usebeamerfont{author}\insertauthor
    \end{beamercolorbox}
	\begin{beamercolorbox}[sep=0pt,center,#1]{author}
		\usebeamerfont{institute}\insertinstitute
	\end{beamercolorbox}
	\vspace*{8pt}
	\begin{beamercolorbox}[sep=0pt,center,#1]{author}
		On behalf of the NNPDF Collaboration \\
        {\small \href{https://arxiv.org/abs/2109.02653}{Eur.Phys.J.C 82 (2022);\quad arXiv:2109.02653}}
	\end{beamercolorbox}
	\vspace*{16pt}
    \begin{beamercolorbox}[sep=0pt,center,#1]{date}
      \usebeamerfont{date}\insertdate
    \end{beamercolorbox}\vskip0.5em
    {\usebeamercolor[fg]{titlegraphic}\inserttitlegraphic\par}
  \endgroup
  \vfill
}

\makeatletter
\setbeamertemplate{title page}[noinstitute][colsep=-4bp,rounded=true,shadow=\beamer@themerounded@shadow]
\makeatother




\definecolor{Red}{rgb}{1,0,0}
\definecolor{Green}{rgb}{0,1,0}
\definecolor{Blue}{rgb}{0,0,1}
\definecolor{Gray}{gray}{0.9}
\definecolor{springgreen}   {cmyk}{0.26, 0   , 0.76, 0   }
\definecolor{olivegreen}    {cmyk}{0.64, 0   , 0.95, 0.40}
\definecolor{emerald}       {cmyk}{1   , 0   , 0.50, 0   }
\definecolor{junglegreen}   {cmyk}{0.99, 0   , 0.52, 0   }
\definecolor{seagreen}      {cmyk}{0.69, 0   , 0.50, 0   }
\definecolor{green}         {cmyk}{1   , 0   , 1   , 0   }
\definecolor{forestgreen}   {cmyk}{0.91, 0   , 0.88, 0.12}
\definecolor{pinegreen}     {cmyk}{0.92, 0   , 0.59, 0.25}
\definecolor{sepia}         {cmyk}{0   , 0.83, 1   , 0.70}
\definecolor{cerulean}      {cmyk}{0.94, 0.11, 0   , 0   }
\definecolor{salmon}        {cmyk}{0   , 0.53, 0.38, 0   }
\definecolor{greenyellow}   {cmyk}{0.15, 0   , 0.69, 0   }
\definecolor{arsenic}       {rgb}{0.23, 0.27, 0.29}
\definecolor{britishracinggreen}{rgb}{0.0, 0.26, 0.15}
\definecolor{oxfordblue}{rgb}{0.0, 0.13, 0.28}
\definecolor{bostonuniversityred}{rgb}{0.8, 0.0, 0.0}
\definecolor{goldenyellow}{rgb}{1.0, 0.87, 0.0}

\definecolor{darkgreen}{rgb}{0.0, 0.5, 0.13}
\definecolor{darkred}{rgb}{0.55, 0.0, 0.0}
\newcommand{\gct}{\color{darkgreen}\checkmark}
\newcommand{\rma}{\color{red}\ding{55}}
\newcommand{\bct}{\color{blue}\checkmark}
\newcommand{\arrowdownunder}{\begin{center}$\big\downarrow$\end{center}\vspace{-0.3cm}}
\newcommand{\mycolutitle}[1]{\vspace{-0.7cm}\begin{center}#1\end{center}\vspace{-0.1cm}}



%======================= watermark (ICHEP talk only) ===========================
\tikzset{near start abs/.style={xshift=10cm}}




\begin{document}
{
\setbeamertemplate{headline}{} % remove headline from titlepage
\begin{frame}
  \titlepage
\end{frame}
}


%======================= tikz settings =========================================
\usetikzlibrary{shapes, arrows}
\usetikzlibrary{decorations.pathreplacing}
\usetikzlibrary{positioning, calc}
\tikzstyle{fitted} = [rectangle, minimum width=5cm, minimum height=1cm, text centered, draw=black, fill=red!30]
\tikzstyle{operations} = [rectangle, rounded corners, minimum width=2cm,text centered, draw=black, fill=red!30]
\tikzstyle{roundtext} = [rectangle, rounded corners, minimum width=2cm, minimum height=0.8cm, text centered, draw=black, fill=red!30]
\tikzstyle{n3py} = [rectangle, rounded corners, minimum width=3cm, minimum height=1cm, text centered, draw=black, fill=green!30]
\tikzstyle{myarrow} = [thick,->,>=stealth]
\tikzstyle{line} =[draw, -latex']
\tikzstyle{decision} = [diamond, draw, fill=red!20, text width=7.5em, text centered,  inner sep=0pt, minimum height=2em, aspect=4]
\tikzstyle{cloud} = [draw, ellipse,fill=green!20, minimum height=2em]
\tikzstyle{inout} = [rectangle, draw, fill=green!20, text width=9.5em, text centered, rounded corners, minimum height=2em, minimum width=10em]
\tikzstyle{block}=[rectangle, draw, fill=blue!20, text width=9.5em, 
                   text centered, rounded corners, minimum height=2em, 
                   minimum width=10em]
\tikzstyle{arrow} = [thick,->,>=stealth]

\pgfdeclarelayer{bg}    % declare background layer
\pgfsetlayers{bg,main}  % set the order of the layers (main is the standard layer)



% INTRO ========================================================================
\section*{NNPDF4.0}

\begin{frame}{Status of modern PDF sets}

	\begin{center}
        \begin{tikzpicture}
            \node[anchor=south west,inner sep=0] (image) at (0,0) {
                \includegraphics[width=0.45\textwidth]{ZH_xsec_wrong}\includegraphics[width=0.45\textwidth]{HttbarH_xsec_wrong}
                };
                \begin{scope}[x={(image.south east)},y={(image.north west)}]
                    \node[align=center, text=gray,font=\bfseries\fontsize{30}{0}\selectfont] at (0.5,0.7) {BUGGED FIGURES};
                \end{scope}
        \end{tikzpicture}
        % \includegraphics[width=0.45\textwidth]{ZH_xsec_wrong}
		% \includegraphics[width=0.45\textwidth]{HttbarH_xsec_wrong}
	\end{center}
    Plots as shown in Snowmass 2021, 2203.13923\textbf{v1}\\
    Error in the calculation of correlations for Monte Carlo PDFs!
\end{frame}


\begin{frame}{Status of modern PDF sets}
    PDF predictions are consistent but with different uncertainties
	\begin{center}
		\includegraphics[width=0.45\textwidth]{ZH_xsec_fixed.png}
		\includegraphics[width=0.45\textwidth]{HttbarH_xsec_fixed.png}
	\end{center}
    \begin{center}
	    \textbf{How is the improved precision from NNPDF3.1 to NNPDF4.0 achieved?}
	\end{center}
\end{frame}



% DATA ========================================================================
\section{Data}

\begin{frame}{Data from NNPDF1.0 to NNPDF4.0}
	\begin{center}
		\includegraphics[width=0.5\textwidth]{NNPDF_data_history.pdf}
	\end{center}
	The number of datasets -- normally corresponding to different processes -- is generally more relevant than the number of datapoints
\end{frame}


\begin{frame}{Experimental data in NNPDF4.0}
    \begin{columns}
        \column{0.7\linewidth}
            \includegraphics[width=1.0\textwidth]{Markers0_plot_xq2}
        \column{0.25\linewidth}
            New processes:
            \begin{itemize}
                \item direct photon
                \item single top
                \item dijets
                \item W+jet
                \item DIS jet
            \end{itemize}
    \begin{block}{\footnotesize Theoretical improvement}
    {\footnotesize
    Nuclear uncertainties are included
    }
    \end{block}
    \end{columns}
\end{frame}



% METHODOLOGY ==================================================================
\section{Methodology}

\begin{frame}[t]{Improved fitting methodology}
    \begin{columns}[T]
        \begin{column}{0.48\textwidth}
            \begin{itemize}
                \item Improved implementation of physical constraints
                \begin{itemize}
                    \item[-] PDF positivity
                    \item[-] Integrability of non-singlet distributions(Gottfried sum rules)
                \end{itemize}
                \item New fitting code
                \begin{itemize}
                    \item[-] Optimization based on Stochastic Gradient Descent using TensorFlow
                    \item[-] Automated selection of model hyperparameters
                    \item[-] Modular python code
                \end{itemize}
                \item Extended validation of PDFs
                \begin{itemize}
                    \item[-] Closure tests to validate the uncertainty in the data region
                    \item[-] Future tests for the extrapolation region
                    \item[-] Explicit check of basis independence
                \end{itemize}
            \end{itemize}
        \end{column}
        \begin{column}{0.48\textwidth}
            \includegraphics[width=1.0\textwidth]{NNarch}
            \begin{equation*}
                f_{i}\left(x, Q_{0}\right)=x^{-\alpha_{i}}(1-x)^{\beta_{i}} \mathrm{NN}_{i}(x)
            \end{equation*}
        \end{column}
    \end{columns}
\end{frame}



% VALIDATION ===================================================================
\section{Validation}


\begin{frame}[t]{Closure test}{See \href{https://arxiv.org/pdf/2103.08606.pdf}{\color{blue}Eur.Phys.J.C 82 (2022); arxiv:2111.05787}}
    Closure test of a known input assumption
    \begin{enumerate}
        \item Assume a ``true'' underlying PDF (e.g. a single PDF replica)
        \item Produce data distributed according to the experimental covariance matrices
        \item Perform a fit to this data
    \end{enumerate}
    \vspace*{1em}
    Example of statistical extimator:

    \begin{itemize}
        \item Bias: squared difference between central value and true observable\\
        Variance: variance of the model predictions\\
        faithful uncertainties require $E[\textrm{bias}]=\textrm{variance}$
        \item Is truth within one sigma 68\% of cases?
    \end{itemize}
    \vspace*{1em}
    Requires many fits - now possible with the new code
    \begin{textblock*}{\textwidth}(10cm,6cm) % {block width} (coords)
        \includegraphics[width=5cm]{closure_results}
    \end{textblock*}
\end{frame}



\begin{frame}[t]{Future tests}{See \href{https://arxiv.org/pdf/2111.05787.pdf}{\color{blue}Acta Phys.Polon.B 52 (2021) arxiv:2103.08606}}

    \begin{columns}
        \column{0.60\linewidth}
        \begin{enumerate}
            \item Take a historic dataset \\ e.g. pre-HERA or pre-LHC
            \item Perform fit
            \item Compare predictions to ``future'' data
        \end{enumerate}
        \centering
        \includegraphics[width=0.7\textwidth]{kincov}
        \column{0.46\linewidth}
        \vspace{-1.7cm}
        \only<1>{
            \begin{table}
                \tiny
                \centering
                \caption*{\scriptsize $\chi^{2}/N$ (only exp. covmat)}
                \begin{tabular}{c | c c c} \toprule
                    (dataset) & NNPDF4.0 & pre-LHC & pre-Hera  \\ \midrule
                    pre-HERA  & 1.09 & 1.01 & 0.90 \\
                    pre-LHC   & 1.21 & 1.20 & \hlme{23.1} \\
                    NNPDF4.0  & 1.29 & \hlme{3.30} & \hlme{23.1} \\
                    \bottomrule
                \end{tabular}
            \end{table}
        }
        \only<2>{
            \begin{table}
                \tiny
                \centering
                \caption*{\scriptsize $\chi^{2}/N$ (exp. and PDF covmat)}
                \begin{tabular}{c | c c c} \toprule
                    (dataset) & NNPDF4.0 & pre-LHC & pre-Hera  \\ \midrule
                    pre-HERA  &  & & 0.86 \\
                    pre-LHC   &  & 1.17 & \hlme{1.22} \\
                    NNPDF4.0  & 1.12 & \hlme{1.30} & \hlme{1.38} \\
                    \bottomrule
                \end{tabular}
            \end{table}
        }


        \includegraphics[width=0.9\textwidth]{diffu}
    \end{columns}
        \only<2>{\small The total uncertainty increases, and accommodates for difference between predictions and new data.}
\end{frame}


% PHENO ========================================================================
\section{Implications for phenomenology}

\begin{frame}[t]{LHC phenomenology}
    \begin{center}
        Reduced luminosity uncertainties $\rightarrow$ Reduced uncertainty at the level of observables\\
        \vspace*{-0.5em}
        \begin{columns}
            \begin{column}{0.48\textwidth}
                \begin{center}
                    \includegraphics[width=0.78\textwidth]{NNPDF_TTB_14TEV_40_PHENO-internal} 
                \end{center}
            \end{column}
            \begin{column}{0.48\textwidth}
                \includegraphics[width=0.7\textwidth]{NNPDF_H_14TEV_40_PHENO-integrated}\\
                    \includegraphics[width=0.7\textwidth]{NNPDF_TTB_14TEV_40_PHENO-integrated}
            \end{column}
        \end{columns}
    \end{center}
\end{frame}


\begin{frame}{High $M_{ll}$ extrapolation}
    \centering
    \includegraphics[width=0.38\textwidth]{CMS_DY_14TEV_MLL_5000_COSTH}
    \hspace*{0.02\textwidth}
    \includegraphics[width=0.38\textwidth]{CMS_DY_14TEV_MLL_5000_YLL}\\
    Large uncertainties outside the data region due to flexibility of the neural network
\end{frame}



% DELIVERY ====================================================================
\section{Open-source code}
\begin{frame}[t]{The open-source NNPDF code}
    The full NNPDF code has been made public along with user friendly documentation\\
    \vspace*{1em}
    This includes: fitting, hyperoptimization, theory, data processing, visualization\\
    \vspace*{1em}
    It is possible to reproduce all results of NNPDF4.0 and more!\\
    \vspace*{2em}
    \begin{block}{}
        \centering
		\href{https://link.springer.com/article/10.1140/epjc/s10052-021-09747-9}{Eur.Phys.J.C 81 (2021) 10, 958} \\
		\url{https://github.com/NNPDF/nnpdf} \\
		\url{https://docs.nnpdf.science}
    \end{block}
\end{frame}




% CONCLUSION ===================================================================
\section{Summary and Outlook}
\begin{frame}[t]{Summary and Outlook}
    \begin{itemize}
        \item NNPDF4.0 is the latest release in the NNPDF family of PDF sets
        \item 44 new datasets from many new processes are included
        \item Improved methodology with Stochastic Gradient Descent and hyperoptimization
        \item Validation of PDF uncertainties using closure test, future test and parametrization basis independence
        \item[$\Rightarrow$] NNPDF4.0 achieves a high precision over a broad kinematic range
    \end{itemize}
	\vspace*{1em}
    \begin{itemize}
        \item The current level of PDF uncertainties challenges the accuracy of theoretical predictions and demands an increased effort towards the systematic inclusion in the fit of theoretical uncertainties (nuclear, higher orders, SM parameters, \ldots ) and higher-order corrections
    \end{itemize}


    \vspace*{1em}
    \only<2>{
    \begin{center}
        {\Large \textbf{Thank you!}}
    \end{center}
    }
\end{frame}



\appendix

\section{Backup}
\begin{frame}{K-folding}
\begin{figure}[t]
  \centering
  \begin{tikzpicture}[node distance = 1.0cm]\small
    \node[roundtext, fill=green!30] (hyperopt) {\texttt{hyperopt}};
    \coordinate [above = 1.5cm of hyperopt] (abovehyperopt) {};

    \node[roundtext, right = 2.5cm of abovehyperopt] (xplain) {Generate new hyperparameter configuration};
    \draw[myarrow] (hyperopt) -- (abovehyperopt) -- (xplain);

    \coordinate [below = 1.85cm of xplain.west] (fold4v) {};
    \coordinate [below = 1.85cm of xplain.east] (fold1v) {};
    \coordinate (arrowcenter) at ($(fold4v)!0.5!(fold1v)$);
    \coordinate (fold3v) at ($(fold4v)!0.66!(arrowcenter)$);
    \coordinate (fold2v) at ($(arrowcenter)!0.33!(fold1v)$);

    \node [roundtext, fill=green!30, above = 0.33cm of arrowcenter] (fitto) {Fit to subset of folds};
    \draw[thick] (xplain) -- (fitto);

    \draw[thick] (fold4v) -- (fold1v);
    \draw[thick] (fitto) -- (arrowcenter);

    \node[roundtext, below = 0.4cm of fold4v] (fold4) {folds 1,2,3};
    \node[roundtext, below = 0.4cm of fold1v] (fold1) {folds 2,3,4};
    \node[roundtext, below = 0.4cm of fold3v] (fold3) {folds 1,2,4};
    \node[roundtext, below = 0.4cm of fold2v] (fold2) {folds 1,3,4};
    \draw[myarrow] (fold1v) -- (fold1);
    \draw[myarrow] (fold4v) -- (fold4);
    \draw[myarrow] (fold2v) -- (fold2);
    \draw[myarrow] (fold3v) -- (fold3);

    \node[roundtext, fill=green!30, below = 0.30cm of fold4] (chi24) {$\chi^{2}_{4}$};
    \node[roundtext, fill=green!30, below = 0.30cm of fold3] (chi23) {$\chi^{2}_{3}$};
    \node[roundtext, fill=green!30, below = 0.30cm of fold2] (chi22) {$\chi^{2}_{2}$};
    \node[roundtext, fill=green!30, below = 0.30cm of fold1] (chi21) {$\chi^{2}_{1}$};

    \draw[thick] (fold1) -- (chi21);
    \draw[thick] (fold2) -- (chi22);
    \draw[thick] (fold3) -- (chi23);
    \draw[thick] (fold4) -- (chi24);

    \coordinate [below = 0.3cm of chi24] (below4) {};
    \coordinate [below = 0.3cm of chi21] (below1) {};
    \coordinate [below = 0.3cm of chi22] (below2) {};
    \coordinate [below = 0.3cm of chi23] (below3) {};

    \draw[thick] (below1) -- (below4);
    \draw[thick] (chi24) -- (below4);
    \draw[thick] (chi23) -- (below3);
    \draw[thick] (chi22) -- (below2);
    \draw[thick] (chi21) -- (below1);

    \coordinate (belowcenter) at ($(below4)!0.5!(below1)$);
    \node[operations, below = 0.5cm of belowcenter] (loss) {$L = \frac{1}{4}\displaystyle\sum^{4}_{i}\chi^{2}_{i}$};
    \draw[myarrow] (belowcenter) -- (loss);
    \path let \p1 = (hyperopt), \p2 = (loss)
      in coordinate (lleft) at (\x1,\y2);

    \draw[myarrow] (loss) -- (lleft) -- (hyperopt);

  \end{tikzpicture}
\end{figure}
\end{frame}



\begin{frame}[t]{Self-correlation of PDF sets}
The PDF$_i$-PDF$_j$ correlation for a given flavour is defined as
$$corr_{i,j}(x)= \frac{\sum_{n=1}^{N_{rep}} (f_{i,n}(x) -  f_{i,0}(x) )(f_{j,n}(x) -  f_{j,0}(x)) }{\sqrt{\sum_{n=1}^{N_{rep}}(f_{i,n}(x) - f_{i,0}(x))^2} \sqrt{\sum_{n=1}^{N_{rep}}(f_{j,n}(x) -  f_{j,0}(x))^2}}$$
where $f_{i,n}(x)$ is a PDF replica, and $n=0$ corresponds to the central value of the PDF set. 

\end{frame}

\begin{frame}[t]{Correlated combination of PDFs}

When combining PDF sets $f_i$ in a correlated way, for each momentum fraction $x$ and flavour, the central value is calculated using
$$
\left\langle f\right\rangle_{comb}=\sum_{i=1}^{N_{sets}} w_{i} f_{i,0}
$$

and the variance using
$$
V_{comb}=\sum_{i, j=1}^{N_{sets}} w_{i} \sigma_{i j} w_{j}
$$

where $\sigma$ is the covariance matrix, $f_{0,i}$ is the central value of PDF set $i$, and $w_i$ are the weights 
$$
w_{i}=\frac{\sum_{j=1}^{N_{sets}}\left(\sigma^{-1}\right)_{i j}}{\sum_{k, l=1}^{N_{sets}}\left(\sigma^{-1}\right)_{k l}}
$$

\end{frame}


\end{document}
