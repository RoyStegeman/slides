% !TeX root = main.tex
% !TeX spellcheck = en_US

\documentclass[aspectratio=169,9pt]{beamer}

\usepackage{royslides}
\usepackage{graphicx}     % More options for \includegraphics
\graphicspath{{figures/}} % Setting the graphicspath




\title{Quantum advantage using high-dimensional twisted photons as quantum finite automata}
\date{Milan University, 2022}
\author{Roy Stegeman}
\institute{University of Milan and INFN Milan}



% Overview
  % Finite Automata
  % Structured photon
% Finite Automata
% Structured photon
% Photonic Qubits
% 4-Qubit Photonic QFA
% Experiment
% Measurement
% Results 
  % 1-Qubit
  % Multi-Qubit(more efficient than single qubit, lower accepting probability)
% Summary
  % Photonic realization of QFA
  % Strucutred photons as multi-qubit states
  % Demonstrates state efficiency with 2-qubit QFA compared to classical FA (4 states vs 5)



\begin{document}
% TITLEPAGE ====================================================================
{
\setbeamertemplate{headline}{} % remove headline from titlepage
\begin{frame}
  \titlepage


  Based on arXiv:2202.04915

\end{frame}
}


% INTRO ========================================================================

\begin{frame}[t]{Content}
  \begin{itemize}
    \item This paper: detecting prime numbers with QFAs using structured photons
    \item What is QFA?
    \item What are twisted photons?
    \item Experimental setup
    \item Results
  \end{itemize}
\end{frame}


% QUANTUM FINITE AUTOMATA ======================================================
\section{Quantum Finite Automata}
\begin{frame}[t]{Finite State Automaton}
  \begin{itemize}
    \item abstract machine (very simple computational model) that can be in exactly one of a finite number of states at a given time
    \item State can change from one to another as response to inputs
    \item FSA is defined by initial state, list of states and ipnuts that trigger each transition (think elevator or combination lock)
    \item Memory of a FSA is represented by the finite number of states in which a FSA can be
    \item FSA can be deterministic (will explain first for simplicity) or probabilistic (a QFA is probabilistic)
  \end{itemize}
\end{frame}


\begin{frame}[t]{Deterministic Finite Automata}
  \begin{itemize}
    \item $MOD_n=\{a^j|j \mod n \equiv 0 \}$ for $n>1$
    \item decision problem: is the lengtho f an input string a multiple of $n$ or not?
    \item (a) simple example of graph representation
    \item (b) representation of our $MOD_n$ problem (only accepting state is $s_0$):\\
    for each symbol $a$, the DFA performs a transition between the states\\
    so after if the length is a multiple of $n$, the system transitions back to its initial (and only accepting) state. Hence $MOD_n$ langauge.
    \item \textbf{There is no DFA with less than $n$ states for $MOD_n$}

    \includegraphics[width=0.4\textwidth]{DFA.png}\\
    Source: https://www.nature.com/articles/s41534-019-0163-x
  \end{itemize}
\end{frame}


\begin{frame}[t]{Probabilitic Finite Automata}
  \begin{itemize}
    \item can make more than one tranisition with probabilities that sum to one
    \item The input $x=x[1] x[2] \cdots x[l]$ can be traced linearly
    \item at any step the PDFA is in a probability distribution of its states represented by $v=\left(\begin{array}{llll}p_{1} & p_{2} & \cdots & p_{m}\end{array}\right)^{T}$
    \item transformation: $v_{f}=A_{\$} A_{x[l]} A_{x[l-1]} \cdots A_{x[1]} A_{\mathbb{C}} v_{0}$
    \item probablity of accepting $x$ is the sum of probablities corresponding to the accepting states in $v_f$
  \end{itemize}
\end{frame}



\begin{frame}[t]{Quantum Finite Automata}
  \begin{itemize}
    \item Quantum counter part of PFA
    \item QFA with $m$ basis states: $N=\left\{\left|q_{1}\right\rangle, \ldots,\left|q_{m}\right\rangle\right\}$
    \item $\left|v_{f}\right\rangle=U_{\$} U_{x[l]} U_{x[l-1]} \cdots U_{x[1]} U_{\mathbb{C}}\left|v_{0}\right\rangle$
    \item $\sum_{q_{j} \in N_{a}}\left|\left\langle q_{j} \mid v_{f}\right\rangle\right|^{2}$, $N_a$ are the accepting states
    \item \textbf{Very similar to PFA but exponentially smaller than any classical (even randomized) FA (Ambainis and Freivalds, 1998)}
  \end{itemize}
\end{frame}


\begin{frame}[t]{2-state QFA}
  \begin{itemize}
    \item Two basis states: $|0\rangle,|1\rangle$ ($|0\rangle$ is the accepting state)
    \item Identity applied when reading $\$,\mathbb{C}$ 
    \item Input $x=a^n$, each a rotates by $\theta=2\pi/p$, thus reading of $a$
    leads to $U_{a}=\left(\begin{array}{rr}\cos \theta & -\sin \theta \\ \sin \theta & \cos \theta\end{array}\right)$
    \item thus the final state is $\left|v_{f}\right\rangle=\left(\begin{array}{c}\cos n \theta \\ \sin n \theta\end{array}\right)$
    \item Input accpted with $\cos^2(n\theta)$
    \item False acceptance rate
    \item 2d-state QFA reduced false acceptance rate of 2-state QFA
    \item Different acceptance rate for differnet $k$ run in parallel
    \item Eq. (14)
  \end{itemize}
\end{frame}



% EXPERIMENTAL SETUP ===========================================================
\section{Experimental Setup}

\begin{frame}[t]{QFA using photon OAM}
  \begin{itemize}
    \item QFA alorithm for MODp is one of the first quantum alogirhtms with exponental advantage over classical methods
    \item Performing the calculation on a quantum computer bacame random iddediately after reading a few symbols due to relateveily large noise in the copmutation process.
    \vspace*{1em}
    \item LG mode as lab realization of high-dimensional quantum states
    \item all modes are orthoganol
    \item the modes have high-diemnsional states (infintely many)
    \item used to encode single photons with quantum states
  \end{itemize}
  \includegraphics[width=0.2\textwidth]{LG_modes_structured_photons.png}
\end{frame}


\begin{frame}[t]{Photonic Qubit}
  \begin{itemize}
    \item physical realizaiton of a qubit
    \item $|\psi\rangle=\cos(\theta/2)|L_1\rangle + e^{i\phi}\sin(\theta/2)|L_{-1}\rangle$
  \end{itemize}
  \includegraphics[width=0.2\textwidth]{photonic_superposition_qubit.png}
  \includegraphics[width=0.2\textwidth]{photonic_bloch_sphere.png}
\end{frame}

\begin{frame}[t]{Photonic Qubit}
  Superposition correpond to the multiple 2-steate QFA of the theory part(?)
  \includegraphics[width=0.5\textwidth]{4photon_qubit_superposition.png}
\end{frame}


\begin{frame}[t]{Experimental setup}
  \begin{itemize}
    \item Heralded single photons pass through first SLM
    \item Spatial light modulation
  \end{itemize}
  \includegraphics[width=0.3\textwidth]{experimental_setup.png}\\
  Loop that performs the unitary operation on the input state
  Remember eq(14)
\end{frame}


\begin{frame}[t]{Measurement}
  \begin{itemize}
    \item principle used for prime number search
    \item example: prime number 5
  \end{itemize}
  \includegraphics[width=0.3\textwidth]{example_measurement.png}
\end{frame}


% RESULTS ======================================================================
\section{Results}
\begin{frame}[t]{Recognizing MOD\_5}
  \begin{itemize}
    \item Only 4 states gives good results(as opposed to 5 for classical)
  \end{itemize}
  \includegraphics[width=0.3\textwidth]{2states_4states.png}
\end{frame}


\begin{frame}[t]{Recognizing MOD\_11}
  \begin{itemize}
    \item Only 4 states gives good results(as opposed to 5 for classical)
  \end{itemize}
  \includegraphics[width=0.3\textwidth]{4states_8states.png}
\end{frame}


\begin{frame}[t]{Summary}
  \begin{itemize}
    \item Photonic realization of QFA
    \item Structured photons as multi-qubit states
    \item Demonstrates efficiency with 2-qubit QFA compared to classical FA (4 instead of 5 staes / 8 instead of 11 states)
  \end{itemize}
\end{frame}


% 
\appendix

\section{Backup}
\begin{frame}{K-folding}
\begin{figure}[t]
  \centering
  \begin{tikzpicture}[node distance = 1.0cm]\small
    \node[roundtext, fill=green!30] (hyperopt) {\texttt{hyperopt}};
    \coordinate [above = 1.5cm of hyperopt] (abovehyperopt) {};

    \node[roundtext, right = 2.5cm of abovehyperopt] (xplain) {Generate new hyperparameter configuration};
    \draw[myarrow] (hyperopt) -- (abovehyperopt) -- (xplain);

    \coordinate [below = 1.85cm of xplain.west] (fold4v) {};
    \coordinate [below = 1.85cm of xplain.east] (fold1v) {};
    \coordinate (arrowcenter) at ($(fold4v)!0.5!(fold1v)$);
    \coordinate (fold3v) at ($(fold4v)!0.66!(arrowcenter)$);
    \coordinate (fold2v) at ($(arrowcenter)!0.33!(fold1v)$);

    \node [roundtext, fill=green!30, above = 0.33cm of arrowcenter] (fitto) {Fit to subset of folds};
    \draw[thick] (xplain) -- (fitto);

    \draw[thick] (fold4v) -- (fold1v);
    \draw[thick] (fitto) -- (arrowcenter);

    \node[roundtext, below = 0.4cm of fold4v] (fold4) {folds 1,2,3};
    \node[roundtext, below = 0.4cm of fold1v] (fold1) {folds 2,3,4};
    \node[roundtext, below = 0.4cm of fold3v] (fold3) {folds 1,2,4};
    \node[roundtext, below = 0.4cm of fold2v] (fold2) {folds 1,3,4};
    \draw[myarrow] (fold1v) -- (fold1);
    \draw[myarrow] (fold4v) -- (fold4);
    \draw[myarrow] (fold2v) -- (fold2);
    \draw[myarrow] (fold3v) -- (fold3);

    \node[roundtext, fill=green!30, below = 0.30cm of fold4] (chi24) {$\chi^{2}_{4}$};
    \node[roundtext, fill=green!30, below = 0.30cm of fold3] (chi23) {$\chi^{2}_{3}$};
    \node[roundtext, fill=green!30, below = 0.30cm of fold2] (chi22) {$\chi^{2}_{2}$};
    \node[roundtext, fill=green!30, below = 0.30cm of fold1] (chi21) {$\chi^{2}_{1}$};

    \draw[thick] (fold1) -- (chi21);
    \draw[thick] (fold2) -- (chi22);
    \draw[thick] (fold3) -- (chi23);
    \draw[thick] (fold4) -- (chi24);

    \coordinate [below = 0.3cm of chi24] (below4) {};
    \coordinate [below = 0.3cm of chi21] (below1) {};
    \coordinate [below = 0.3cm of chi22] (below2) {};
    \coordinate [below = 0.3cm of chi23] (below3) {};

    \draw[thick] (below1) -- (below4);
    \draw[thick] (chi24) -- (below4);
    \draw[thick] (chi23) -- (below3);
    \draw[thick] (chi22) -- (below2);
    \draw[thick] (chi21) -- (below1);

    \coordinate (belowcenter) at ($(below4)!0.5!(below1)$);
    \node[operations, below = 0.5cm of belowcenter] (loss) {$L = \frac{1}{4}\displaystyle\sum^{4}_{i}\chi^{2}_{i}$};
    \draw[myarrow] (belowcenter) -- (loss);
    \path let \p1 = (hyperopt), \p2 = (loss)
      in coordinate (lleft) at (\x1,\y2);

    \draw[myarrow] (loss) -- (lleft) -- (hyperopt);

  \end{tikzpicture}
\end{figure}
\end{frame}



\begin{frame}[t]{Self-correlation of PDF sets}
The PDF$_i$-PDF$_j$ correlation for a given flavour is defined as
$$corr_{i,j}(x)= \frac{\sum_{n=1}^{N_{rep}} (f_{i,n}(x) -  f_{i,0}(x) )(f_{j,n}(x) -  f_{j,0}(x)) }{\sqrt{\sum_{n=1}^{N_{rep}}(f_{i,n}(x) - f_{i,0}(x))^2} \sqrt{\sum_{n=1}^{N_{rep}}(f_{j,n}(x) -  f_{j,0}(x))^2}}$$
where $f_{i,n}(x)$ is a PDF replica, and $n=0$ corresponds to the central value of the PDF set. 

\end{frame}

\begin{frame}[t]{Correlated combination of PDFs}

When combining PDF sets $f_i$ in a correlated way, for each momentum fraction $x$ and flavour, the central value is calculated using
$$
\left\langle f\right\rangle_{comb}=\sum_{i=1}^{N_{sets}} w_{i} f_{i,0}
$$

and the variance using
$$
V_{comb}=\sum_{i, j=1}^{N_{sets}} w_{i} \sigma_{i j} w_{j}
$$

where $\sigma$ is the covariance matrix, $f_{0,i}$ is the central value of PDF set $i$, and $w_i$ are the weights 
$$
w_{i}=\frac{\sum_{j=1}^{N_{sets}}\left(\sigma^{-1}\right)_{i j}}{\sum_{k, l=1}^{N_{sets}}\left(\sigma^{-1}\right)_{k l}}
$$

\end{frame}

\end{document}





