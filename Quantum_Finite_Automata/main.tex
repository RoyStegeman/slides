% !TeX root = main.tex
% !TeX spellcheck = en_US

\documentclass[aspectratio=169,9pt]{beamer}

\usepackage{royslides}
\usepackage{graphicx}     % More options for \includegraphics
\usepackage{subfig}
\graphicspath{{figures/}} % Setting the graphicspath




\title{Quantum advantage using high-dimensional twisted photons as quantum finite automata}
\date{Milan University, 2022}
\author{Roy Stegeman}
\institute{University of Milan and INFN Milan}


% The "twisting" of a photon is done by introducing orbatial angular momentum (OAM). See: https://opg.optica.org/optica/fulltext.cfm?uri=optica-5-6-682&id=389941
% Structured photons are photons with well-engineerd polarizationa nd spatial and temporal modes. See:  https://onlinelibrary.wiley.com/doi/abs/10.1002/9781119662945.ch14
% In this case structured refers to the fact that we control the LG modes


\begin{document}
% TITLEPAGE ====================================================================
{
\setbeamertemplate{headline}{} % remove headline from titlepage
\begin{frame}
  \titlepage


  Based on arXiv:2202.04915

\end{frame}
}


% INTRO ========================================================================

\begin{frame}[t]{Content}
  \textit{Quantum advantage using high-dimensional twisted photons as quantum finite automata}
  \vspace*{1em}
  \begin{itemize}
    \item What are quantum finite automata (QFA)?
    \item Photons as quantum states
    \item Experimental setup
    \item Results
  \end{itemize}
  \vspace*{5em}
  Unless otherwise specified, figures are taken from:
  \begin{itemize}
    \item arXiv: 2202.04915
    \item ``Multi-Qubit Quantum Finite
    Automata Using Structured
    Photons'', 2021, Thesis, S. Plachta
  \end{itemize}
\end{frame}


% QUANTUM FINITE AUTOMATA ======================================================
\section{Quantum Finite Automata}
\begin{frame}[t]{Finite State Automaton}
  \begin{itemize}
    \item Abstract machine that can be in one of a finite number of states at a given time
    \item Computational model simpler than a Turing machine
    \item FSA is fully defined by initial state, list of states and inputs that trigger each transition
    % \item Memory of a FSA is represented by the finite number of states in which a FSA can be
    \item FSA can be deterministic or probabilistic (a QFA is probabilistic)
  \end{itemize}
  \includegraphics[width=0.35\textwidth]{Automata_hierarchy.png}
  \includegraphics[width=0.4\textwidth]{turnstile.png}
\end{frame}


\begin{frame}[t]{Deterministic Finite Automata for $MOD_p$}
  \begin{itemize}
    \item This talk/paper: prime number search / recognizing $MOD_p$  
    %  recognizing $MOD_n=\{a^j|j \mod n \equiv 0 \}$ for $n>1$
    \item Can also be represented schematically
    % \item Decision problem: is the length of an input string a multiple of $n$?
    % \item See the representation of our $MOD_n$ problem (only accepting state is $s_0$):\\
    % for each symbol $a$, the DFA performs a transition between the states\\
    % so after if the length is a multiple of $n$, the system transitions back to its initial (and only accepting) state. Hence $MOD_n$ langauge.
    \item \textbf{There is no deterministic FA with less than $p$ states for $MOD_p$}
  \end{itemize}
  \vspace*{2em}
  \begin{figure}
    \centering
      \includegraphics[width=0.3\textwidth]{QFA_MODn.png}\\
      \caption*{Schematic representation of the $MOD_p$ language}
  \end{figure}
\end{frame}


\begin{frame}[t]{Quantum Finite Automaton}
  \begin{itemize}
    % \item QFA are a special case of PFA
    \item Can make more than one transition with probabilities that sum to one
    \item QFA algorithm for $MOD_p$ can be more time and memory efficient than classical methods
    % \item The input $x=x[1] x[2] \cdots x[l]$ can be traced linearly
    % \item at any step an $m$-state PFA is in a probability distribution of its states represented by $v=\left(\begin{array}{llll}p_{1} & p_{2} & \cdots & p_{m}\end{array}\right)^{T}$
    % \item transformation: $v_{f}=A_{\$} A_{x[l]} A_{x[l-1]} \cdots A_{x[1]} A_{\mathbb{C}} v_{0}$
    \item Probablity of accepting is the sum of probablities corresponding to the accepting states
  \end{itemize}
  \begin{figure}
    \includegraphics[width=0.6\textwidth]{DFA.png}
    \caption*{image: npj Quantum information 5, 56 (2019)}
  \end{figure}
\end{frame}


% \begin{frame}[t]{Quantum Finite Automata}
%   \begin{itemize}
%     \item Quantum counter part of PFA
%     \item QFA alorithm for MODp is one of the first quantum alogirhtms with exponental advantage over classical methods
%     \item QFA with $m$ basis states: $N=\left\{\left|q_{1}\right\rangle, \ldots,\left|q_{m}\right\rangle\right\}$
%     \item $\left|v_{f}\right\rangle=U_{\$} U_{x[l]} U_{x[l-1]} \cdots U_{x[1]} U_{\mathbb{C}}\left|v_{0}\right\rangle$
%     \item $\sum_{q_{j} \in N_{a}}\left|\left\langle q_{j} \mid v_{f}\right\rangle\right|^{2}$, $N_a$ are the accepting states
%     \item \textbf{Very similar to PFA but exponentially smaller than any classical (even randomized) FA (Ambainis and Freivalds, 1998)}
%   \end{itemize}
% \end{frame}


% \begin{frame}[t]{2-state QFA}
%   \begin{itemize}
%     \item Two basis states: $|0\rangle,|1\rangle$ ($|0\rangle$ is the accepting state)
%     % \item Identity applied when reading $\$,\mathbb{C}$ 
%     \item Input $x=a^n$, each a rotates by $\theta=2\pi/p$, thus reading of $a$
%     leads to $U_{a}=\left(\begin{array}{rr}\cos \theta & -\sin \theta \\ \sin \theta & \cos \theta\end{array}\right)$
%     \item thus the final state is $\left|v_{f}\right\rangle=\left(\begin{array}{c}\cos n \theta \\ \sin n \theta\end{array}\right)$
%     \item Input accepted with $\cos^2(2n\pi/p)$, so 1 if $n=p$
%     \item False acceptance rate up to $\cos^2(\pi/p)$
%     \item 2d-state QFA reduced false acceptance rate of 2-state QFA
%     \item Different acceptance rate for differnet $k$ run in parallel
%     \item Eq. (14)
%   \end{itemize}
% \end{frame}



% QFA using photons ===========================================================
\section{Photons as quantum states}

\begin{frame}[t]{Photon beam}
  \begin{columns}[T]
    \begin{column}[]{0.45\textwidth}
      \begin{itemize}
        \item Performing the calculation on a quantum computer bacame random immediately after reading a few symbols due to relatively large noise in the copmutation process.
        \vspace*{1em}
        \item Photon beam described by paraxial wave equation $\left(\Delta_\perp^2+2ik\frac{\partial}{\partial z}\right)u=0$
        \item Special propagation property: described by radial $p$ and azimuthal $l$ quantum numbers
        \item Laguerre-Gaussian modes $L_p^l$ as lab realization of quantum states
        % \item all modes are orthoganol
        % \item the modes have high-dimensional states (infintely many)
        % \item used to encode single photons with quantum states
      \end{itemize}
    \end{column}
    \begin{column}[]{0.05\textwidth}
    \end{column}
    \begin{column}[]{0.45\textwidth}
      \vspace*{-3em}
      \begin{figure}
        \subfloat[\centering $l=1$, $p=0$]{\includegraphics[width=0.4\textwidth]{l_1_p_0.png}}
        \subfloat[\centering $l=-1$, $p=0$]{\includegraphics[width=0.4\textwidth]{l_-1_p_0.png}}
      \end{figure}
      \begin{figure}
        \subfloat[\centering $l=4$, $p=0$]{\includegraphics[width=0.4\textwidth]{l_4_p_0.png}}
        \subfloat[\centering $l=-1$, $p=2$]{\includegraphics[width=0.4\textwidth]{l_-1_p_2.png}}
      \end{figure}
    \end{column}
  \end{columns}
\end{frame}


\begin{frame}[t]{Photonic qubits}
  \begin{columns}
    \begin{column}[]{0.5\textwidth}
      \begin{itemize}
        \item physical realization of a qubit
        \item block sphere analogue with LG modes: $|\psi\rangle=\cos(\theta/2)|L_1\rangle + e^{i\phi}\sin(\theta/2)|L_{-1}\rangle$
        \item choise of basis states: \\
        $|0\rangle= \frac{1}{\sqrt{2}}(|L_1\rangle + |L_{-1}\rangle)$\\
        $|1\rangle= \frac{1}{\sqrt{2}}(|L_1\rangle - |L_{-1}\rangle)$
      \end{itemize}
      \includegraphics[width=0.99\textwidth]{photonic_superposition_qubit.png}
    \end{column}
    \begin{column}[]{0.5\textwidth}
      \includegraphics[width=0.99\textwidth]{photonic_bloch_sphere.png}
    \end{column}
  \end{columns}
\end{frame}

\begin{frame}[t]{Photonic qubits}
  % Superposition correpond to the multiple 2-steate QFA of the theory part(?)\\
  \includegraphics[width=0.5\textwidth]{4photon_qubit_superposition.png}
  \begin{itemize}
    \item $|0\rangle= \frac{1}{\sqrt{2}}(|L_1\rangle + |L_{-1}\rangle)$
    \item $|1\rangle= \frac{1}{\sqrt{2}}(|L_1\rangle - |L_{-1}\rangle)$
    \item $|2\rangle= \frac{1}{\sqrt{2}}(|L_2\rangle + |L_{-2}\rangle)$
    \item $|3\rangle= \frac{1}{\sqrt{2}}(|L_2\rangle - |L_{-2}\rangle)$
    \item \ldots
  \end{itemize}
  $|\psi\rangle = \frac{1}{{2}}(|0\rangle+|2\rangle+|4\rangle+|6\rangle)$\\
\end{frame}


% EXPERIMENTAL SETUP ===========================================================
\section{Experimental Setup}
\begin{frame}[t]{Experimental setup}
  \begin{columns}[T]
    \begin{column}{0.4\textwidth}
      \begin{itemize}
        \item Single photons
        \item Detector 1 \& time tagger
        \item Spatial Light Modulator (SLM)
        \item 50:50 Beam Splitter (BS)
        \item Dove prism performs the rotation operation
      \end{itemize}
    \end{column}
    \begin{column}{0.6\textwidth}
      \includegraphics[width=0.8\textwidth]{experimental_setup.png}\\
    \end{column}
  \end{columns}
  % Remember eq(14)
\end{frame}


\begin{frame}[t]{Measurement}
  \begin{columns}[T]
    \begin{column}{0.4\textwidth}
      \begin{itemize}
        \item recognize decay as result of 50:50 beamsplitter
        \item principle used for prime number search
        \item example: prime number 5
      \end{itemize}
    \end{column}
    \begin{column}{0.6\textwidth}
      \includegraphics[width=0.9\textwidth]{example_measurement.png}
    \end{column}
  \end{columns}
\end{frame}



\begin{frame}[t]{Acceptance rate}
  \begin{itemize}
    \item Consider two basis states: $|0\rangle,|1\rangle$ ($|0\rangle$ is the accepting state)
    % \item Identity applied when reading $\$,\mathbb{C}$
    \item Input string of $n$ symbols, 
    \item each symbol rotates by $\theta=2\pi/p \Rightarrow U_{a}=\left(\begin{array}{rr}\cos \theta & -\sin \theta \\ \sin \theta & \cos \theta\end{array}\right)$
    \item thus the final state is $\left|v_{f}\right\rangle=\left(\begin{array}{c}\cos n \theta \\ \sin n \theta\end{array}\right)$
    \item Input accepted with $\cos^2(2n\pi/p)$, so probability is 1 if $n$ is multiple of prime number $p$
    \item False acceptance if $n$ is not a multiple of $p$
    \item multiple-state QFA reduces false acceptance rate of 2-state QFA
    % \item Different acceptance rate for differnet $k$ run in parallel
    % \item Eq. (14)
    \includegraphics[width=0.6\textwidth]{2dstate.png}
  \end{itemize}
\end{frame}


% RESULTS ======================================================================
\section{Results}
\begin{frame}[t]{Recognizing $MOD_5$}
  \begin{columns}
    \begin{column}{0.5\textwidth}
      2 states: $|0\rangle$\\
      4 states: $\frac{1}{\sqrt{2}}(|0\rangle + |4\rangle)$\\
      Dove prism rotates by $36^{\circ}$ \\
      $5 \times 36^{\circ} = 180^{\circ}$ (global phase symmetry)
      \includegraphics[width=0.9\textwidth]{multi_qbit_photon_states.png}
    \end{column}
    \begin{column}{0.5\textwidth}
      \includegraphics[width=0.9\textwidth]{2states_4states.png}
    \end{column}
  \end{columns}
  \begin{itemize}
    \item 2 states result in a high false acceptance rate
    \item 4 states gives good results (as opposed to 5 for classical)
  \end{itemize}
\end{frame}


\begin{frame}[t]{Recognizing $MOD_{11}$}
  \begin{itemize}
    \item 70:30 beam splitter to increase the number of loops
  \end{itemize}
  \begin{center}
    \includegraphics[width=0.45\textwidth]{4states_8states.png}
  \end{center}
\end{frame}


% \begin{frame}[t]{Summary}
%   \begin{itemize}
%     \item Photonic realization of QFA
%     \item Structured photons as multi-qubit states
%     \item Demonstrates efficiency with 2-qubit QFA compared to classical FA (4 instead of 5 states / 8 instead of 11 states)
%   \end{itemize}
% \end{frame}


{ % to remove the header for this frame
\makeatletter % to change template
    \setbeamertemplate{headline}[default] % not mandatory, but I though it was better to set it blank
    \def\beamer@entrycode{\vspace*{-\headheight}} % here is the part we are interested in :)
\makeatother
\begin{frame}
  \begin{center}
    \textbf{\Large Thank you!}
  \end{center}
\end{frame}
}





\appendix

\section{Backup}
\begin{frame}{K-folding}
\begin{figure}[t]
  \centering
  \begin{tikzpicture}[node distance = 1.0cm]\small
    \node[roundtext, fill=green!30] (hyperopt) {\texttt{hyperopt}};
    \coordinate [above = 1.5cm of hyperopt] (abovehyperopt) {};

    \node[roundtext, right = 2.5cm of abovehyperopt] (xplain) {Generate new hyperparameter configuration};
    \draw[myarrow] (hyperopt) -- (abovehyperopt) -- (xplain);

    \coordinate [below = 1.85cm of xplain.west] (fold4v) {};
    \coordinate [below = 1.85cm of xplain.east] (fold1v) {};
    \coordinate (arrowcenter) at ($(fold4v)!0.5!(fold1v)$);
    \coordinate (fold3v) at ($(fold4v)!0.66!(arrowcenter)$);
    \coordinate (fold2v) at ($(arrowcenter)!0.33!(fold1v)$);

    \node [roundtext, fill=green!30, above = 0.33cm of arrowcenter] (fitto) {Fit to subset of folds};
    \draw[thick] (xplain) -- (fitto);

    \draw[thick] (fold4v) -- (fold1v);
    \draw[thick] (fitto) -- (arrowcenter);

    \node[roundtext, below = 0.4cm of fold4v] (fold4) {folds 1,2,3};
    \node[roundtext, below = 0.4cm of fold1v] (fold1) {folds 2,3,4};
    \node[roundtext, below = 0.4cm of fold3v] (fold3) {folds 1,2,4};
    \node[roundtext, below = 0.4cm of fold2v] (fold2) {folds 1,3,4};
    \draw[myarrow] (fold1v) -- (fold1);
    \draw[myarrow] (fold4v) -- (fold4);
    \draw[myarrow] (fold2v) -- (fold2);
    \draw[myarrow] (fold3v) -- (fold3);

    \node[roundtext, fill=green!30, below = 0.30cm of fold4] (chi24) {$\chi^{2}_{4}$};
    \node[roundtext, fill=green!30, below = 0.30cm of fold3] (chi23) {$\chi^{2}_{3}$};
    \node[roundtext, fill=green!30, below = 0.30cm of fold2] (chi22) {$\chi^{2}_{2}$};
    \node[roundtext, fill=green!30, below = 0.30cm of fold1] (chi21) {$\chi^{2}_{1}$};

    \draw[thick] (fold1) -- (chi21);
    \draw[thick] (fold2) -- (chi22);
    \draw[thick] (fold3) -- (chi23);
    \draw[thick] (fold4) -- (chi24);

    \coordinate [below = 0.3cm of chi24] (below4) {};
    \coordinate [below = 0.3cm of chi21] (below1) {};
    \coordinate [below = 0.3cm of chi22] (below2) {};
    \coordinate [below = 0.3cm of chi23] (below3) {};

    \draw[thick] (below1) -- (below4);
    \draw[thick] (chi24) -- (below4);
    \draw[thick] (chi23) -- (below3);
    \draw[thick] (chi22) -- (below2);
    \draw[thick] (chi21) -- (below1);

    \coordinate (belowcenter) at ($(below4)!0.5!(below1)$);
    \node[operations, below = 0.5cm of belowcenter] (loss) {$L = \frac{1}{4}\displaystyle\sum^{4}_{i}\chi^{2}_{i}$};
    \draw[myarrow] (belowcenter) -- (loss);
    \path let \p1 = (hyperopt), \p2 = (loss)
      in coordinate (lleft) at (\x1,\y2);

    \draw[myarrow] (loss) -- (lleft) -- (hyperopt);

  \end{tikzpicture}
\end{figure}
\end{frame}



\begin{frame}[t]{Self-correlation of PDF sets}
The PDF$_i$-PDF$_j$ correlation for a given flavour is defined as
$$corr_{i,j}(x)= \frac{\sum_{n=1}^{N_{rep}} (f_{i,n}(x) -  f_{i,0}(x) )(f_{j,n}(x) -  f_{j,0}(x)) }{\sqrt{\sum_{n=1}^{N_{rep}}(f_{i,n}(x) - f_{i,0}(x))^2} \sqrt{\sum_{n=1}^{N_{rep}}(f_{j,n}(x) -  f_{j,0}(x))^2}}$$
where $f_{i,n}(x)$ is a PDF replica, and $n=0$ corresponds to the central value of the PDF set. 

\end{frame}

\begin{frame}[t]{Correlated combination of PDFs}

When combining PDF sets $f_i$ in a correlated way, for each momentum fraction $x$ and flavour, the central value is calculated using
$$
\left\langle f\right\rangle_{comb}=\sum_{i=1}^{N_{sets}} w_{i} f_{i,0}
$$

and the variance using
$$
V_{comb}=\sum_{i, j=1}^{N_{sets}} w_{i} \sigma_{i j} w_{j}
$$

where $\sigma$ is the covariance matrix, $f_{0,i}$ is the central value of PDF set $i$, and $w_i$ are the weights 
$$
w_{i}=\frac{\sum_{j=1}^{N_{sets}}\left(\sigma^{-1}\right)_{i j}}{\sum_{k, l=1}^{N_{sets}}\left(\sigma^{-1}\right)_{k l}}
$$

\end{frame}

\end{document}





