\documentclass[aspectratio=169,10pt]{beamer}
\graphicspath{{figures/}} % Setting the graphicspath

% Theme settings
\usetheme{Madrid}
\usecolortheme{default}
\setbeamertemplate{navigation symbols}{}   % removes navigation symbols such as 'next page'
\setbeamertemplate{footline}{}             % remove line with name, date, page nr.
\setbeamercolor*{frametitle}{bg=white}     % remove background from frametitle
\usepackage{caption}
% \captionsetup[figure]{labelformat=empty}% redefines the caption setup of the figures environment in the beamer class.
\setbeamersize{text margin left=20pt,text margin right=10pt}

\usefonttheme[onlymath]{serif} % makes beamer math look like article math


%======================= import packages =======================
\usepackage{graphicx}     % More options for \includegraphics
\usepackage{appendixnumberbeamer} % separate appendix numbering
\usepackage{url}
\usepackage{hyperref}
\usepackage{booktabs}
\usepackage{xcolor}

\definecolor{amethyst}{rgb}{0.6, 0.4, 0.8} % for the hopscotch chi2 table

%======================= page numbering =======================
\addtobeamertemplate{navigation symbols}{}{ \usebeamerfont{footline}
  \insertframenumber / \inserttotalframenumber \hspace*{2mm} \\ \vspace*{1mm}
}


\newcommand{\chitwo}{$\chi^2$}

%=================================== colors ===================================
\definecolor{RoyBlue}{RGB}{22, 46, 69}
\definecolor{RoyGrey}{RGB}{64, 88, 128}

\newcommand{\hlme}[1]{{\color{red}\bf #1}} % highlihgt me

\setbeamercolor{structure}{fg=RoyBlue} % itemize, enumerate, etc
\setbeamercolor{frametitle}{fg=RoyGrey}
\setbeamercolor{section in head/foot}{bg=RoyBlue}


%======================= add progress dots to headline =======================
\setbeamertemplate{headline}{%
    \begin{beamercolorbox}[ht=4mm,dp=4mm]{section in head/foot}
        \insertnavigation{\paperwidth}
    \end{beamercolorbox}%
}%
\makeatother


%======================= add section title page =======================
\AtBeginSection[]{
  \begin{frame}
  \vfill
  \centering
    \usebeamerfont{title}\insertsectionhead\par%
  \vfill
  \end{frame}
}


%================================== TITLEPAGE ==================================
\title{Hopscotch: a quick recap}
\date{NNPDF meeting, 30 August 2022, Gargnano}
\author{Roy Stegeman}
\institute{University of Milan and INFN Milan}
\titlegraphic{\vspace*{6mm}
    \includegraphics[height=0.8cm]{logos/LOGO-ERC.jpg} \hspace{10mm}
	\includegraphics[height=0.8cm]{logos/n3pdflogo_noback.png} \hspace{10mm}
	\includegraphics[height=0.6cm]{logos/nnpdf_logo_official.pdf} \hspace{10mm}
	\includegraphics[height=0.8cm]{logos/Logo_Università_degli_Studi_di_Milano(not_mandatory).png}
	\includegraphics[height=0.8cm]{logos/INFN_logo.png}
    \vspace*{5mm} \\
	\centering{
	\fontsize{7.0pt}{0.0pt}\selectfont This project has received funding from the European Union’s Horizon 2020 \\
    \vspace*{-1mm}
	research and innovation programme under grant agreement No 740006.
	}
}




%================================== SLIDES ==================================

\begin{document}
{
\setbeamertemplate{headline}{} % remove headline from titlepage
\begin{frame}
  \titlepage
\end{frame}
}



\section{The hopscotch scan}

\begin{frame}[t]{The hopscotch scan}{From \url{https://arxiv.org/pdf/2205.10444.pdf}}
  Hopscotch replicas are generated by sampling along Hessian eigenvector directions that result in large changes to $\sigma_H$ and $\sigma_Z$
  \begin{figure}
    \centering
    \includegraphics[width=.2\textwidth]{220510_t0_vs_exp_parabolas.pdf}
    \includegraphics[width=.3\textwidth]{parabola_zoomin.png}
    \hspace*{1em}
    \includegraphics[width=.3\textwidth]{220512_scatter_HZ_exp_t0.pdf}
    \caption*{The paper contains two figures that consider the $t_0$ prescription, Fig. 6 (left/middle) and Fig. 7 (right)}
  \end{figure}
\end{frame}

\begin{frame}[t]{Where are the negative $\Delta\chi^2_{t_0}$ solutions?}
  The ``outside'' replicas are defined by drawing a rectangular box around the NNPDF4.0 predictions.\\
  \vspace*{0.5em}
  All outside replicas are defined by larger $\sigma_H$
  \begin{columns}
    \begin{column}{0.48\textwidth}
      \vspace*{-1mm}
      \includegraphics[width=.9\textwidth]{NNPDF40.pdf}
    \end{column}
    \hspace*{0.01\textwidth}
    \begin{column}{0.48\textwidth}
      \includegraphics[width=.7\textwidth]{220512_scatter_HZ_exp_t0.pdf}
    \end{column}
  \end{columns}
\end{frame}

\begin{frame}[t]{The LO differential cross-section}
  Most of the discrepancy is in the low rapidity $y$ region\\
  \vspace*{0.5em}
  discrepancy is there at LO, suggesting a shift of the $gg$ luminosity
  \begin{center}
    \includegraphics[width=.48\textwidth]{differential_Higgs.pdf}
    \includegraphics[width=.48\textwidth]{differential_Higgs_Leading_Order.pdf}
  \end{center}
\end{frame}

\begin{frame}[t]{What PDF feature dominates the discrepancy?}
  Around the Higgs mass the luminosities of the hopscotch outliers are outside the region that has been sampled by the NNPDF4.0 1000 replica PDF\\
  \begin{center}
    \includegraphics[width=.48\textwidth]{plot_lumi1d_replicas.pdf}
  \end{center}
\end{frame}


\begin{frame}[t]{What PDF feature dominates the discrepancy?}{\href{https://vp.nnpdf.science/OMu9K6ElThiPGrsQKPNt7A==/}{\underline{report} comparing NNPDF4.0 to CT outliers}}
  There is a clear kink in the hopscotch gluon PDFs, does this correspond to a deterioration of the PDFs?
  \begin{columns}
    \begin{column}{.38\textwidth}
      \includegraphics[width=4cm]{plot_pdfreplicas_g.pdf}\\
      \includegraphics[width=4cm]{zoomin.png}
    \end{column}
    \begin{column}{.6\textwidth}
      \begin{tabular}{lr|r|r|r}
        \toprule
        group &  n &  NNPDF 40 $\chi^{2}$ &  CTEQ $\chi^{2}$ \\
        \midrule
        DIS NC &        2100 &    1.218623 &  1.210005 \\
        DIS CC &         989 &    0.893750 &  0.881195 \\
        \color{blue} DY &         893 &   \color{blue} 1.261028 & \color{blue} 1.226119 \\
        \color{red} TOP &          66 &   \color{red}  1.210918 & \color{red}  1.312839 \\
        \color{amethyst} JETS &         356 &     \color{amethyst} 0.943735 &  \color{amethyst} 0.879847 \\
        DIJET &         144 &    2.007296 &  1.991372 \\
        PHOTON &          53 &    0.763523 &  0.787850 \\
        SINGLETOP &          17 &    0.363610 &  0.352023 \\
        \bottomrule
      \end{tabular}
    \end{column}
  \end{columns}
\end{frame}



\section{Study so far}

\begin{frame}[t]{Can we reproduce the CT results?}{\url{https://vp.nnpdf.science/LNz_b5jcQvScphQYtgmPjQ==/}}
      A fit to level-0 closure data with a CT `outside' replica as input.\\
      \includegraphics[height=.65\textheight]{level-0_closure_data.pdf}
      \includegraphics[height=.65\textheight]{level-0_gluon.pdf}
\end{frame}

\begin{frame}[t]{`everything' seems to find the CT solutions}
  \vspace*{-1.5em}
  \begin{columns}
    \begin{column}{.28\textwidth}
      Only the full NNPDF4.0 dataset together seems unable to find the CT solutions
    \end{column}
    \begin{column}{.7\textwidth}
      \centering
      \includegraphics[height=.45\textheight]{DIS_only.pdf}
      \includegraphics[height=.45\textheight]{without_jets.pdf}\\
      \includegraphics[height=.45\textheight]{no_t0_fit.pdf}
      \includegraphics[height=.45\textheight]{reweight_bundle.pdf}
    \end{column}
  \end{columns}
\end{frame}

\begin{frame}[t]{training/validation, overfitting}
  Remember that the \chitwo improved for some processes while it deteriorated for others, one possible explanation for this could be that the early-stopping prevents these solutions.\\\vspace*{.5em}
  However, this is not the case, since even without a training/validation split, the CT solutions are not found:
  \begin{figure}
    \centering
    \includegraphics[height=.55\textheight]{without_trvl.pdf}
  \end{figure}
\end{frame}

\begin{frame}[t]{Does $\chi^2$ deteriorate away from the center?}
  Looking at the \chitwo distribution, low values of \chitwo are not restricted to the CT region
  \begin{figure}
    \centering
    \includegraphics[height=.55\textheight]{nnpdf40_chi2_scatter.pdf}
    \includegraphics[height=.55\textheight]{chi2_fit_to_central_value.pdf}
  \end{figure}
\end{frame}

\begin{frame}[t]{Training/validation $\chi^2$ distribution}
  Training and validation losses seem uncorrelated with the value their position in the $\sigma_H,\sigma_Z$ plane
  \begin{figure}
    \centering
    \includegraphics[height=.55\textheight]{chi2_training_chi2.pdf}
    \includegraphics[height=.55\textheight]{chi2_validation_chi2.pdf}
  \end{figure}
\end{frame}


\begin{frame}[t]{The impact of physical constraints}
  While dataset variations seem to generate solutions in the CT region, this does not seem the case when the physical constraints are relaxed.
  \begin{figure}
    \centering
    \includegraphics[height=.55\textheight]{with_31_positivity.pdf}
    \includegraphics[height=.55\textheight]{without_integrability.pdf}
  \end{figure}
\end{frame}

\begin{frame}[t]{Methodology variations - training fraction}
  If we change the training fraction to 0.50, the training loss increases while the validation loss decreases\\\vspace*{.5em}
  There is no reason to think a fraction of 0.75 is better, and 0.50 is more conservative\\\vspace*{.5em}
  The lack of PDFs in the CT region (as result of a low likelihood?) may in part be understood as a result of the cross-validation regularization
  \begin{center}
    \includegraphics[height=.5\textheight]{0.50_training_fraction.pdf}
    \includegraphics[height=.5\textheight]{frac50_training_validation.pdf}
    \includegraphics[height=.5\textheight]{NNPDF40NNLO_plot_training_validation.pdf}
  \end{center}
\end{frame}


\begin{frame}[t]{Referee report-like plots, \chitwo per replica}
  \vspace*{-0.1cm}
  No surprises here: if the sample is large enough, we do find replicas with a \chitwo smaller than replica0\\
  \begin{columns}
    \begin{column}{0.5\textwidth}
      \begin{figure}
        \includegraphics[height=.3\textheight]{chi2_replicas_nnpdf40_1000.png}
        \caption*{\footnotesize NNPDF4.0 (1000 replicas)}
      \end{figure}
      \vspace*{-0.8cm}
      \begin{figure}
        \includegraphics[height=.3\textheight]{chi2_replicas_level1_fit.png}
        \caption*{\footnotesize  fit to experimental central values}
      \end{figure}
    \end{column}
    \begin{column}{0.5\textwidth}
      \begin{figure}
        \includegraphics[height=.3\textheight]{chi2_replicas_nnpdf40_3000.png}
        \caption*{\footnotesize  NNPDF4.0 (3000 replicas)}
      \end{figure}
      \vspace*{-0.8cm}
      \begin{figure}
        \includegraphics[height=.3\textheight]{chi2_replicas_0.5trvl.png}
        \caption*{\footnotesize  0.50 training fraction}
      \end{figure}
    \end{column}
  \end{columns}
\end{frame}


\section{How to proceed}

\begin{frame}[t]{Understanding what points we need to address}
  There are some critical problems in the understanding of the NNPDF methodology within the wider community as well as other criticisms. Namely, it (by some) believed that:
  \begin{itemize}
    \item $\chi^2$ is a measure of the likelihood
    \begin{itemize}
      \item so \chitwo must be minimum at the best fit
      \item and good \chitwo can't be far away
    \end{itemize}
    \item There must be a reproducible best solution
    \begin{itemize}
      \item There is no unique best solution (even for level-0 data)
    \end{itemize}
    \item NN has unclear assumptions as opposed to a fixed functional form
    \begin{itemize}
      \item Maybe the NN disfavors wiggles where there should be wiggles. The Hessian method avoids this through tolerance
    \end{itemize}
    \item Perhaps more?
  \end{itemize}
  Some of the skepticism seems to have been inspired by the basis independence check in the NNPDF4.0 paper
\end{frame}

\begin{frame}[t]{Organizing the rebuttal}
  So far we have tried to address these criticisms in talks and conversations (at conferences), but perhaps we should address it in a public document\\
  \vspace*{0.5em}
  Do we understand what arguments we would need to put in the rebuttal?\\
  \vspace*{0.5em}
  What results/checks will we need to do?\\
  \vspace*{2em}

  % \textbf{Publishing}\\
  % We don't want to be limited to a small number of pages (so no proceedings)\\
  % {\footnotesize\qquad P.S. deadline for ICHEP proceedings is 31 October}\\
  % \vspace*{0.5em}
  % Paper, arxiv preprint, note on website, \ldots. are all options\\
\end{frame}



\appendix

\section{Backup}
\begin{frame}{K-folding}
\begin{figure}[t]
  \centering
  \begin{tikzpicture}[node distance = 1.0cm]\small
    \node[roundtext, fill=green!30] (hyperopt) {\texttt{hyperopt}};
    \coordinate [above = 1.5cm of hyperopt] (abovehyperopt) {};

    \node[roundtext, right = 2.5cm of abovehyperopt] (xplain) {Generate new hyperparameter configuration};
    \draw[myarrow] (hyperopt) -- (abovehyperopt) -- (xplain);

    \coordinate [below = 1.85cm of xplain.west] (fold4v) {};
    \coordinate [below = 1.85cm of xplain.east] (fold1v) {};
    \coordinate (arrowcenter) at ($(fold4v)!0.5!(fold1v)$);
    \coordinate (fold3v) at ($(fold4v)!0.66!(arrowcenter)$);
    \coordinate (fold2v) at ($(arrowcenter)!0.33!(fold1v)$);

    \node [roundtext, fill=green!30, above = 0.33cm of arrowcenter] (fitto) {Fit to subset of folds};
    \draw[thick] (xplain) -- (fitto);

    \draw[thick] (fold4v) -- (fold1v);
    \draw[thick] (fitto) -- (arrowcenter);

    \node[roundtext, below = 0.4cm of fold4v] (fold4) {folds 1,2,3};
    \node[roundtext, below = 0.4cm of fold1v] (fold1) {folds 2,3,4};
    \node[roundtext, below = 0.4cm of fold3v] (fold3) {folds 1,2,4};
    \node[roundtext, below = 0.4cm of fold2v] (fold2) {folds 1,3,4};
    \draw[myarrow] (fold1v) -- (fold1);
    \draw[myarrow] (fold4v) -- (fold4);
    \draw[myarrow] (fold2v) -- (fold2);
    \draw[myarrow] (fold3v) -- (fold3);

    \node[roundtext, fill=green!30, below = 0.30cm of fold4] (chi24) {$\chi^{2}_{4}$};
    \node[roundtext, fill=green!30, below = 0.30cm of fold3] (chi23) {$\chi^{2}_{3}$};
    \node[roundtext, fill=green!30, below = 0.30cm of fold2] (chi22) {$\chi^{2}_{2}$};
    \node[roundtext, fill=green!30, below = 0.30cm of fold1] (chi21) {$\chi^{2}_{1}$};

    \draw[thick] (fold1) -- (chi21);
    \draw[thick] (fold2) -- (chi22);
    \draw[thick] (fold3) -- (chi23);
    \draw[thick] (fold4) -- (chi24);

    \coordinate [below = 0.3cm of chi24] (below4) {};
    \coordinate [below = 0.3cm of chi21] (below1) {};
    \coordinate [below = 0.3cm of chi22] (below2) {};
    \coordinate [below = 0.3cm of chi23] (below3) {};

    \draw[thick] (below1) -- (below4);
    \draw[thick] (chi24) -- (below4);
    \draw[thick] (chi23) -- (below3);
    \draw[thick] (chi22) -- (below2);
    \draw[thick] (chi21) -- (below1);

    \coordinate (belowcenter) at ($(below4)!0.5!(below1)$);
    \node[operations, below = 0.5cm of belowcenter] (loss) {$L = \frac{1}{4}\displaystyle\sum^{4}_{i}\chi^{2}_{i}$};
    \draw[myarrow] (belowcenter) -- (loss);
    \path let \p1 = (hyperopt), \p2 = (loss)
      in coordinate (lleft) at (\x1,\y2);

    \draw[myarrow] (loss) -- (lleft) -- (hyperopt);

  \end{tikzpicture}
\end{figure}
\end{frame}



\begin{frame}[t]{Self-correlation of PDF sets}
The PDF$_i$-PDF$_j$ correlation for a given flavour is defined as
$$corr_{i,j}(x)= \frac{\sum_{n=1}^{N_{rep}} (f_{i,n}(x) -  f_{i,0}(x) )(f_{j,n}(x) -  f_{j,0}(x)) }{\sqrt{\sum_{n=1}^{N_{rep}}(f_{i,n}(x) - f_{i,0}(x))^2} \sqrt{\sum_{n=1}^{N_{rep}}(f_{j,n}(x) -  f_{j,0}(x))^2}}$$
where $f_{i,n}(x)$ is a PDF replica, and $n=0$ corresponds to the central value of the PDF set. 

\end{frame}

\begin{frame}[t]{Correlated combination of PDFs}

When combining PDF sets $f_i$ in a correlated way, for each momentum fraction $x$ and flavour, the central value is calculated using
$$
\left\langle f\right\rangle_{comb}=\sum_{i=1}^{N_{sets}} w_{i} f_{i,0}
$$

and the variance using
$$
V_{comb}=\sum_{i, j=1}^{N_{sets}} w_{i} \sigma_{i j} w_{j}
$$

where $\sigma$ is the covariance matrix, $f_{0,i}$ is the central value of PDF set $i$, and $w_i$ are the weights 
$$
w_{i}=\frac{\sum_{j=1}^{N_{sets}}\left(\sigma^{-1}\right)_{i j}}{\sum_{k, l=1}^{N_{sets}}\left(\sigma^{-1}\right)_{k l}}
$$

\end{frame}

\end{document}
