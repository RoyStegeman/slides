\documentclass[aspectratio=169,9pt]{beamer}
\graphicspath{{figures/}} % Setting the graphicspath

% Theme settings
\usetheme{Madrid}
\usecolortheme{default}
\setbeamertemplate{navigation symbols}{}   % removes navigation symbols such as 'next page'
\setbeamertemplate{footline}{}             % remove line with name, date, page nr. 
\setbeamercolor*{frametitle}{bg=white}     % remove background from frametitle
\usepackage{caption}
% \captionsetup[figure]{labelformat=empty}% redefines the caption setup of the figures environment in the beamer class.
\setbeamersize{text margin left=20pt,text margin right=10pt}

\usefonttheme[onlymath]{serif} % makes beamer math look like article math


%======================= import packages =======================
\usepackage{pifont}       % Pi fonts (Digbats, symbol, etc.)
\usepackage{graphicx}     % More options for \includegraphics
\usepackage{tikz}
\usepackage{appendixnumberbeamer} % separate appendix numbering
\usepackage{booktabs}
\usepackage{hyperref}
\usepackage{tabularx}
\usepackage{slashbox} % for the slash in table
\usepackage{amsmath, nccmath}


%======================= page numbering =======================
\addtobeamertemplate{navigation symbols}{}{ \usebeamerfont{footline}
  \insertframenumber / \inserttotalframenumber \hspace*{2mm} \\ \vspace*{1mm} 
}


%=================================== colors ===================================
\definecolor{RoyBlue}{RGB}{22, 46, 69}
\definecolor{RoyGrey}{RGB}{64, 88, 128} 

\newcommand{\hlme}[1]{{\color{red}\bf #1}} % highlihgt me

\setbeamercolor{structure}{fg=RoyBlue} % itemize, enumerate, etc
\setbeamercolor{frametitle}{fg=RoyGrey}
 \setbeamercolor{section in head/foot}{bg=RoyBlue}


%======================= add progress dots to headline =======================
\setbeamertemplate{headline}{%
    \begin{beamercolorbox}[ht=4mm,dp=4mm]{section in head/foot}
        \insertnavigation{\paperwidth}
    \end{beamercolorbox}%
}%
\makeatother


%======================= add section title page =======================
\AtBeginSection[]{
  \begin{frame}
  \vfill
  \centering
    \usebeamerfont{title}\insertsectionhead\par%
  \vfill
  \end{frame}
}


%=================================== titlepage ===================================
\title{NNPDF4.0: Towards a high-precision Determination of the Proton Structure}
\date{EPS-HEP 2021, 26 July 2021}
\author{Roy Stegeman}
\institute{University of Milan and INFN Milan}
\titlegraphic{\vspace*{6mm}
    \includegraphics[height=0.8cm]{logos/LOGO-ERC.jpg} \hspace{10mm}
	\includegraphics[height=0.8cm]{logos/n3pdflogo_noback.png} \hspace{10mm}
	\includegraphics[height=0.6cm]{logos/nnpdf_logo_official.pdf} \hspace{10mm}
	\includegraphics[height=0.8cm]{logos/Logo_Università_degli_Studi_di_Milano(not_mandatory).png}
	\includegraphics[height=0.8cm]{logos/INFN_logo.png}
    \vspace*{5mm} \\
	\centering{ 
	\fontsize{7.0pt}{0.0pt}\selectfont This project has received funding from the European Union’s Horizon 2020 \\	
    \vspace*{-1mm}
	research and innovation programme under grant agreement No 740006.
	}
}


\definecolor{Red}{rgb}{1,0,0}
\definecolor{Green}{rgb}{0,1,0}
\definecolor{Blue}{rgb}{0,0,1}
\definecolor{Gray}{gray}{0.9}
\definecolor{springgreen}   {cmyk}{0.26, 0   , 0.76, 0   }
\definecolor{olivegreen}    {cmyk}{0.64, 0   , 0.95, 0.40}
\definecolor{emerald}       {cmyk}{1   , 0   , 0.50, 0   }
\definecolor{junglegreen}   {cmyk}{0.99, 0   , 0.52, 0   }
\definecolor{seagreen}      {cmyk}{0.69, 0   , 0.50, 0   }
\definecolor{green}         {cmyk}{1   , 0   , 1   , 0   }
\definecolor{forestgreen}   {cmyk}{0.91, 0   , 0.88, 0.12}
\definecolor{pinegreen}     {cmyk}{0.92, 0   , 0.59, 0.25}
\definecolor{sepia}         {cmyk}{0   , 0.83, 1   , 0.70}
\definecolor{cerulean}      {cmyk}{0.94, 0.11, 0   , 0   }
\definecolor{salmon}        {cmyk}{0   , 0.53, 0.38, 0   }
\definecolor{greenyellow}   {cmyk}{0.15, 0   , 0.69, 0   }
\definecolor{arsenic}       {rgb}{0.23, 0.27, 0.29}
\definecolor{britishracinggreen}{rgb}{0.0, 0.26, 0.15}
\definecolor{oxfordblue}{rgb}{0.0, 0.13, 0.28}
\definecolor{bostonuniversityred}{rgb}{0.8, 0.0, 0.0}
\definecolor{goldenyellow}{rgb}{1.0, 0.87, 0.0}

\definecolor{darkgreen}{rgb}{0.0, 0.5, 0.13}
\definecolor{darkred}{rgb}{0.55, 0.0, 0.0}
\newcommand{\gct}{\color{darkgreen}\checkmark}
\newcommand{\rma}{\color{red}\ding{55}}
\newcommand{\bct}{\color{blue}\checkmark}
\newcommand{\arrowdownunder}{\begin{center}$\big\downarrow$\end{center}\vspace{-0.3cm}}
\newcommand{\mycolutitle}[1]{\vspace{-0.7cm}\begin{center}#1\end{center}\vspace{-0.1cm}}



\begin{document}
{
\setbeamertemplate{headline}{} % remove headline from titlepage
\begin{frame}
  \titlepage
\end{frame}
}



\section*{Towards NNPDF4.0}



%\begin{frame}{PDFs as a ML problem: the NNPDF approach}
%    Obtain PDFs by fitting Neural Networks to experimental data\\
%    Monte Carlo sample of functions $\rightarrow$ probability density in function space
%    
%    \begin{center}
%        \includegraphics[width=0.5\textwidth]{NNPDF_MC_strategy}
%    \end{center}
%\end{frame}

\begin{frame}[t]{High-precision: gluon}
	\begin{equation*}
	\mathcal{L}_{i j}\left(M_{X}, y, \sqrt{s}\right)
	=\frac{1}{s} \sum_{i, j} f_{i}\left(\frac{M_{X} e^{y}}{\sqrt{s}}, M_{X}\right) f_{j}\left(\frac{M_{X} e^{-y}}{\sqrt{s}}, M_{X}\right)
	\end{equation*}
	\includegraphics[width=0.45\textwidth]{plot_lumi2d_uncertainty_NNPDF31_gg}
	\includegraphics[width=0.45\textwidth]{plot_lumi2d_uncertainty_NNPDF40_gg}
    \begin{center}
	    \textbf{How did we get here?}
	\end{center}
\end{frame}

\begin{frame}[t]{High-precision: singlet }
	\begin{equation*}
	\mathcal{L}_{i j}\left(M_{X}, y, \sqrt{s}\right)
	=\frac{1}{s} \sum_{i, j} f_{i}\left(\frac{M_{X} e^{y}}{\sqrt{s}}, M_{X}\right) f_{j}\left(\frac{M_{X} e^{-y}}{\sqrt{s}}, M_{X}\right)
	\end{equation*}
	\includegraphics[width=0.45\textwidth]{plot_lumi2d_uncertainty_NNPDF31_qq}
	\includegraphics[width=0.45\textwidth]{plot_lumi2d_uncertainty_NNPDF40_qq}\\
	\begin{center}
	    \textbf{How did we get here?}
	\end{center}
\end{frame}


\begin{frame}
 \frametitle{The path to NNPDF4.0}
 \begin{columns}
    \column{0.70\linewidth}
	 \footnotesize
	 \begin{block}{}
	  \centering
	  Progress towards extending {\textcolor{red}{data}}, {\textcolor{blue}{theory}} and {\textcolor{forestgreen}{methodology}}\\
	 \end{block}
	 \scriptsize
	 \renewcommand*{\arraystretch}{1.4}
	 \begin{tabularx}{\textwidth}{lXr}
	  06/2017 & {\bf NNPDF3.1}                                                              
	          & {\tiny{[{\textcolor{salmon}{EPJ\,C77\,(2017)\,663}}]}}\\
	  10/2017 & \textcolor{blue}{NNPDF3.1sx}: {\scriptsize PDFs with small-$x$ resummation}                                                
	          & {\tiny{[{\textcolor{salmon}{EPJ\,C78\,(2018)\,321}}]}}\\
	  12/2017 & \textcolor{blue}{NNPDF3.1luxQED}: {\scriptsize consistent photon PDF \`a la luxQED}                                            
	          & {\tiny{[{\textcolor{salmon}{SciPost\,Phys.\,5\,(2018)\,008}}]}}\\
	  02/2018 & \textcolor{red}{NNPDF3.1+ATLASphoton}: {\scriptsize inclusion of direct photon data}                                       
	          & {\tiny{[{\textcolor{salmon}{EPJ\,C78\,(2018)\,470}}]}}\\
	  12/2018 & \textcolor{forestgreen}{NNPDF3.1alphas}: {\scriptsize $\alpha_s$ from a correlated-replica method}                                     
	          & {\tiny{[{\textcolor{salmon}{EPJ\,C78\,(2018)\,408}}]}}\\
	  12/2018 & \textcolor{forestgreen}{NNPDF3.1nuc}: {\scriptsize heavy ion nuclear uncertainties in a fit}                                        
	          & {\tiny{[{\textcolor{salmon}{EPJ\,C79\,(2019)\,282}}]}}\\
	  05/2019 & \textcolor{forestgreen}{NNPDF3.1th}: {\scriptsize missing higher-order uncertainties in a fit}                                         
	          & {\tiny{[{\textcolor{salmon}{EPJ\,C79\,(2019)\,838; ibid.\,931}}]}}\\
	  07/2019 & \textcolor{forestgreen}{Gradient descent and hyperoptimisation in PDF fits} 
	          & {\tiny{[{\textcolor{salmon}{EPJ\,C79\,(2019)\,676}}]}}\\
	  12/2019 & \textcolor{red}{NNPDF3.1singletop}: {\scriptsize inclusion of single top $t$-channel data}                                          
	          & {\tiny{[{\textcolor{salmon}{JHEP\,05\,(2020)\,067}}]}}\\
	  05/2020 & \textcolor{red}{NNPDF3.1dijets}: {\scriptsize comparative study of single- and di-jets}                                             
	          & {\tiny{[{\textcolor{salmon}{EPJ\,C80\,(2020)\,797}}]}}\\
	  06/2020 & \textcolor{blue}{Positivity of $\overline{\rm MS}$ PDFs}                    
	          & {\tiny{[{\textcolor{salmon}{JHEP\,11\,(2020)\,129}}]}}\\
	  08/2020 & \textcolor{forestgreen}{PineAPPL}: {\scriptsize fast evaluation of EW$\times$QCD corrections}                                           
	          & {\tiny{[{\textcolor{salmon}{JHEP\,12\,(2020)\,108}}]}}\\
	  08/2020 & \textcolor{red}{NNPDF3.1strangeness}: {\scriptsize assessment of strange-sensitive data}                                        
	          & {\tiny{[{\textcolor{salmon}{EPJ\,C80\,(2020)\,1168}}]}}\\
	  11/2020 & \textcolor{forestgreen}{NNPDF3.1deu}: {\scriptsize deuteron uncertainties in a fit}                                        
	          & {\tiny{[{\textcolor{salmon}{EPJ\,C81\,(2021)\,37}}]}}\\
	  03/2021 & \textcolor{forestgreen}{Future tests}                                       
	          & {\tiny{[{\textcolor{salmon}{arXiv:2103.08606}}]}}\\
	  2021    & {\bf NNPDF4.0}                                                              
	          & {\tiny{[{\textcolor{salmon}{to appear}}]}}\\
	 \end{tabularx}
    \end{columns}
\end{frame}


\section*{Dataset}



\begin{frame}{Experimental data in NNPDF4.0}
	\begin{columns}
	    \column{0.48\linewidth}
	        {\footnotesize
	        \begin{itemize}
	            \item $\mathcal{O}(35)$ datasets investigated
	            \item $\mathcal{O}(400)$ more data points in NNPDF4.0 \\ than in NNPDF3.1
	            \item New data is mostly from the LHC RUN II
	        \end{itemize}
	        }
	        \includegraphics[width=0.9\textwidth]{atlas_data_table}
	
	    \column{0.48\linewidth}
	        \includegraphics[width=0.9\textwidth]{cms_data_table} \\
	        \includegraphics[width=0.9\textwidth]{lhcb_data_table}
	\end{columns}
\end{frame}


\begin{frame}{Experimental data in NNPDF4.0}
    \begin{columns}
        \column{0.7\linewidth}
            \includegraphics[width=1.0\textwidth]{Markers0_plot_xq2}
        \column{0.25\linewidth}
            New processes:
            \begin{itemize}
                \item direct photon
                \item single top
                \item dijets
                \item W+jet
                \item DIS jet
            \end{itemize}
    \begin{block}{\footnotesize Theoretical improvement}
    {\footnotesize
    Nuclear uncertainties are included
    }
    \end{block}
    \end{columns}
\end{frame}




\section*{Methodology}




%\begin{frame}[t]
%    \frametitle{Improved fitting methodology}
%    \footnotesize
%    \begin{columns}[t]
%        \column{0.5\linewidth}
%            \mycolutitle{NNPDF 3.1 code}
%            \begin{list}{\color{darkred} $\rightarrow$}{}  
%                \item {\bf Fit parameters manually chosen }
%                \item {\bf Fitting times of up to various days}
%                \item { Genetic Algorithm optimizer}
%                \item One network per flavour
%                \item Physical constraints imposed independently of optimization
%                \item Preprocessing fixed per each of the replicas
%                \item C++ monolithic codebase
%                \item In-house Machine Learning optimization framework
%            \end{list}
%        \column{0.5\linewidth}
%            \mycolutitle{NNPDF 4.0 code}
%            \begin{list}{\color{darkgreen} $\rightarrow$}{}
%                \item {\bf Fit parameters chosen automatically (hyperparameter scan)}
%                \item {\bf Results available in less than an hour}
%                \item {Gradient Descent optimization}
%                \item One network for all flavours
%                \item Physical constraints integrated in the optimization
%                \item Preprocessing can be fitted within replicas
%                \item Python object oriented codebase
%                \item Freedom to use external libraries (default: TensorFlow)
%            \end{list}
%    \end{columns}
%\end{frame}


\begin{frame}[t]{Improved fitting methodology}
    \begin{columns}[T]
        \begin{column}{0.48\textwidth}
            \begin{itemize}
                \item \textbf{Stochastic Gradient Descent} for NN training using TensorFlow
                \item Automated optimization of \\ \textbf{ model hyperparameters}
                \item Methodology is validated using 
                {\bf closure tests} (data region), {\bf future tests} (extrapolation region), and {\bf parametrization basis independence}
            \end{itemize}
        \vspace*{1em}
        Physical constraints:
        \begin{itemize}
            \item PDF positivity {\footnotesize{{\textcolor{blue}{[JHEP\,11\,(2020)\,129]}}}}
            \item Integrability of nonsinglet distributions (Gottfried sum rules)
        \end{itemize}
        \end{column}
        \begin{column}{0.48\textwidth}
            \vspace*{-3em}
            \includegraphics[width=1.0\textwidth]{NNarch}
            \begin{equation*}
                f_{i}\left(x, Q_{0}\right)=x^{-\alpha_{i}}(1-x)^{\beta_{i}} \mathrm{NN}_{i}(x)
            \end{equation*}
            \vspace*{-1em}
            \begin{block}{}
                \textbf{Different strategies} to parametrize the quark PDF flavour combinations lead to \textbf{identical results}
            \end{block}
        \end{column}
    \end{columns}
\end{frame}


%\begin{frame}{Parametrization basis independence}
%     \begin{columns}
%         \begin{column}[T]{0.48\textwidth}
%         \vspace*{0pt}%
% 	        \begin{center}
% 	            \includegraphics[width=0.8\textwidth]{flavour_evolution_V} \\
% 	        \end{center}
%         \end{column}
%         \begin{column}[t]{0.48\textwidth}
%         \vspace{0pt}%
% 	        \begin{center}
% 	            \includegraphics[width=0.8\textwidth]{flavour_evolution_T3} \\
% 	        \end{center}
%         \end{column}
%     \end{columns}
%     \begin{columns}
%         \column{0.4\linewidth}
% 		    Evolution Basis:
% 		    {\footnotesize
% 		    \begin{fleqn}
% 		    \begin{align*}
% 		       \qquad x V\left(x, Q_{0}\right) &\propto \mathrm{NN}_{V}(x)\\
% 		        x T_{3}\left(x, Q_{0}\right) &\propto \mathrm{NN}_{T_{3}}(x)
% 		    \end{align*}
% 		    \end{fleqn}
% 		    }
%         \column{0.55\linewidth}
%             \begin{block}{}
%                 \textbf{Different strategies} to parametrize the quark PDF flavour combinations lead to \textbf{identical results}
%             \end{block}
%     \end{columns}
%     \vspace*{-0.5em}
%     Flavour Basis:
%     {\footnotesize
%     \begin{fleqn}
%     \begin{align*}
%         \qquad x V\left(x, Q_{0}\right) &\propto\left(\mathrm{NN}_{u}(x)-\mathrm{NN}_{\bar{u}}(x)+\mathrm{NN}_{d}(x)-\mathrm{NN}_{\bar{d}}(x)+\mathrm{NN}_{s}(x)-\mathrm{NN}_{\bar{s}}(x)\right) \\
%         x T_{3}\left(x, Q_{0}\right) &\propto\left(\mathrm{NN}_{u}(x)+\mathrm{NN}_{\bar{u}}(x)-\mathrm{NN}_{d}(x)-\mathrm{NN}_{\bar{d}}(x)\right)
%     \end{align*}
%     \end{fleqn}
%     }
% \end{frame}



\begin{frame}[t]{Automated model selection}
	NNPDF aims to minimize sources of bias in the PDF:
	\begin{itemize}
	    \item Functional form $\rightarrow$ Neural Network
	    \item Model parameters $\rightarrow$ ?
	\end{itemize}
\end{frame}


\begin{frame}[t]{Automated model selection}
	NNPDF aims to minimize sources of bias in the PDF:
	\begin{itemize}
	    \item Functional form $\rightarrow$ Neural Network
	    \item Model parameters $\rightarrow$ \textbf{Hyperoptimization}
	\end{itemize}
	\begin{center}
	    \includegraphics[width=0.9\textwidth]{hyperopt_scan}
	\end{center}
    Scan over thousands of hyperparameter combinations and select the best one \\
    {\bf k-fold cross-validation}: used to define the reward function based on a {\bf test dataset}
\end{frame}


%\begin{frame}[t]{Hyperoptimization: the reward function}
%    \begin{columns}[T]
%        \begin{column}{0.48\textwidth}
%            \vspace{\topsep}
%            Choosing as the hyperoptimization target the $\chi^2$ of fitted data results in overfitting.
%        \end{column}
%        \begin{column}{0.48\textwidth}
%            \includegraphics[width=0.9\textwidth]{overfit_nnpdf31}
%        \end{column}
%    \end{columns}
%\end{frame}



%\begin{frame}[t]{Hyperoptimization: the reward function}
%    \begin{columns}[T]
%        \begin{column}{0.48\textwidth}
%            \vspace{\topsep}
%            Choosing as the hyperoptimization target the $\chi^2$ of fitted data results in overfitting.\\
%			\vspace*{2em}			
%			We solve this using \textbf{k-fold cross-validation}:
%			\begin{enumerate}
%			    \item Divide the data into $k$ {representative subsets}
%			    \item Fit $k-1$ sets and use $k$-th as test set
%			    \begin{itemize}
%			        \item[$\Rightarrow$] $k$ values of $\chi^2_\mathrm{test}$
%			    \end{itemize}
%			    \item Optimize the average $\chi^2_\mathrm{test}$ of the $k$ test sets
%			\end{enumerate}
%			\vspace*{0.5em}
%			$\Rightarrow$ The hyperoptimization target is not based on data that entered the fit. 
%        \end{column}
%        \begin{column}{0.48\textwidth}
%            \includegraphics[width=0.9\textwidth]{best_model_vs_nnpdf31}
%            \begin{itemize}
%                \item No overfitting\\
%			    \vspace*{0.2em}
%			    \item Compared to NNPDF3.1:
%			    \begin{itemize}
%			        \item Increased stability
%			        \item Reduced uncertainties 
%			    \end{itemize}
%			\end{itemize}
%        \end{column}
%    \end{columns}
%\end{frame}





\section*{PDFs and Phenomenology}




%\begin{frame}
% \frametitle{Impact of the new data and fitting methodology}
% \footnotesize
% \centering
% \begin{columns}[c]
%  \begin{column}{0.5\textwidth}
%   \begin{overlayarea}{\columnwidth}{4cm}
%    \only<1>
%    {
%     \centering
%     \includegraphics[width=0.65\columnwidth]{lumi1d_gg_NNPDF31_NNPDF40}\\    
%    }
%    \only<2>
%    {
%     \centering
%     \includegraphics[width=0.65\columnwidth]{lumi1d_gg_NNPDF31meth_NNPDF40}\\    
%    }
%   \end{overlayarea}
%  \end{column}
%  \begin{column}{0.5\textwidth}
%   \begin{overlayarea}{\columnwidth}{4cm}
%    \vspace*{-0.6cm}    
%    \only<1>
%    {
%     \centering
%     \includegraphics[width=0.65\columnwidth]{lumi1d_qq_NNPDF31_NNPDF40}\\    
%    }
%    \only<2>
%    {
%     \centering
%     \includegraphics[width=0.65\columnwidth]{lumi1d_qq_NNPDF31meth_NNPDF40}\\    
%    }
%   \end{overlayarea}   
%  \end{column} 
% \end{columns}
% \begin{columns}[c]
%  \begin{column}{0.5\textwidth}
%   \begin{overlayarea}{\columnwidth}{4cm}
%    \vspace*{-0.55cm}
%    \only<1>
%    {
%     \centering
%     \includegraphics[width=0.65\columnwidth]{lumi1d_qqb_NNPDF31_NNPDF40}\\
%    }
%    \only<2>
%    {
%     \centering
%     \includegraphics[width=0.65\columnwidth]{lumi1d_qqb_NNPDF31meth_NNPDF40}\\    
%    }
%   \end{overlayarea}   
%  \end{column}
%  \begin{column}{0.5\textwidth}
%   \begin{overlayarea}{\columnwidth}{4cm}
%    \vspace*{-0.8cm}    
%    \only<1>
%    {
%     \centering
%     \tiny
%     \renewcommand*{\arraystretch}{1.35}
%     \begin{tabular}{lcc}
%      \toprule
%       \backslashbox{data set ($N_{\rm dat}$)}{methodology} & NNPDF3.1         & NNPDF4.0        \\
%       \midrule
%       NNPDF3.1 (4093)                                     & \alert{\bf 1.19} &            1.11  \\
%       NNPDF4.0 (4618)                                     &      1.25        & \alert{\bf 1.16} \\
%      \bottomrule
%     \end{tabular}\\
%     \vspace{0.3cm}
%%     \scriptsize
%%     \underline{Consistency} between PDF sets\\
%%     \vspace{0.2cm}
%%     NNPDF4.0 \underline{more precise}\\
%%     (combination of data set and methodology)\\
%%     \vspace{0.2cm}
%%     NNPDF4.0 \underline{more accurate}\\
%%     (superiority of the NNPDF4.0 methodology)\\
%    }
%    \only<2>
%    {
%      \centering
%     \tiny
%     \renewcommand*{\arraystretch}{1.35}
%     \begin{tabular}{lcc}
%      \toprule
%       \backslashbox{data set ($N_{\rm dat}$)}{methodology} & NNPDF3.1         & NNPDF4.0        \\
%       \midrule
%       NNPDF3.1 (4093)                                      &            1.19  &            1.11  \\
%       NNPDF4.0 (4618)                                      & \alert{\bf 1.25} & \alert{\bf 1.16} \\
%      \bottomrule
%     \end{tabular}\\
%     \vspace{0.3cm}
%%     \scriptsize
%%     \underline{Consistency} between PDF sets\\
%%     \vspace{0.2cm}
%%     NNPDF4.0 \underline{more precise}\\
%%     (combination of data set and methodology)\\
%%     \vspace{0.2cm}
%%     NNPDF4.0 \underline{more accurate}\\
%%     (superiority of the NNPDF4.0 methodology)\\  
%      {Moderate reduction of PDF ucertainties \\
%      shifts in central values at the one-sigma level}
%    }
%   \end{overlayarea}   
%  \end{column}   
% \end{columns}
%\end{frame}


\begin{frame}[t]{Impact of the new data}
%    \begin{center}
%        Consistency between data sets \\
%    \end{center}
	\includegraphics[width=0.45\textwidth]{lumi1d_gg_NNPDF40meth_NNPDF31data}
	\includegraphics[width=0.45\textwidth]{lumi1d_qq_NNPDF40meth_NNPDF31data}
	Individual datasets have a limited impact, but collectively they result in:
	\begin{itemize}
	    \item Moderate reduction of PDF uncertainties
	    \item Shifts in central value at the one-sigma level
	\end{itemize}
\end{frame}


\begin{frame}[t]{Impact of the new fitting methodology}
%    \begin{center}
%        Consistency between methodologies \\
%    \end{center}
	\includegraphics[width=0.45\textwidth]{lumi1d_gg_NNPDF31meth_NNPDF40data}
	\includegraphics[width=0.45\textwidth]{lumi1d_qq_NNPDF31meth_NNPDF40data}
	\begin{columns}
	    \column{0.45\linewidth}
			\begin{itemize}
	    	        \item Significant reduction of PDF uncertainties
		        \item Good agreement between the central values
		    \end{itemize}
        \column{0.5\linewidth}
            \begin{block}{}
                \fontsize{7}{6}\selectfont
                PDF uncertainties are validated using closure tests and future tests\\
                Validation tests successful for both NNPDF4.0 and NNPDF3.1 
            \end{block}
    \end{columns}
\end{frame}



\begin{frame}[t]{Implications for phenomenology}
    \begin{center}
        Reduced luminosity uncertainties $\rightarrow$ Reduced uncertainty at the level of observables\\
        \vspace*{-0.5em}
        \begin{columns}
	        \begin{column}{0.48\textwidth}
	            \begin{center}
	                \includegraphics[width=0.78\textwidth]{NNPDF_TTB_14TEV_40_PHENO-internal} 
	            \end{center}
	        \end{column}
	        \begin{column}{0.48\textwidth}
	            \includegraphics[width=0.7\textwidth]{NNPDF_H_14TEV_40_PHENO-integrated}\\
        	        \includegraphics[width=0.7\textwidth]{NNPDF_TTB_14TEV_40_PHENO-integrated}
	        \end{column}
        \end{columns}
    \end{center}
\end{frame}

%\begin{frame}{Implications for phenomenology}
%    \begin{center}
%        Reduced luminosity uncertainties $\rightarrow$ Reduced uncertainty at the observable level
%        \includegraphics[width=0.55\textwidth]{pheno_h}
%    \end{center}
%\end{frame}




\section*{Conclusions}



\begin{frame}[t]{Summary}
    \begin{itemize}
        \item Added $\mathcal{O}(400)$ new data points from many new processes
        \item Improved methodology with Stochastic Gradient Descent and hyperoptimization
        \item Validation of PDF uncertainties using closure test, future test and parametrization basis independence
        \item[$\Rightarrow$] NNPDF4.0 achieves a high precision over a broad kinematic range
    \end{itemize}
    \vspace*{2em}
    \begin{block}{}
        \centering
        The {\bf NNPDF code} will be made {\bf publicly available} along with user-friendly documentation
    \end{block}
    \vspace*{2em}
    \only<2>{
    \begin{center}
        {\vspace*{2em} \Large \textbf{Thank you!}}
    \end{center}
    }
\end{frame}



\appendix

\section{Backup}
\begin{frame}{K-folding}
\begin{figure}[t]
  \centering
  \begin{tikzpicture}[node distance = 1.0cm]\small
    \node[roundtext, fill=green!30] (hyperopt) {\texttt{hyperopt}};
    \coordinate [above = 1.5cm of hyperopt] (abovehyperopt) {};

    \node[roundtext, right = 2.5cm of abovehyperopt] (xplain) {Generate new hyperparameter configuration};
    \draw[myarrow] (hyperopt) -- (abovehyperopt) -- (xplain);

    \coordinate [below = 1.85cm of xplain.west] (fold4v) {};
    \coordinate [below = 1.85cm of xplain.east] (fold1v) {};
    \coordinate (arrowcenter) at ($(fold4v)!0.5!(fold1v)$);
    \coordinate (fold3v) at ($(fold4v)!0.66!(arrowcenter)$);
    \coordinate (fold2v) at ($(arrowcenter)!0.33!(fold1v)$);

    \node [roundtext, fill=green!30, above = 0.33cm of arrowcenter] (fitto) {Fit to subset of folds};
    \draw[thick] (xplain) -- (fitto);

    \draw[thick] (fold4v) -- (fold1v);
    \draw[thick] (fitto) -- (arrowcenter);

    \node[roundtext, below = 0.4cm of fold4v] (fold4) {folds 1,2,3};
    \node[roundtext, below = 0.4cm of fold1v] (fold1) {folds 2,3,4};
    \node[roundtext, below = 0.4cm of fold3v] (fold3) {folds 1,2,4};
    \node[roundtext, below = 0.4cm of fold2v] (fold2) {folds 1,3,4};
    \draw[myarrow] (fold1v) -- (fold1);
    \draw[myarrow] (fold4v) -- (fold4);
    \draw[myarrow] (fold2v) -- (fold2);
    \draw[myarrow] (fold3v) -- (fold3);

    \node[roundtext, fill=green!30, below = 0.30cm of fold4] (chi24) {$\chi^{2}_{4}$};
    \node[roundtext, fill=green!30, below = 0.30cm of fold3] (chi23) {$\chi^{2}_{3}$};
    \node[roundtext, fill=green!30, below = 0.30cm of fold2] (chi22) {$\chi^{2}_{2}$};
    \node[roundtext, fill=green!30, below = 0.30cm of fold1] (chi21) {$\chi^{2}_{1}$};

    \draw[thick] (fold1) -- (chi21);
    \draw[thick] (fold2) -- (chi22);
    \draw[thick] (fold3) -- (chi23);
    \draw[thick] (fold4) -- (chi24);

    \coordinate [below = 0.3cm of chi24] (below4) {};
    \coordinate [below = 0.3cm of chi21] (below1) {};
    \coordinate [below = 0.3cm of chi22] (below2) {};
    \coordinate [below = 0.3cm of chi23] (below3) {};

    \draw[thick] (below1) -- (below4);
    \draw[thick] (chi24) -- (below4);
    \draw[thick] (chi23) -- (below3);
    \draw[thick] (chi22) -- (below2);
    \draw[thick] (chi21) -- (below1);

    \coordinate (belowcenter) at ($(below4)!0.5!(below1)$);
    \node[operations, below = 0.5cm of belowcenter] (loss) {$L = \frac{1}{4}\displaystyle\sum^{4}_{i}\chi^{2}_{i}$};
    \draw[myarrow] (belowcenter) -- (loss);
    \path let \p1 = (hyperopt), \p2 = (loss)
      in coordinate (lleft) at (\x1,\y2);

    \draw[myarrow] (loss) -- (lleft) -- (hyperopt);

  \end{tikzpicture}
\end{figure}
\end{frame}



\begin{frame}[t]{Self-correlation of PDF sets}
The PDF$_i$-PDF$_j$ correlation for a given flavour is defined as
$$corr_{i,j}(x)= \frac{\sum_{n=1}^{N_{rep}} (f_{i,n}(x) -  f_{i,0}(x) )(f_{j,n}(x) -  f_{j,0}(x)) }{\sqrt{\sum_{n=1}^{N_{rep}}(f_{i,n}(x) - f_{i,0}(x))^2} \sqrt{\sum_{n=1}^{N_{rep}}(f_{j,n}(x) -  f_{j,0}(x))^2}}$$
where $f_{i,n}(x)$ is a PDF replica, and $n=0$ corresponds to the central value of the PDF set. 

\end{frame}

\begin{frame}[t]{Correlated combination of PDFs}

When combining PDF sets $f_i$ in a correlated way, for each momentum fraction $x$ and flavour, the central value is calculated using
$$
\left\langle f\right\rangle_{comb}=\sum_{i=1}^{N_{sets}} w_{i} f_{i,0}
$$

and the variance using
$$
V_{comb}=\sum_{i, j=1}^{N_{sets}} w_{i} \sigma_{i j} w_{j}
$$

where $\sigma$ is the covariance matrix, $f_{0,i}$ is the central value of PDF set $i$, and $w_i$ are the weights 
$$
w_{i}=\frac{\sum_{j=1}^{N_{sets}}\left(\sigma^{-1}\right)_{i j}}{\sum_{k, l=1}^{N_{sets}}\left(\sigma^{-1}\right)_{k l}}
$$

\end{frame}







\end{document}
